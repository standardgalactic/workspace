\documentclass[12pt]{article}

\usepackage[letterpaper,left=1.25in,right=1.25in,top=1in,bottom=1in]{geometry}
\usepackage{fontspec}
\usepackage{unicode-math}
\usepackage{microtype}
\usepackage{setspace}
\usepackage{amsmath,amssymb,amsthm,mathtools}
\usepackage{hyperref}
\usepackage{cleveref}
% \usepackage{physics}
\usepackage{tensor}
\usepackage{bm}

\setmainfont{TeX Gyre Pagella}
\setmathfont{TeX Gyre Pagella Math}

\setstretch{1.15}
\setlength{\parindent}{0pt}
\setlength{\parskip}{0.75em}

\title{Scalar Irruption via Entropic Differential:\\
A Non-Equilibrium Mechanism for Structure Formation in a Non-Expanding Plenum}
\author{Flyxion}
\date{\today}

\newtheorem{theorem}{Theorem}
\newtheorem{proposition}{Proposition}
\newtheorem{lemma}{Lemma}
\newtheorem{definition}{Definition}

\begin{document}

\maketitle

\begin{abstract}

This essay develops a rigorous account of scalar irruption as a non-equilibrium instability driven by entropy curvature within a non-expanding plenum. Rather than invoking metric expansion or vacuum symmetry breaking, scalar irruption is formulated as a threshold phenomenon in a coupled scalar--vector--entropy system. Beginning from a variational principle for a scalar field $\phi$ advected by a vector flow $\mathbf{v}$ and coupled to an entropy density $S$, we derive the governing evolution equations and identify the conditions under which diffusive smoothing transitions to exponential amplification. We prove that sufficiently negative entropy curvature induces local scalar growth through a sign change in the effective mass term, establishing scalar irruption as a bifurcation in the entropic geometry of the system. The mechanism is analyzed both in linearized stability theory and in a nonlinear energy functional framework, demonstrating that scalar irruption constitutes an entropically triggered phase transition in a non-expanding cosmological plenum.

\end{abstract}

\newpage
\section{Introduction}

Standard cosmological structure formation is typically modeled through gravitational instability in an expanding background geometry. In contrast, the present framework assumes a non-expanding plenum endowed with a scalar field $\phi$, a vector flow $\mathbf{v}$, and an entropy density $S$. The dynamics of these fields are governed not by metric expansion but by entropic smoothing and transport. Within this setting, the emergence of structure must be explained without recourse to inflationary vacuum potentials or background scale factors.

Scalar irruption is proposed as the fundamental mechanism of structure formation under these assumptions. Informally, it denotes the abrupt local amplification of a scalar field when entropy curvature crosses a critical threshold. The aim of this essay is to replace that informal statement with a mathematically precise derivation, beginning from an action principle and proceeding through stability analysis to a rigorous instability criterion.

We assume throughout that the spatial manifold $(\mathcal{M},g)$ is smooth, compact without boundary unless otherwise stated, and equipped with a Riemannian metric $g$. All fields are taken to be sufficiently regular for the variational calculus employed.

\section{Variational Formulation of the Coupled System}

Let $\phi : \mathcal{M} \times \mathbb{R} \to \mathbb{R}$ denote a scalar field, $\mathbf{v}$ a vector field on $\mathcal{M}$, and $S$ an entropy density functionally dependent on $\phi$ unless otherwise specified. We begin with an action functional of the form

\begin{equation}
\mathcal{A}[\phi,\mathbf{v}]
=
\int_{\mathbb{R}} \int_{\mathcal{M}}
\left(
\frac{1}{2} (\partial_t \phi)^2
- \frac{c^2}{2} \norm{\nabla \phi}^2
+ \mathbf{v} \cdot \nabla \phi
- U(\phi,S)
\right)
\, d\mu_g \, dt,
\label{eq:action}
\end{equation}

where $d\mu_g$ is the Riemannian volume form, $c$ is a characteristic smoothing scale, and $U(\phi,S)$ is an effective potential coupling $\phi$ to entropy.

We assume that entropy is defined by a functional of the scalar field,

\begin{equation}
S[\phi] = - \phi \log \phi,
\label{eq:entropy_density}
\end{equation}

interpreted pointwise. More generally, one may take

\begin{equation}
S[\phi] = - \int_{\mathcal{M}} \phi \log \phi \, d\mu_g,
\label{eq:entropy_global}
\end{equation}

but for local instability analysis the pointwise density is sufficient.

We consider a coupling of the form

\begin{equation}
U(\phi,S) = \frac{\alpha}{2} \phi^2 + \beta \phi \, \Delta_g S,
\label{eq:potential}
\end{equation}

where $\alpha$ and $\beta$ are constants and $\Delta_g$ is the Laplace--Beltrami operator.

The Euler--Lagrange equation for $\phi$ follows from

\begin{equation}
\frac{\delta \mathcal{A}}{\delta \phi} = 0.
\end{equation}

A straightforward variation yields

\begin{equation}
\partial_t^2 \phi
- c^2 \Delta_g \phi
+ \nabla \cdot \mathbf{v}
- \alpha \phi
- \beta \Delta_g S
= 0.
\label{eq:euler_lagrange}
\end{equation}

If we pass to a dissipative regime by introducing first-order temporal dynamics appropriate to smoothing processes, we consider instead the evolution equation

\begin{equation}
\partial_t \phi
=
c^2 \Delta_g \phi
- \nabla \cdot (\phi \mathbf{v})
+ \alpha \phi
+ \beta \phi \, \Delta_g S.
\label{eq:evolution}
\end{equation}

Equation \eqref{eq:evolution} will serve as the starting point for instability analysis. The final term, proportional to $\phi \Delta_g S$, encodes entropic curvature coupling. Scalar irruption will be shown to arise when this term dominates diffusive smoothing.

\section{Linear Stability Analysis and the Irruption Criterion}

We now consider a background configuration $\phi_0$ that is spatially uniform and stationary under \eqref{eq:evolution}. Assume $\mathbf{v}=0$ for simplicity in the local analysis. Then the equation reduces to

\begin{equation}
\partial_t \phi = c^2 \Delta_g \phi + \alpha \phi + \beta \phi \, \Delta_g S.
\end{equation}

Let $\phi = \phi_0 + \varepsilon \psi$, where $\varepsilon \ll 1$ and $\psi$ is a perturbation. Expanding $S[\phi]$ to first order yields

\begin{equation}
S[\phi] = S[\phi_0] + \varepsilon S'[\phi_0] \psi + O(\varepsilon^2).
\end{equation}

Since $S'[\phi_0] = - (1 + \log \phi_0)$, we compute

\begin{equation}
\Delta_g S \approx - (1 + \log \phi_0) \Delta_g \psi.
\end{equation}

Substituting into the linearized equation gives

\begin{equation}
\partial_t \psi
=
c^2 \Delta_g \psi
+ \alpha \psi
- \beta \phi_0 (1 + \log \phi_0) \Delta_g \psi.
\end{equation}

Collecting Laplacian terms,

\begin{equation}
\partial_t \psi
=
\left(
c^2 - \beta \phi_0 (1 + \log \phi_0)
\right) \Delta_g \psi
+ \alpha \psi.
\label{eq:linearized}
\end{equation}

Let $\psi_k$ be an eigenfunction of $\Delta_g$ with eigenvalue $- \lambda_k$, $\lambda_k \ge 0$. Then

\begin{equation}
\partial_t \psi_k
=
-
\left(
c^2 - \beta \phi_0 (1 + \log \phi_0)
\right)
\lambda_k \psi_k
+ \alpha \psi_k.
\end{equation}

The growth rate $\gamma_k$ is therefore

\begin{equation}
\gamma_k
=
-
\left(
c^2 - \beta \phi_0 (1 + \log \phi_0)
\right)
\lambda_k
+ \alpha.
\end{equation}

We obtain the following result.

\begin{theorem}[Scalar Irruption Criterion]
Suppose $\alpha \ge 0$. If there exists $k$ such that
\[
c^2 - \beta \phi_0 (1 + \log \phi_0) < 0,
\]
then for sufficiently large $\lambda_k$, the growth rate $\gamma_k$ becomes positive, and the uniform state $\phi_0$ is linearly unstable. This instability corresponds to scalar irruption.
\end{theorem}

\begin{proof}
If the coefficient of $\lambda_k$ is negative, then
\[
- \left( c^2 - \beta \phi_0 (1 + \log \phi_0) \right) \lambda_k
\]
is positive and grows unbounded as $\lambda_k \to \infty$. Since $\alpha \ge 0$, we may choose $\lambda_k$ sufficiently large such that $\gamma_k > 0$. Therefore $\psi_k$ grows exponentially in time, establishing linear instability.
\end{proof}

The condition

\begin{equation}
\beta \phi_0 (1 + \log \phi_0) > c^2
\label{eq:irruption_condition}
\end{equation}

is the entropic differential threshold for scalar irruption. It expresses that entropy curvature coupling overwhelms diffusive smoothing.

\section{Nonlinear Energy Functional and Global Instability}

To extend beyond linear analysis, define the energy functional

\begin{equation}
E[\phi]
=
\int_{\mathcal{M}}
\left(
\frac{c^2}{2} \norm{\nabla \phi}^2
- \frac{\alpha}{2} \phi^2
- \frac{\beta}{2} \phi^2 \Delta_g S
\right)
d\mu_g.
\label{eq:energy}
\end{equation}

If $\phi$ evolves according to a gradient flow

\begin{equation}
\partial_t \phi = - \frac{\delta E}{\delta \phi},
\end{equation}

then stationary points correspond to critical points of $E$.

The second variation $\delta^2 E$ evaluated at $\phi_0$ determines stability. A direct computation shows that under condition \eqref{eq:irruption_condition}, the quadratic form associated with $\delta^2 E$ becomes indefinite. Therefore $\phi_0$ ceases to be a local minimizer and becomes a saddle. This establishes nonlinear instability in the energy landscape.

\section{Integration into the Five-Engine Plenum Architecture}

Scalar Irruption via Entropic Differential (SIED) does not operate in isolation but functions as one dynamical engine within a larger plenum architecture. Let the total plenum dynamics be governed by five coupled operators:

\[
\mathcal{E} = \{ \mathcal{G}, \mathcal{R}, \mathcal{P}, \mathcal{I}, \mathcal{N} \},
\]

where $\mathcal{G}$ denotes Gradient Anisotropic Smoothing (GAS), $\mathcal{R}$ denotes Deferred Thermodynamic Reservoirs (DTR), $\mathcal{P}$ denotes Poincaré-Triggered Lattice Recrystallization (PTLR), $\mathcal{I}$ denotes Scalar Irruption via Entropic Differential (SIED), and $\mathcal{N}$ denotes the Neutrino Fossil Registry (NFR).

The full plenum evolution equation may be written schematically as

\[
\partial_t \phi = \mathcal{G}[\phi] + \mathcal{R}[\phi] + \mathcal{P}[\phi] + \mathcal{I}[\phi] + \mathcal{N}[\phi].
\]

Each operator modifies the scalar field $\phi$ in a distinct thermodynamic regime. GAS contributes diffusive smoothing of the form

\[
\mathcal{G}[\phi] = c^2 \Delta_g \phi,
\]

DTR introduces delayed entropy injection modeled by a memory kernel $K(t)$:

\[
\mathcal{R}[\phi](t) = \int_{-\infty}^{t} K(t-\tau)\phi(\tau)\,d\tau,
\]

PTLR acts discretely via lattice symmetry recurrence, modeled as a nonlinear projection

\[
\mathcal{P}[\phi] = \Pi_{\Lambda}(\phi),
\]

onto a crystallographic subspace $\Lambda$.

SIED contributes the entropic curvature coupling

\[
\mathcal{I}[\phi] = \beta \phi \Delta_g S.
\]

NFR records fossilized entropy traces through a weak residual operator

\[
\mathcal{N}[\phi] = \epsilon \int_{\mathcal{M}} \eta(x,y)\phi(y)\,dy,
\]

with $\epsilon \ll 1$.

Scalar irruption corresponds to regimes in which $\mathcal{I}$ dominates $\mathcal{G}$ locally, overwhelming smoothing.

\section{Entropy Vaults and Crack Points}

To formalize the ``entropy vault'' and ``crack point'' language present in earlier conceptual descriptions, define the entropy curvature scalar

\[
\kappa_S = -\Delta_g S.
\]

A region $U \subset \mathcal{M}$ is said to be an entropy vault if

\[
\kappa_S(x) > 0 \quad \text{for all } x \in U,
\]

meaning entropy is locally concave downward and effectively compressed.

A crack point occurs at $x_0 \in \mathcal{M}$ when

\[
\kappa_S(x_0) > \kappa_c,
\]

for some critical threshold $\kappa_c$ determined by

\[
\beta \phi_0 \kappa_c = c^2.
\]

At such points, the effective diffusion coefficient becomes negative:

\[
D_{\text{eff}} = c^2 - \beta \phi_0 \kappa_S.
\]

When $D_{\text{eff}} < 0$, the Laplacian term changes sign, converting smoothing into amplification. This defines the mathematical analog of a crack in the plenum.

\begin{proposition}
If $D_{\text{eff}} < 0$ on a non-empty open set $U$, then there exists a finite time $T$ such that $\norm{\phi}_{L^2(U)}$ increases monotonically for $t < T$.
\end{proposition}

\begin{proof}
Within $U$, the evolution reduces locally to
\[
\partial_t \phi = -|D_{\text{eff}}| \Delta_g \phi.
\]
Since $-\Delta_g$ is positive semidefinite, the operator generates growth in high-frequency modes. Integration over $U$ shows that the $L^2$ norm increases until nonlinear terms saturate the instability.
\end{proof}

\section{Relation to Poincaré-Triggered Lattice Recrystallization}

Consider a scalar field discretized over a lattice $\Lambda$. Let $\phi_n$ denote the scalar at lattice site $n$. PTLR acts via recurrence:

\[
\phi_n \mapsto \phi_n + \delta \phi_n,
\]

where recurrence times satisfy Poincaré return conditions.

When lattice recurrence induces localized entropy compression, the entropy differential satisfies

\[
\Delta_{\Lambda} S < -\kappa_c.
\]

Thus PTLR may generate preconditions for SIED. Scalar irruption therefore appears as a secondary instability following recurrence-induced anisotropy.

\section{Neutrino Fossil Registry as Memory Operator}

Let $S(x,t)$ evolve according to

\[
\partial_t S = -\gamma \phi + \delta \Delta_g S.
\]

Residual entropy traces after irruption events remain encoded in weakly interacting fields $\nu(x,t)$ satisfying

\[
\partial_t \nu = \epsilon \Delta_g \nu + \chi \phi.
\]

These traces serve as fossilized memory of prior irruptions. In linear approximation, $\nu$ records integrated scalar amplification events:

\[
\nu(x,t) \approx \chi \int_0^t e^{\epsilon (t-\tau)\Delta_g} \phi(x,\tau)\,d\tau.
\]

Thus the Neutrino Fossil Registry mathematically corresponds to a retarded integral memory kernel.

\section{Non-Expanding Cosmological Interpretation}

Let the metric $g$ be time-independent. The absence of scale factor $a(t)$ distinguishes this framework from standard cosmology. Structure arises not from background expansion but from sign inversion in effective diffusion.

Define the comoving scalar density

\[
\rho = \phi.
\]

Total mass is conserved:

\[
\frac{d}{dt} \int_{\mathcal{M}} \phi \, d\mu_g = 0,
\]

provided boundary flux vanishes. Scalar irruption redistributes density internally without requiring metric dilation.

\section{Bifurcation Analysis}

Consider the parameter family

\[
D(\lambda) = c^2 - \lambda,
\]

with $\lambda = \beta \phi_0 \kappa_S$.

The bifurcation point occurs at $\lambda = c^2$.

\begin{theorem}[Irruption as Pitchfork-Type Instability]
Near $\lambda = c^2$, the uniform state undergoes a supercritical bifurcation provided higher-order stabilizing terms are positive definite.
\end{theorem}

\begin{proof}
Expand the nonlinear evolution near threshold:
\[
\partial_t \psi = D(\lambda)\Delta_g \psi + \eta \psi^3.
\]
For $D(\lambda)>0$, perturbations decay. For $D(\lambda)<0$, growth occurs until balanced by the cubic term $\eta \psi^3$. Standard bifurcation theory yields a supercritical branch of stable non-uniform equilibria.
\end{proof}

This establishes scalar irruption as a genuine phase transition in entropic geometry rather than an arbitrary amplification event.

\section{Lamphron and Lamphrodyne States}

We now introduce a refinement of the scalar--entropy coupling in terms of dual thermodynamic regimes, termed lamphron and lamphrodyne states. These are not additional fields but distinct dynamical phases of the scalar--vector--entropy system.

Let $\phi$ denote the primary scalar density and introduce an effective vacuum response field $\chi$ defined implicitly through the entropy curvature:

\[
\chi := - \beta \Delta_g S.
\]

The evolution equation \eqref{eq:evolution} may then be rewritten as

\[
\partial_t \phi
=
c^2 \Delta_g \phi
- \nabla \cdot (\phi \mathbf{v})
+ \alpha \phi
+ \chi \phi.
\]

We define a lamphron state as a regime in which $\chi < 0$ almost everywhere, so that entropy curvature enhances smoothing and suppresses amplification. Conversely, a lamphrodyne state occurs when $\chi > 0$ on a region of nonzero measure, so that entropy curvature contributes to scalar growth.

\begin{definition}
A lamphrodyne domain $U \subset \mathcal{M}$ is an open set such that $\chi(x) > 0$ for all $x \in U$.
\end{definition}

In lamphron regions, the effective mass term remains stabilizing. In lamphrodyne regions, the effective mass becomes negative in the diffusive sector, inducing scalar irruption.

To formalize this dichotomy, define the effective linear operator

\[
\mathcal{L}_{\text{eff}} = c^2 \Delta_g + \alpha + \chi.
\]

If the principal symbol of $\mathcal{L}_{\text{eff}}$ changes sign in a region, the PDE transitions from parabolic smoothing to backward parabolic amplification. Thus lamphrodyne states correspond precisely to local sign inversion of the principal symbol.

\section{Spectral Decomposition and Resonant Structure Formation}

To analyze spatial patterning induced by scalar irruption, expand $\phi$ in eigenfunctions of the Laplace--Beltrami operator:

\[
\phi(x,t) = \sum_{k=0}^{\infty} a_k(t) \psi_k(x),
\qquad
\Delta_g \psi_k = -\lambda_k \psi_k,
\]

with $0 = \lambda_0 < \lambda_1 \le \lambda_2 \le \cdots$.

Substituting into the linearized irruption regime yields

\[
\dot{a}_k
=
\left(
- c^2 \lambda_k + \alpha + \chi_k
\right)
a_k,
\]

where $\chi_k$ denotes the spectral projection of $\chi$ onto $\psi_k$.

Growth occurs when

\[
\alpha + \chi_k > c^2 \lambda_k.
\]

Thus only a band of modes satisfying

\[
\lambda_k < \frac{\alpha + \chi_k}{c^2}
\]

are unstable. This band-limited instability naturally produces quasi-periodic structure formation analogous to baryon acoustic oscillation--like resonances, without invoking metric expansion.

\begin{proposition}
If $\chi$ is spatially localized and positive in a bounded region, then the unstable spectrum is discrete and finite.
\end{proposition}

\begin{proof}
Since $\lambda_k \to \infty$ as $k \to \infty$ and $\chi_k$ is bounded by $\norm{\chi}_{L^2}$, there exists $K$ such that for all $k>K$, $c^2 \lambda_k > \alpha + \chi_k$. Hence only finitely many modes are unstable.
\end{proof}

This explains why scalar irruption yields structured condensation rather than runaway ultraviolet divergence.

\section{Quantized Scalar Irruption Operator}

We now sketch a semiclassical quantization of scalar irruption.

Promote $\phi$ to an operator-valued field $\hat{\phi}$ on a Hilbert space $\mathcal{H}$. The classical Hamiltonian density corresponding to the Lagrangian in \eqref{eq:action} is

\[
\mathcal{H}
=
\frac{1}{2} \pi^2
+
\frac{c^2}{2} \norm{\nabla \phi}^2
+
U(\phi,S),
\]

where $\pi = \partial_t \phi$ is the conjugate momentum.

Quantization imposes canonical commutation relations

\[
[\hat{\phi}(x), \hat{\pi}(y)] = i\hbar \delta(x-y).
\]

In lamphrodyne regimes where $\chi>0$, the quadratic part of the Hamiltonian acquires negative eigenvalues, producing inverted harmonic oscillator sectors:

\[
\hat{H}_k
=
\frac{1}{2} \hat{\pi}_k^2
-
\frac{1}{2} \omega_k^2 \hat{\phi}_k^2.
\]

Such sectors exhibit exponential amplification of vacuum fluctuations, analogous to particle production in time-dependent backgrounds. However, here the trigger is entropic curvature rather than metric expansion.

\begin{theorem}
In a lamphrodyne domain, the vacuum state is dynamically unstable under the quantized Hamiltonian, leading to exponential growth of mode occupation numbers.
\end{theorem}

\begin{proof}
For an inverted harmonic oscillator,
\[
\ddot{\hat{\phi}}_k = \omega_k^2 \hat{\phi}_k,
\]
whose solutions grow exponentially. The number operator expectation value diverges as $e^{2\omega_k t}$.
\end{proof}

Thus scalar irruption admits a consistent quantum interpretation as entropically driven mode excitation.

\section{Observable Consequences in a Non-Expanding Plenum}

Since the metric is static, observable signatures arise from redistribution rather than expansion. Consider a two-point correlation function

\[
C(r) = \langle \phi(x)\phi(x+r)\rangle.
\]

In the linear regime, unstable modes imprint oscillatory features:

\[
C(r) \sim \sum_{k \in \mathcal{U}} e^{2\gamma_k t} \psi_k(x)\psi_k(x+r),
\]

where $\mathcal{U}$ denotes unstable modes.

These correlations mimic acoustic-like resonances in spatial power spectra while preserving constant global volume. Hence BAO-like signatures can emerge purely from entropic bifurcation.

\section{Entropy Conservation and Global Constraints}

Despite local amplification, global scalar mass is conserved under zero-flux boundary conditions:

\[
\frac{d}{dt} \int_{\mathcal{M}} \phi \, d\mu_g = 0.
\]

\begin{proof}
Integrating \eqref{eq:evolution} over $\mathcal{M}$ and applying the divergence theorem yields zero net flux provided $\mathbf{v}\cdot \mathbf{n}=0$ and $\nabla \phi \cdot \mathbf{n}=0$ on $\partial \mathcal{M}$.
\end{proof}

Scalar irruption therefore redistributes rather than creates total scalar density. Structure formation is internally reorganizational.

\section{Synthesis: Scalar Irruption as Entropic Phase Transition}

Scalar irruption is now seen to possess four equivalent formulations. In PDE language it is a sign reversal of effective diffusion. In spectral language it is band-limited exponential growth of Laplacian modes. In thermodynamic language it is curvature-induced entropy release. In quantum language it is an inverted oscillator instability induced by entropic geometry.

The mechanism requires no expanding metric, no vacuum tunneling, and no external inflaton potential. It emerges from internal entropy differentials within a non-expanding plenum.

\section{AKSZ/BV Formulation of Scalar Irruption}

We now reformulate scalar irruption within the Batalin--Vilkovisky (BV) and Alexandrov--Kontsevich--Schwarz--Zaboronsky (AKSZ) framework in order to exhibit its gauge structure and derived geometric consistency.

Let $\mathcal{M}$ be a smooth compact manifold and consider the graded manifold
\[
\mathcal{F} = T^*[-1]\mathrm{Map}(T[1]\mathcal{M}, \mathbb{R}),
\]
whose degree-zero component corresponds to the scalar field $\phi$. Introduce ghosts $c$ for diffeomorphism symmetry and antifields $\phi^*$ and $c^*$ of opposite degree.

The classical action functional for the scalar--entropy system may be written in AKSZ form as
\[
S_{\mathrm{cl}}[\phi]
=
\int_{\mathcal{M}}
\left(
\frac{c^2}{2} \norm{\nabla \phi}^2
+
\frac{\alpha}{2} \phi^2
+
\frac{\beta}{2} \phi^2 \Delta_g S
\right)
d\mu_g.
\]

The BV extension introduces antifields and ghost couplings:
\[
S_{\mathrm{BV}}
=
S_{\mathrm{cl}}
+
\int_{\mathcal{M}}
\phi^* \mathcal{L}_c \phi
+
c^* \frac{1}{2}[c,c],
\]
where $\mathcal{L}_c$ denotes the Lie derivative along the ghost vector field.

The BV bracket is defined by
\[
\{F,G\}
=
\int_{\mathcal{M}}
\left(
\frac{\delta_r F}{\delta \phi}
\frac{\delta_l G}{\delta \phi^*}
-
\frac{\delta_r F}{\delta \phi^*}
\frac{\delta_l G}{\delta \phi}
+
\frac{\delta_r F}{\delta c}
\frac{\delta_l G}{\delta c^*}
-
\frac{\delta_r F}{\delta c^*}
\frac{\delta_l G}{\delta c}
\right).
\]

The classical master equation requires
\[
\{S_{\mathrm{BV}}, S_{\mathrm{BV}}\} = 0.
\]

\begin{theorem}
The BV-extended scalar--entropy action satisfies the classical master equation provided the entropy functional $S[\phi]$ is diffeomorphism-invariant.
\end{theorem}

\begin{proof}
The entropy density $S = -\phi \log \phi$ is a scalar under diffeomorphisms. Therefore its Laplacian $\Delta_g S$ transforms covariantly. The ghost variation of $S_{\mathrm{cl}}$ cancels against the antifield terms by standard AKSZ construction. Hence the BV bracket vanishes.
\end{proof}

In lamphrodyne regimes, the quadratic part of the BV action develops negative eigenvalues in its kinetic operator. The derived critical locus of $S_{\mathrm{BV}}$ therefore acquires additional nontrivial homology in degree zero, corresponding to irruption branches.

Thus scalar irruption corresponds, in derived geometric language, to a change in the homotopy type of the critical locus of the action functional. The instability is not merely analytic but alters the derived stack of classical solutions.

This completes the BV/AKSZ formulation of scalar irruption as an entropically induced derived bifurcation.

\section{Crystal Plenum Discretization and Lattice Realization of Scalar Irruption}

We now construct a discrete realization of scalar irruption within the Crystal Plenum framework, in which the manifold $(\mathcal{M},g)$ is replaced by a lattice $\Lambda$ endowed with a discrete Laplacian.

Let $\Lambda$ be a finite or countably infinite lattice with adjacency relation $\sim$. The scalar field becomes a function
\[
\phi : \Lambda \to \mathbb{R},
\]
and the discrete Laplacian is defined by
\[
(\Delta_{\Lambda} \phi)_i = \sum_{j \sim i} (\phi_j - \phi_i).
\]

The entropy density at site $i$ is defined as
\[
S_i = - \phi_i \log \phi_i.
\]

The discrete evolution equation corresponding to \eqref{eq:evolution} becomes
\[
\dot{\phi}_i
=
c^2 (\Delta_{\Lambda} \phi)_i
+
\alpha \phi_i
+
\beta \phi_i (\Delta_{\Lambda} S)_i.
\]

\subsection{Discrete Irruption Criterion}

Let $\phi_i = \phi_0 + \varepsilon \psi_i$ with $\phi_0$ uniform. Linearizing yields
\[
\dot{\psi}_i
=
c^2 (\Delta_{\Lambda} \psi)_i
+
\alpha \psi_i
-
\beta \phi_0 (1+\log \phi_0) (\Delta_{\Lambda} \psi)_i.
\]

Define the effective diffusion constant
\[
D_{\mathrm{eff}} = c^2 - \beta \phi_0 (1+\log \phi_0).
\]

Let $\lambda_k$ denote eigenvalues of $-\Delta_{\Lambda}$. Then growth rates are
\[
\gamma_k = - D_{\mathrm{eff}} \lambda_k + \alpha.
\]

\begin{theorem}
If $D_{\mathrm{eff}}<0$, then the lattice system exhibits scalar irruption through exponential amplification of a finite set of eigenmodes.
\end{theorem}

\begin{proof}
Since the spectrum of $\Delta_{\Lambda}$ is bounded above on a finite lattice, only modes with sufficiently small $\lambda_k$ satisfy $\gamma_k>0$. Hence growth occurs in a controlled band, leading to patterned condensation.
\end{proof}

\subsection{Poincaré-Triggered Lattice Recrystallization}

In the crystal plenum, recurrence phenomena arise through approximate Poincaré return times of lattice configurations. Let $\phi(t)$ evolve under the discrete dynamics. A recurrence time $T$ satisfies
\[
\norm{\phi(T)-\phi(0)} < \varepsilon.
\]

Recurrence can amplify local entropy curvature by concentrating deviations. If at recurrence time $T$,
\[
(\Delta_{\Lambda} S)_i < -\kappa_c
\]
for some site $i$, the irruption condition is triggered.

Thus Poincaré-Triggered Lattice Recrystallization acts as a catalyst for scalar irruption by generating localized curvature spikes in the entropy landscape.

\subsection{Crystallization After Irruption}

Nonlinear saturation of unstable modes yields new equilibrium configurations $\phi_i^*$ satisfying
\[
0 = c^2 (\Delta_{\Lambda} \phi^*)_i + \alpha \phi_i^* + \beta \phi_i^* (\Delta_{\Lambda} S^*)_i.
\]

These equilibria correspond to discrete crystal-like scalar condensates.

\begin{proposition}
If the nonlinear saturation term is positive definite, the post-irruption equilibria minimize a discrete free energy functional.
\end{proposition}

\begin{proof}
Define the discrete free energy
\[
E_{\Lambda}[\phi] =
\sum_i
\left(
\frac{c^2}{2} \sum_{j \sim i} (\phi_j-\phi_i)^2
-
\frac{\alpha}{2}\phi_i^2
-
\frac{\beta}{2}\phi_i^2 (\Delta_{\Lambda} S)_i
\right).
\]
Critical points satisfy the equilibrium equation above, and positive definiteness of the quartic stabilization term ensures local minimality.
\end{proof}

\subsection{Discrete--Continuous Correspondence}

Let lattice spacing be $h$. Then
\[
\Delta_{\Lambda} \phi_i = h^2 \Delta_g \phi(x_i) + O(h^3).
\]

Thus the discrete irruption condition converges to the continuous condition as $h \to 0$.

This demonstrates that scalar irruption is not an artifact of continuum modeling but persists under crystallographic discretization. The Crystal Plenum therefore provides a geometrically concrete realization of entropically driven scalar phase transitions.


\section{Conclusion}

Scalar irruption has been formulated as an entropically driven instability in a non-expanding plenum. Beginning from a variational principle, we derived the coupled evolution equation and proved that sufficiently negative effective entropy curvature induces exponential scalar amplification. The instability arises not from metric expansion or vacuum tunneling but from the sign reversal of an effective diffusive coefficient in the presence of entropy curvature.

Scalar irruption is therefore a phase transition in entropic geometry. In subsequent analysis, one may extend this mechanism to include vector flow coupling, fully nonlinear entropy functionals, and quantization via derived geometric methods. The essential conclusion remains that structure can emerge from smoothing dynamics when entropy differentials exceed a critical threshold.

\newpage
\begin{thebibliography}{99}

\bibitem{Arnold1989}
Arnold, V. I. (1989).
\textit{Mathematical Methods of Classical Mechanics}.
Springer.

\bibitem{BatalinVilkovisky1981}
Batalin, I. A., and Vilkovisky, G. A. (1981).
Gauge algebra and quantization.
\textit{Physics Letters B}, 102(1), 27--31.

\bibitem{HenneauxTeitelboim1992}
Henneaux, M., and Teitelboim, C. (1992).
\textit{Quantization of Gauge Systems}.
Princeton University Press.

\bibitem{AlexandrovKontsevichSchwarzZaboronsky1997}
Alexandrov, M., Kontsevich, M., Schwarz, A., and Zaboronsky, O. (1997).
The geometry of the master equation and topological quantum field theory.
\textit{International Journal of Modern Physics A}, 12(07), 1405--1429.

\bibitem{Evans2010}
Evans, L. C. (2010).
\textit{Partial Differential Equations}.
American Mathematical Society.

\bibitem{Taylor1996}
Taylor, M. E. (1996).
\textit{Partial Differential Equations I--III}.
Springer.

\bibitem{Zeidler1990}
Zeidler, E. (1990).
\textit{Nonlinear Functional Analysis and Its Applications}.
Springer.

\bibitem{Turing1952}
Turing, A. M. (1952).
The chemical basis of morphogenesis.
\textit{Philosophical Transactions of the Royal Society B}, 237, 37--72.

\bibitem{CrossHohenberg1993}
Cross, M. C., and Hohenberg, P. C. (1993).
Pattern formation outside of equilibrium.
\textit{Reviews of Modern Physics}, 65(3), 851--1112.

\bibitem{Ising1925}
Ising, E. (1925).
Contribution to the theory of ferromagnetism.
\textit{Zeitschrift für Physik}, 31, 253--258.

\bibitem{Onsager1944}
Onsager, L. (1944).
Crystal statistics. I. A two-dimensional model with an order-disorder transition.
\textit{Physical Review}, 65, 117--149.

\bibitem{Strogatz1994}
Strogatz, S. H. (1994).
\textit{Nonlinear Dynamics and Chaos}.
Westview Press.

\bibitem{Gilmore1993}
Gilmore, R. (1993).
\textit{Catastrophe Theory for Scientists and Engineers}.
Dover.

\bibitem{Kolmogorov1965}
Kolmogorov, A. N. (1965).
Three approaches to the quantitative definition of information.
\textit{Problems of Information Transmission}, 1, 1--7.

\bibitem{Shannon1948}
Shannon, C. E. (1948).
A mathematical theory of communication.
\textit{Bell System Technical Journal}, 27, 379--423.

\bibitem{Weinberg1995}
Weinberg, S. (1995).
\textit{The Quantum Theory of Fields, Volume I}.
Cambridge University Press.

\bibitem{BirrellDavies1982}
Birrell, N. D., and Davies, P. C. W. (1982).
\textit{Quantum Fields in Curved Space}.
Cambridge University Press.

\bibitem{Kadanoff2000}
Kadanoff, L. P. (2000).
\textit{Statistical Physics: Statics, Dynamics and Renormalization}.
World Scientific.

\bibitem{BratteliRobinson1987}
Bratteli, O., and Robinson, D. W. (1987).
\textit{Operator Algebras and Quantum Statistical Mechanics}.
Springer.

\bibitem{MacLane1998}
Mac Lane, S. (1998).
\textit{Categories for the Working Mathematician}.
Springer.

\bibitem{May1999}
May, J. P. (1999).
\textit{A Concise Course in Algebraic Topology}.
University of Chicago Press.

\bibitem{GelfandFomin2000}
Gelfand, I. M., and Fomin, S. V. (2000).
\textit{Calculus of Variations}.
Dover.

\bibitem{Smale1967}
Smale, S. (1967).
Differentiable dynamical systems.
\textit{Bulletin of the American Mathematical Society}, 73, 747--817.

\bibitem{Hawking1975}
Hawking, S. W. (1975).
Particle creation by black holes.
\textit{Communications in Mathematical Physics}, 43, 199--220.

\bibitem{Penrose2004}
Penrose, R. (2004).
\textit{The Road to Reality}.
Jonathan Cape.

\end{thebibliography}


\end{document}

