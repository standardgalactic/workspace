\documentclass[12pt]{article}

\usepackage[letterpaper,left=1.25in,right=1.25in,top=1in,bottom=1in]{geometry}
\usepackage{setspace}
\usepackage{microtype}
\usepackage{amsmath,amssymb,amsfonts}
\usepackage{mathtools}
\usepackage{bm}
\usepackage{hyperref}

\setstretch{1.15}
\setlength{\parindent}{0pt}
\setlength{\parskip}{0.75em}


\title{Scalar Irruption via Entropic Differential:\\
A Non-Equilibrium Mechanism for Structure Formation in a Non-Expanding Plenum}

\author{Flyxion}
\date{\today}

\begin{document}

\maketitle

\begin{abstract}

We present a mathematically controlled mechanism for band-limited instability in a conserved relativistic fluid, formulated as a coarse-grained free-energy theory with explicit ultraviolet regulation. The model identifies scalar irruption with the loss of convexity of an effective thermodynamic functional, rather than with backward diffusion. Instability occurs only within a finite wavenumber band and is suppressed at high frequency by either gradient-energy regularization or finite-correlation kernels. The resulting dynamics are well-posed, conserve particle number exactly, and admit a causal constitutive completion. Gravitational coupling is introduced only after the minimal closed system is established, ensuring that conservation laws and regulator structure are fixed at the level of the matter sector. The framework thereby provides a defensible route from spinodal instability to scale-selected structure formation without invoking metric expansion or ultraviolet pathologies.
\end{abstract}


\newpage
\section{Introduction}

Standard cosmological structure formation is typically modeled through gravitational instability in an expanding background geometry. In contrast, the present framework assumes a non-expanding plenum endowed with a scalar field $\phi$, a vector flow $\mathbf{v}$, and an entropy density $S$. The dynamics of these fields are governed not by metric expansion but by entropic smoothing and transport. Within this setting, the emergence of structure must be explained without recourse to inflationary vacuum potentials or background scale factors.

Scalar irruption is proposed as the fundamental mechanism of structure formation under these assumptions. Informally, it denotes the abrupt local amplification of a scalar field when entropy curvature crosses a critical threshold. The aim of this essay is to replace that informal statement with a mathematically precise derivation, beginning from an action principle and proceeding through stability analysis to a rigorous instability criterion.

We assume throughout that the spatial manifold $(\mathcal{M},g)$ is smooth, compact without boundary unless otherwise stated, and equipped with a Riemannian metric $g$. All fields are taken to be sufficiently regular for the variational calculus employed.

\section{Effective-Theory Status and Conservation Structure}

The model developed here is explicitly an effective non-equilibrium fluid theory. The fundamental dynamical variable is a strictly conserved scalar density $n$, interpreted as particle-number density or comoving matter density. Entropy is not introduced as an independent ontological field; instead, coarse-grained thermodynamic structure is encoded in an effective free-energy functional $\mathcal{F}[n]$.

We assume a spacetime manifold $(\mathcal{M},g_{\mu\nu})$ with signature $(-,+,+,+)$ and a timelike four-velocity field $u^\mu$ satisfying
\[
u^\mu u_\mu = -1.
\]

Particle number conservation is imposed exactly through the current
\[
J^\mu = n u^\mu + \mathcal{J}^\mu,
\qquad
u_\mu \mathcal{J}^\mu = 0,
\]
with
\[
\nabla_\mu J^\mu = 0.
\]

No violation of local conservation is permitted at any stage. All instability arises through constitutive structure rather than through modification of conservation laws.

\section{Coarse-Grained Free Energy and Regulator}

Let $h_{\mu\nu} = g_{\mu\nu} + u_\mu u_\nu$ denote the spatial projector orthogonal to $u^\mu$, and define the spatial covariant derivative
\[
D_\mu := h_\mu{}^\nu \nabla_\nu.
\]

On hypersurfaces orthogonal to $u^\mu$, define the regulated free-energy functional
\[
\mathcal{F}[n]
=
\int_{\Sigma_t}
\sqrt{\gamma}
\left(
f(n)
+
\frac{\kappa}{2}
D_i n D^i n
\right)
\, d^3x,
\]
where $\gamma$ is the induced spatial metric determinant, $f(n)$ is an effective coarse-grained free-energy density, and $\kappa>0$ is a regulator coefficient encoding finite correlation length.

The associated chemical potential is the variational derivative
\[
\mu
=
\frac{\delta \mathcal{F}}{\delta n}
=
f'(n)
-
\kappa D^2 n,
\qquad
D^2 := D_i D^i.
\]

The coefficient $\kappa$ provides an explicit gradient-energy cost, suppressing arbitrarily short-wavelength structure and thereby regularizing the ultraviolet sector.

\section{Constitutive Closure and Causality}

The simplest constitutive relation generalizing Fick's law is
\[
\mathcal{J}^\mu
=
-
M
h^{\mu\nu}
\nabla_\nu \mu,
\]
with mobility $M>0$.

This closure produces a relativistic Cahn–Hilliard-type equation. However, as a parabolic theory it implies infinite signal speed. To maintain compatibility with relativistic causality at the effective level, we adopt a minimal causal relaxation law:
\[
\tau\, u^\alpha \nabla_\alpha \mathcal{J}^\mu
+
\mathcal{J}^\mu
=
-
M h^{\mu\nu} \nabla_\nu \mu,
\qquad
u_\mu \mathcal{J}^\mu = 0,
\]
with relaxation time $\tau>0$.

In the limit $t \gg \tau$, the parabolic form is recovered, while short-time dynamics remain hyperbolic. The instability spectrum derived below is unaffected at leading order in this regime.

\section{Linear Stability and Band-Limited Growth}

We first analyze the minimal closed matter system on a fixed background. Let $n = \bar n + \delta n$ with $\bar n$ spatially homogeneous.

Expanding the chemical potential to first order gives
\[
\delta \mu
=
f''(\bar n)\,\delta n
-
\kappa D^2 \delta n.
\]

In the comoving frame on subhorizon scales,
\[
D^2 \approx a^{-2}\nabla^2,
\]
and linearized number conservation yields
\[
\partial_t \delta n
=
M a^{-2} \nabla^2 \delta \mu
-
3H \delta n,
\]
where $H$ is the background expansion rate (which may be set to zero in a static plenum limit).

In Fourier space, $\delta n \sim e^{i\vec{k}\cdot\vec{x}}$, we obtain
\[
\partial_t \delta n_k
=
\gamma(k,t)\, \delta n_k,
\]
with growth rate
\[
\gamma(k,t)
=
M\left(
-
f''(\bar n)\frac{k^2}{a^2}
-
\kappa \frac{k^4}{a^4}
\right)
-
3H.
\]

In the spinodal regime where
\[
f''(\bar n) < 0,
\]
define
\[
a := M |f''(\bar n)|,
\qquad
b := M\kappa.
\]

Then
\[
\gamma(k,t)
=
a \frac{k^2}{a^2}
-
b \frac{k^4}{a^4}
-
3H.
\]

This immediately implies:

\begin{itemize}
\item Instability occurs only for finite band $0 < k_{\mathrm{phys}} < \sqrt{\frac{|f''|}{\kappa}}$.
\item Ultraviolet modes are suppressed by the $k^4$ term.
\item The fastest-growing physical wavenumber satisfies
\[
k_{\mathrm{phys},*}^2 = \frac{|f''(\bar n)|}{2\kappa}.
\]
\end{itemize}

The instability is therefore band-limited and well-posed. The regulator $\kappa$ defines a physical correlation length
\[
\ell \sim \sqrt{\frac{\kappa}{|f''|}},
\]
which suppresses arbitrarily small-scale growth.

No backward-parabolic pathology appears because the highest-order operator remains positive-definite.

\section{Interpretation}

Scalar irruption is identified with the crossing of the convexity boundary
\[
\delta^2 \mathcal{F} = 0,
\]
that is, the transition from positive to negative second variation of the coarse-grained free energy. The instability corresponds to spinodal phase separation in the conserved density sector.

The model thus describes phase decomposition in an effective thermodynamic landscape, not reverse entropy flow. The regulator ensures ultraviolet stability, and conservation is preserved exactly.

\section{Variational Formulation of the Coupled System}

Let $\phi : \mathcal{M} \times \mathbb{R} \to \mathbb{R}$ denote a scalar field, $\mathbf{v}$ a vector field on $\mathcal{M}$, and $S$ an entropy density functionally dependent on $\phi$ unless otherwise specified. We begin with an action functional of the form

\begin{equation}
\mathcal{A}[\phi,\mathbf{v}]
=
\int_{\mathbb{R}} \int_{\mathcal{M}}
\left(
\frac{1}{2} (\partial_t \phi)^2
- \frac{c^2}{2} \norm{\nabla \phi}^2
+ \mathbf{v} \cdot \nabla \phi
- U(\phi,S)
\right)
\, d\mu_g \, dt,
\label{eq:action}
\end{equation}

where $d\mu_g$ is the Riemannian volume form, $c$ is a characteristic smoothing scale, and $U(\phi,S)$ is an effective potential coupling $\phi$ to entropy.

We assume that entropy is defined by a functional of the scalar field,

\begin{equation}
S[\phi] = - \phi \log \phi,
\label{eq:entropy_density}
\end{equation}

interpreted pointwise. More generally, one may take

\begin{equation}
S[\phi] = - \int_{\mathcal{M}} \phi \log \phi \, d\mu_g,
\label{eq:entropy_global}
\end{equation}

but for local instability analysis the pointwise density is sufficient.

We consider a coupling of the form

\begin{equation}
U(\phi,S) = \frac{\alpha}{2} \phi^2 + \beta \phi \, \Delta_g S,
\label{eq:potential}
\end{equation}

where $\alpha$ and $\beta$ are constants and $\Delta_g$ is the Laplace--Beltrami operator.

The Euler--Lagrange equation for $\phi$ follows from

\begin{equation}
\frac{\delta \mathcal{A}}{\delta \phi} = 0.
\end{equation}

A straightforward variation yields

\begin{equation}
\partial_t^2 \phi
- c^2 \Delta_g \phi
+ \nabla \cdot \mathbf{v}
- \alpha \phi
- \beta \Delta_g S
= 0.
\label{eq:euler_lagrange}
\end{equation}

If we pass to a dissipative regime by introducing first-order temporal dynamics appropriate to smoothing processes, we consider instead the evolution equation

\begin{equation}
\partial_t \phi
=
c^2 \Delta_g \phi
- \nabla \cdot (\phi \mathbf{v})
+ \alpha \phi
+ \beta \phi \, \Delta_g S.
\label{eq:evolution}
\end{equation}

Equation \eqref{eq:evolution} will serve as the starting point for instability analysis. The final term, proportional to $\phi \Delta_g S$, encodes entropic curvature coupling. Scalar irruption will be shown to arise when this term dominates diffusive smoothing.

\section{Linear Stability Analysis and the Irruption Criterion}

We now consider a background configuration $\phi_0$ that is spatially uniform and stationary under \eqref{eq:evolution}. Assume $\mathbf{v}=0$ for simplicity in the local analysis. Then the equation reduces to

\begin{equation}
\partial_t \phi = c^2 \Delta_g \phi + \alpha \phi + \beta \phi \, \Delta_g S.
\end{equation}

Let $\phi = \phi_0 + \varepsilon \psi$, where $\varepsilon \ll 1$ and $\psi$ is a perturbation. Expanding $S[\phi]$ to first order yields

\begin{equation}
S[\phi] = S[\phi_0] + \varepsilon S'[\phi_0] \psi + O(\varepsilon^2).
\end{equation}

Since $S'[\phi_0] = - (1 + \log \phi_0)$, we compute

\begin{equation}
\Delta_g S \approx - (1 + \log \phi_0) \Delta_g \psi.
\end{equation}

Substituting into the linearized equation gives

\begin{equation}
\partial_t \psi
=
c^2 \Delta_g \psi
+ \alpha \psi
- \beta \phi_0 (1 + \log \phi_0) \Delta_g \psi.
\end{equation}

Collecting Laplacian terms,

\begin{equation}
\partial_t \psi
=
\left(
c^2 - \beta \phi_0 (1 + \log \phi_0)
\right) \Delta_g \psi
+ \alpha \psi.
\label{eq:linearized}
\end{equation}

Let $\psi_k$ be an eigenfunction of $\Delta_g$ with eigenvalue $- \lambda_k$, $\lambda_k \ge 0$. Then

\begin{equation}
\partial_t \psi_k
=
-
\left(
c^2 - \beta \phi_0 (1 + \log \phi_0)
\right)
\lambda_k \psi_k
+ \alpha \psi_k.
\end{equation}

The growth rate $\gamma_k$ is therefore

\begin{equation}
\gamma_k
=
-
\left(
c^2 - \beta \phi_0 (1 + \log \phi_0)
\right)
\lambda_k
+ \alpha.
\end{equation}

We obtain the following result.

\begin{theorem}[Scalar Irruption Criterion]
Suppose $\alpha \ge 0$. If there exists $k$ such that
\[
c^2 - \beta \phi_0 (1 + \log \phi_0) < 0,
\]
then for sufficiently large $\lambda_k$, the growth rate $\gamma_k$ becomes positive, and the uniform state $\phi_0$ is linearly unstable. This instability corresponds to scalar irruption.
\end{theorem}

\begin{proof}
If the coefficient of $\lambda_k$ is negative, then
\[
- \left( c^2 - \beta \phi_0 (1 + \log \phi_0) \right) \lambda_k
\]
is positive and grows unbounded as $\lambda_k \to \infty$. Since $\alpha \ge 0$, we may choose $\lambda_k$ sufficiently large such that $\gamma_k > 0$. Therefore $\psi_k$ grows exponentially in time, establishing linear instability.
\end{proof}

The condition

\begin{equation}
\beta \phi_0 (1 + \log \phi_0) > c^2
\label{eq:irruption_condition}
\end{equation}

is the entropic differential threshold for scalar irruption. It expresses that entropy curvature coupling overwhelms diffusive smoothing.

\section{Nonlinear Energy Functional and Global Instability}

To extend beyond linear analysis, define the energy functional

\begin{equation}
E[\phi]
=
\int_{\mathcal{M}}
\left(
\frac{c^2}{2} \norm{\nabla \phi}^2
- \frac{\alpha}{2} \phi^2
- \frac{\beta}{2} \phi^2 \Delta_g S
\right)
d\mu_g.
\label{eq:energy}
\end{equation}

If $\phi$ evolves according to a gradient flow

\begin{equation}
\partial_t \phi = - \frac{\delta E}{\delta \phi},
\end{equation}

then stationary points correspond to critical points of $E$.

The second variation $\delta^2 E$ evaluated at $\phi_0$ determines stability. A direct computation shows that under condition \eqref{eq:irruption_condition}, the quadratic form associated with $\delta^2 E$ becomes indefinite. Therefore $\phi_0$ ceases to be a local minimizer and becomes a saddle. This establishes nonlinear instability in the energy landscape.

\section{Stress--Energy Structure and Variational Consistency}

To ensure that the regulator term is not merely a phenomenological closure but a genuine energetic contribution, we now derive a consistent stress--energy tensor from a covariant effective action. The matter sector is defined by the Lagrangian density
\[
\mathcal{L}
=
\sqrt{-g}
\left(
-\varepsilon(n)
-
\frac{\kappa}{2}
h^{\mu\nu}
\nabla_\mu n \nabla_\nu n
\right),
\]
where $\varepsilon(n)$ is an effective energy density and $\kappa>0$ is the gradient-energy coefficient introduced previously. The projector $h^{\mu\nu}=g^{\mu\nu}+u^\mu u^\nu$ enforces that gradient energy is spatial in the comoving frame.

Variation with respect to the metric yields the stress--energy tensor
\[
T^{\mu\nu}
=
\frac{2}{\sqrt{-g}}
\frac{\delta \mathcal{L}}{\delta g_{\mu\nu}}.
\]

Carrying out the variation produces
\[
T^{\mu\nu}
=
(\varepsilon + p) u^\mu u^\nu
+
p g^{\mu\nu}
+
\kappa
\left(
\nabla^\mu n \nabla^\nu n
-
\frac{1}{2} g^{\mu\nu}
\nabla_\alpha n \nabla^\alpha n
\right)
+
\Theta^{\mu\nu},
\]
where $p(n) = n \varepsilon'(n) - \varepsilon(n)$ is the effective pressure and $\Theta^{\mu\nu}$ contains additional projection terms if one insists on strictly spatial gradients in all frames. In the comoving frame these reduce to the familiar Korteweg-type stress contributions known from capillarity and phase-field theory.

Because this tensor is derived from a covariant action, conservation follows from diffeomorphism invariance,
\[
\nabla_\mu T^{\mu\nu} = 0,
\]
provided the matter equations of motion are satisfied. The regulator therefore has a manifest energetic cost and is not introduced ad hoc. The gradient term modifies both isotropic pressure and anisotropic stress, with contributions scaling as $k^2 \delta n$ and $k^4 \delta n$ in Fourier space. These corrections will later provide the scale-dependent observational signature.

\section{Well-Posedness and Spinodal Instability}

The central mathematical issue is whether the instability mechanism is pathological. The structure derived above ensures that it is not. The highest-order spatial operator entering the linearized evolution equation is positive-definite. The apparent ``negative diffusion'' arises not from reversal of the leading derivative term but from the sign of the second variation of the coarse-grained free-energy density.

To make this precise, consider the second variation of $\mathcal{F}[n]$ about a homogeneous background $\bar n$,
\[
\delta^2 \mathcal{F}
=
\int_{\Sigma_t}
\sqrt{\gamma}
\left(
f''(\bar n)\, (\delta n)^2
+
\kappa\, D_i \delta n D^i \delta n
\right)
d^3x.
\]

The system is linearly stable if and only if this quadratic form is positive-definite. Instability arises when $f''(\bar n)<0$, in which case long-wavelength modes reduce the free energy. The gradient term proportional to $\kappa$ prevents arbitrarily small-scale growth and therefore regularizes the ultraviolet sector. The instability band is finite, and the growth rate is bounded.

This structure is identical in form to relativistic generalizations of spinodal decomposition. The instability reflects loss of convexity of the free-energy density, not reversal of entropy flow. The regulator introduces a physical correlation length
\[
\ell \sim \sqrt{\frac{\kappa}{|f''(\bar n)|}},
\]
below which the continuum description ceases to support amplification.

Because the leading operator remains fourth-order parabolic in the comoving frame, the initial-value problem is well-posed. If the causal relaxation extension is retained, the full system is hyperbolic at short times and diffusive only in the late-time limit.

\section{Minimal Cosmological Embedding}

We now embed the matter sector in a spatially homogeneous Friedmann--Robertson--Walker background,
\[
ds^2 = -dt^2 + a(t)^2 d\vec{x}^2,
\qquad
u^\mu = (1,0,0,0).
\]

On subhorizon scales, the spatial Laplacian reduces to $D^2 \approx a^{-2}\nabla^2$. Linearizing about $n(t,\vec{x})=\bar n(t)+\delta n(t,\vec{x})$ yields the growth equation
\[
\partial_t \delta n_k
=
M\left(
|f''(\bar n)| \frac{k^2}{a^2}
-
\kappa \frac{k^4}{a^4}
\right)\delta n_k
-
3H \delta n_k.
\]

The instability therefore competes with Hubble damping. Modes satisfying
\[
0 < k_{\mathrm{phys}}^2 < \frac{|f''(\bar n)|}{\kappa}
\]
experience amplification, with fastest growth at
\[
k_{\mathrm{phys},*}^2 = \frac{|f''(\bar n)|}{2\kappa}.
\]

The preferred physical wavelength
\[
\lambda_* = \frac{2\pi}{k_{\mathrm{phys},*}}
\]
is directly determined by the regulator scale and the curvature of the free-energy density. If $|f''|$ and $\kappa$ evolve slowly, the selected scale remains approximately constant in physical units. If they evolve significantly, the instability spectrum acquires a calculable time dependence.

This establishes a direct parameter-to-scale map without invoking exponential metric expansion.

\section{Interpretive Constraint}

At this stage the theory contains no independent entropy field and no ontological duplication of degrees of freedom. All instability derives from the curvature of a single effective functional $\mathcal{F}[n]$ governing a conserved density. The language of ``entropy vaults'' or ``crack points'' can therefore be translated rigorously into the condition
\[
f''(\bar n)=0,
\]
which marks the convexity boundary of the coarse-grained free-energy density.

Scalar irruption is thus identified with spinodal phase separation in a relativistic effective fluid possessing a finite correlation length. The instability is band-limited, the equations are well-posed, and conservation laws are maintained.

Further structure, including gravitational backreaction and potential scale-dependent modifications of the scalar perturbation sector, can now be developed on this fixed mathematical foundation without altering the core conservation structure.

\section{Integration into the Five-Engine Plenum Architecture}

Scalar Irruption via Entropic Differential (SIED) does not operate in isolation but functions as one dynamical engine within a larger plenum architecture. Let the total plenum dynamics be governed by five coupled operators:

\[
\mathcal{E} = \{ \mathcal{G}, \mathcal{R}, \mathcal{P}, \mathcal{I}, \mathcal{N} \},
\]

where $\mathcal{G}$ denotes Gradient Anisotropic Smoothing (GAS), $\mathcal{R}$ denotes Deferred Thermodynamic Reservoirs (DTR), $\mathcal{P}$ denotes Poincaré-Triggered Lattice Recrystallization (PTLR), $\mathcal{I}$ denotes Scalar Irruption via Entropic Differential (SIED), and $\mathcal{N}$ denotes the Neutrino Fossil Registry (NFR).

The full plenum evolution equation may be written schematically as

\[
\partial_t \phi = \mathcal{G}[\phi] + \mathcal{R}[\phi] + \mathcal{P}[\phi] + \mathcal{I}[\phi] + \mathcal{N}[\phi].
\]

Each operator modifies the scalar field $\phi$ in a distinct thermodynamic regime. GAS contributes diffusive smoothing of the form

\[
\mathcal{G}[\phi] = c^2 \Delta_g \phi,
\]

DTR introduces delayed entropy injection modeled by a memory kernel $K(t)$:

\[
\mathcal{R}[\phi](t) = \int_{-\infty}^{t} K(t-\tau)\phi(\tau)\,d\tau,
\]

PTLR acts discretely via lattice symmetry recurrence, modeled as a nonlinear projection

\[
\mathcal{P}[\phi] = \Pi_{\Lambda}(\phi),
\]

onto a crystallographic subspace $\Lambda$.

SIED contributes the entropic curvature coupling

\[
\mathcal{I}[\phi] = \beta \phi \Delta_g S.
\]

NFR records fossilized entropy traces through a weak residual operator

\[
\mathcal{N}[\phi] = \epsilon \int_{\mathcal{M}} \eta(x,y)\phi(y)\,dy,
\]

with $\epsilon \ll 1$.

Scalar irruption corresponds to regimes in which $\mathcal{I}$ dominates $\mathcal{G}$ locally, overwhelming smoothing.

\section{Entropy Vaults and Crack Points}

To formalize the ``entropy vault'' and ``crack point'' language present in earlier conceptual descriptions, define the entropy curvature scalar

\[
\kappa_S = -\Delta_g S.
\]

A region $U \subset \mathcal{M}$ is said to be an entropy vault if

\[
\kappa_S(x) > 0 \quad \text{for all } x \in U,
\]

meaning entropy is locally concave downward and effectively compressed.

A crack point occurs at $x_0 \in \mathcal{M}$ when

\[
\kappa_S(x_0) > \kappa_c,
\]

for some critical threshold $\kappa_c$ determined by

\[
\beta \phi_0 \kappa_c = c^2.
\]

At such points, the effective diffusion coefficient becomes negative:

\[
D_{\text{eff}} = c^2 - \beta \phi_0 \kappa_S.
\]

When $D_{\text{eff}} < 0$, the Laplacian term changes sign, converting smoothing into amplification. This defines the mathematical analog of a crack in the plenum.

\begin{proposition}
If $D_{\text{eff}} < 0$ on a non-empty open set $U$, then there exists a finite time $T$ such that $\norm{\phi}_{L^2(U)}$ increases monotonically for $t < T$.
\end{proposition}

\begin{proof}
Within $U$, the evolution reduces locally to
\[
\partial_t \phi = -|D_{\text{eff}}| \Delta_g \phi.
\]
Since $-\Delta_g$ is positive semidefinite, the operator generates growth in high-frequency modes. Integration over $U$ shows that the $L^2$ norm increases until nonlinear terms saturate the instability.
\end{proof}

\section{Relation to Poincaré-Triggered Lattice Recrystallization}

Consider a scalar field discretized over a lattice $\Lambda$. Let $\phi_n$ denote the scalar at lattice site $n$. PTLR acts via recurrence:

\[
\phi_n \mapsto \phi_n + \delta \phi_n,
\]

where recurrence times satisfy Poincaré return conditions.

When lattice recurrence induces localized entropy compression, the entropy differential satisfies

\[
\Delta_{\Lambda} S < -\kappa_c.
\]

Thus PTLR may generate preconditions for SIED. Scalar irruption therefore appears as a secondary instability following recurrence-induced anisotropy.

\section{Neutrino Fossil Registry as Memory Operator}

Let $S(x,t)$ evolve according to

\[
\partial_t S = -\gamma \phi + \delta \Delta_g S.
\]

Residual entropy traces after irruption events remain encoded in weakly interacting fields $\nu(x,t)$ satisfying

\[
\partial_t \nu = \epsilon \Delta_g \nu + \chi \phi.
\]

These traces serve as fossilized memory of prior irruptions. In linear approximation, $\nu$ records integrated scalar amplification events:

\[
\nu(x,t) \approx \chi \int_0^t e^{\epsilon (t-\tau)\Delta_g} \phi(x,\tau)\,d\tau.
\]

Thus the Neutrino Fossil Registry mathematically corresponds to a retarded integral memory kernel.

\section{Non-Expanding Cosmological Interpretation}

Let the metric $g$ be time-independent. The absence of scale factor $a(t)$ distinguishes this framework from standard cosmology. Structure arises not from background expansion but from sign inversion in effective diffusion.

Define the comoving scalar density

\[
\rho = \phi.
\]

Total mass is conserved:

\[
\frac{d}{dt} \int_{\mathcal{M}} \phi \, d\mu_g = 0,
\]

provided boundary flux vanishes. Scalar irruption redistributes density internally without requiring metric dilation.

\section{Bifurcation Analysis}

Consider the parameter family

\[
D(\lambda) = c^2 - \lambda,
\]

with $\lambda = \beta \phi_0 \kappa_S$.

The bifurcation point occurs at $\lambda = c^2$.

\begin{theorem}[Irruption as Pitchfork-Type Instability]
Near $\lambda = c^2$, the uniform state undergoes a supercritical bifurcation provided higher-order stabilizing terms are positive definite.
\end{theorem}

\begin{proof}
Expand the nonlinear evolution near threshold:
\[
\partial_t \psi = D(\lambda)\Delta_g \psi + \eta \psi^3.
\]
For $D(\lambda)>0$, perturbations decay. For $D(\lambda)<0$, growth occurs until balanced by the cubic term $\eta \psi^3$. Standard bifurcation theory yields a supercritical branch of stable non-uniform equilibria.
\end{proof}

This establishes scalar irruption as a genuine phase transition in entropic geometry rather than an arbitrary amplification event.

\section{Frame Choice, Diffusive Flux, and Stress--Energy Consistency}

The presence of gradient--driven fluxes requires a clear distinction between bulk motion and diffusive transport. Without such a distinction, one risks identifying the peculiar velocity with a constitutive diffusion current, which is inconsistent with relativistic conservation structure.

\subsection{Particle Frame and Energy Frame}

Let $J^\mu$ denote the conserved particle-number current,
\[
\nabla_\mu J^\mu = 0.
\]

We decompose
\[
J^\mu = n u^\mu + j^\mu,
\]
where $u^\mu u_\mu = -1$ and $u_\mu j^\mu = 0$.

The choice of frame determines how $u^\mu$ is defined:

In the Eckart (particle) frame, $u^\mu$ is aligned with $J^\mu$, so that $j^\mu=0$.

In the Landau--Lifshitz (energy) frame, $u^\mu$ is defined by the timelike eigenvector of $T^{\mu\nu}$, and $j^\mu$ represents a physical diffusion current relative to the energy flow.

For definiteness, we adopt the Landau--Lifshitz frame. The bulk velocity $u^\mu$ is therefore determined by the stress--energy tensor, while diffusive effects enter only through $j^\mu$.

\subsection{Constitutive Closure}

The regulated scalar sector induces a chemical potential
\[
\mu = \frac{\delta \mathcal{F}}{\delta n}
=
f'(n)
-
\kappa D^2 n.
\]

A conservative diffusive closure takes the form
\[
j^\mu = - M h^{\mu\nu} \nabla_\nu \mu,
\]
where $M>0$ is a mobility and
\[
h_{\mu\nu} = g_{\mu\nu} + u_\mu u_\nu
\]
is the spatial projector orthogonal to $u^\mu$.

Importantly, $j^\mu$ modifies the particle current but does not define the bulk velocity. The bulk velocity remains the velocity appearing in the stress--energy tensor.

\subsection{Stress--Energy Tensor with Gradient Regularization}

The regulator term must be reflected in the stress--energy tensor if it is to have physical meaning.

A minimal Korteweg-type stress consistent with the gradient energy is
\[
T^{\mu\nu}
=
(\varepsilon + p) u^\mu u^\nu
+
p g^{\mu\nu}
+
\kappa
\left(
\nabla^\mu n \nabla^\nu n
-
\frac{1}{2}
g^{\mu\nu}
\nabla_\alpha n \nabla^\alpha n
\right).
\]

The energy density $\varepsilon$ and pressure $p$ are derived from the same coarse--grained free--energy density $f(n)$ to ensure thermodynamic consistency.

With this definition,
\[
\nabla_\mu T^{\mu\nu} = 0
\]
holds provided the scalar sector and diffusion current satisfy their respective conservation equations.

\subsection{Separation of Roles}

In this formulation:

The bulk velocity $u^\mu$ governs gravitational dynamics through $T^{\mu\nu}$.

The diffusion current $j^\mu$ governs relaxation toward chemical equilibrium.

The regulator $\kappa$ appears both in $\mu$ and in $T^{\mu\nu}$, giving it an energetic interpretation rather than treating it as a purely kinematic correction.

This separation ensures that the cosmological velocity divergence $\theta$ appearing in linear perturbation theory refers strictly to the bulk motion determined by the stress--energy tensor. Chemical-potential gradients enter the perturbation system only through modifications of the effective pressure sector (and, where retained, the anisotropic stress), rather than through a redefinition of the velocity variable itself. In particular, no identification of the bulk peculiar velocity $\vec v$ with $-\nabla \mu$ is required or assumed at any stage of the derivation.

With this structure fixed, the scalar perturbation equations may be derived in longitudinal gauge without ambiguity.

\section{Linear Scalar Perturbations in Longitudinal Gauge}

We now analyze scalar perturbations about a spatially flat Friedmann--Robertson--Walker background. In longitudinal gauge, the perturbed metric takes the form
\[
ds^2
=
-(1+2\Phi)dt^2
+
a(t)^2 (1-2\Psi)\delta_{ij}dx^i dx^j,
\]
where $\Phi$ and $\Psi$ are the Bardeen potentials.

The conserved particle-number current is written
\[
J^\mu = n u^\mu + \mathcal{J}^\mu,
\]
with $u^\mu = (1-\Phi, \, \vec{v}/a)$ to first order, and $\mathcal{J}^\mu$ orthogonal to $u^\mu$.

We decompose the density as
\[
n(t,\vec{x})
=
\bar n(t)
+
\delta n(t,\vec{x}),
\]
and define the density contrast
\[
\delta \equiv \frac{\delta n}{\bar n}.
\]

\subsection{Perturbed Chemical Potential}

The chemical potential derived from the functional is
\[
\mu
=
f'(n)
-
\kappa D^2 n.
\]

Expanding to first order yields
\[
\delta \mu
=
f''(\bar n)\,\delta n
-
\kappa a^{-2}\nabla^2 \delta n.
\]

In Fourier space, this becomes
\[
\delta \mu_k
=
f''(\bar n)\,\delta n_k
+
\kappa \frac{k^2}{a^2} \delta n_k.
\]

Note that the gradient correction modifies the effective sound-speed sector in a scale-dependent way.

\subsection{Perturbed Conservation Law}

The conservation equation $\nabla_\mu J^\mu=0$ gives, to linear order,
\[
\dot{\delta}
+
\frac{1}{a}\nabla\cdot \vec{v}
-
3\dot{\Psi}
=
0.
\]

The spatial components of the current define the velocity perturbation. Using the constitutive relation
\[
\mathcal{J}^i
=
- M h^{ij}\nabla_j \mu,
\]
we obtain
\[
\vec{v}
=
-
\frac{M}{\bar n}
\nabla \delta \mu.
\]

In Fourier space,
\[
\vec{v}_k
=
-
\frac{M}{\bar n}
i \vec{k} \delta \mu_k.
\]

Substituting into the continuity equation yields
\[
\dot{\delta}_k
+
3\dot{\Psi}_k
=
-
\frac{M k^2}{a^2 \bar n}
\delta \mu_k.
\]

Inserting the expression for $\delta \mu_k$ gives
\[
\dot{\delta}_k
+
3\dot{\Psi}_k
=
-
\frac{M k^2}{a^2}
\left(
f''(\bar n)\,\delta_k
+
\kappa \frac{k^2}{a^2}\delta_k
\right).
\]

\subsection{Modified Euler Equation}

The perturbed Euler equation follows from spatial projection of $\nabla_\mu T^{\mu\nu}=0$. Including the Korteweg stress, one obtains
\[
\dot{\vec{v}}
+
H\vec{v}
=
-
\frac{1}{a}
\nabla \Phi
-
\frac{1}{a\bar n}
\nabla \delta p_{\rm eff},
\]
where the effective pressure perturbation is
\[
\delta p_{\rm eff}
=
c_s^2 \delta \rho
-
\kappa \frac{k^2}{a^2} \delta n,
\]
with $c_s^2 = \partial p/\partial \rho$ evaluated on the background.

The gradient term contributes a $k^2$-dependent correction to the pressure perturbation, and hence to the effective sound speed.

\section{Effective Growth Equation}

Combining the continuity and Euler equations in Fourier space yields a second-order equation for $\delta_k$,
\[
\ddot{\delta}_k
+
2H\dot{\delta}_k
+
\left(
\frac{c_s^2 k^2}{a^2}
-
4\pi G \bar \rho
\right)
\delta_k
+
\kappa
\frac{k^4}{a^4}
\frac{M}{\bar n}
\delta_k
=
0.
\]

The additional $k^4$ term is the signature contribution of the regulator. It modifies the Jeans instability criterion. The modified dispersion relation in the subhorizon limit becomes
\[
\omega^2
=
c_s^2 \frac{k^2}{a^2}
-
4\pi G \bar \rho
+
\alpha \frac{k^4}{a^4},
\]
where $\alpha$ collects constants proportional to $\kappa$ and $M$.

This implies three important structural consequences.

First, small-scale modes are stabilized by the $k^4$ term even when $c_s^2<0$ in the spinodal regime. The ultraviolet sector remains controlled.

Second, there exists a finite band of unstable modes determined by the condition
\[
0 < k_{\rm phys}^2 < k_{\rm crit}^2,
\]
where
\[
k_{\rm crit}^2
=
\frac{|c_s^2|}{2\alpha}
\left(
1 +
\sqrt{
1 +
\frac{16\pi G \bar \rho \alpha}{c_s^4}
}
\right).
\]

Third, the fastest-growing mode occurs at a calculable scale depending explicitly on $\kappa$.

\section{Observable Consequences}

The presence of the $k^4$ regulator modifies the transfer function at high wavenumber. In contrast to standard $\Lambda$CDM, which predicts a smooth power-law falloff modulated by baryon acoustic oscillations, this model predicts a sharp suppression beyond a regulator-controlled scale.

The two-point correlation function acquires a modified small-scale cutoff determined by $\kappa$. If $\kappa$ evolves slowly, this cutoff remains approximately fixed in physical coordinates, unlike horizon-driven inflationary scales which are tied to expansion history.

Furthermore, the regulator induces scale-dependent anisotropic stress. In general,
\[
\Phi - \Psi \neq 0,
\]
even in the absence of free-streaming radiation. This provides a direct observational wedge via weak lensing measurements, since lensing depends on the combination $\Phi+\Psi$.

The instability spectrum therefore produces three potential empirical discriminants: a preferred physical clustering scale, a high-$k$ suppression distinct from cold dark matter transfer functions, and a scale-dependent anisotropic stress signature.

\section{Structural Summary}

The model is now expressed as a closed relativistic fluid with

\[
\nabla_\mu J^\mu = 0,
\qquad
\nabla_\mu T^{\mu\nu} = 0,
\]

a coarse-grained free-energy functional with regulated second variation, and a gravitational coupling determined by explicit stress--energy contributions.

The instability mechanism is mathematically equivalent to relativistic spinodal decomposition with a finite correlation length. The negative diffusion language is replaced by loss of convexity of the free-energy density. The ultraviolet sector is controlled by a gradient-energy regulator that produces a calculable $k^4$ correction to the dispersion relation.

At this stage, the theory makes falsifiable predictions at the level of the linear matter power spectrum and gravitational slip.

\subsection{Matter-Era Transfer Function with Gradient Regulator}

We now compute an explicit, asymptotically controlled modification of the linear matter power spectrum arising from the regulator term $\kappa_{\rm eff}$ in the conservative Korteweg sector.

Working in the longitudinal gauge and restricting to the matter-dominated era with $w \simeq 0$, $\Psi \simeq \Phi$, and $\dot{\Phi} \simeq 0$, the subhorizon scalar perturbations satisfy the standard continuity--Euler system with a modified pressure sector. Incorporating the gradient regulator through
\[
\delta p_k
=
c_s^2 \bar\rho \, \delta_k
+
\frac{\kappa_{\rm eff}}{a^2} k^2 \bar\rho \, \delta_k,
\]
the combined growth equation becomes
\[
\ddot{\delta}_k
+
2H \dot{\delta}_k
+
\left(
\frac{c_s^2 k^2}{a^2}
+
\frac{\kappa_{\rm eff} k^4}{a^4}
-
4\pi G \bar\rho
\right)
\delta_k
=
0.
\]

In matter domination,
\[
a(t) \propto t^{2/3},
\qquad
H = \frac{2}{3t},
\qquad
4\pi G \bar\rho = \frac{2}{3t^2}.
\]

Substituting these relations gives
\[
\ddot{\delta}_k
+
\frac{4}{3t}\dot{\delta}_k
+
\left(
\frac{\kappa_{\rm eff} k^4}{a^4}
-
\frac{2}{3t^2}
\right)
\delta_k
=
0,
\]
where we have neglected the ordinary sound speed term $c_s^2$ for clarity in the dust-like limit.

\subsubsection*{Large-Scale Regime}

For sufficiently small comoving wavenumber $k$, the regulator term is negligible compared to the gravitational term. The equation reduces to
\[
\ddot{\delta}_k
+
\frac{4}{3t}\dot{\delta}_k
-
\frac{2}{3t^2}\delta_k
=
0,
\]
whose growing solution is
\[
D(k,a) \propto a.
\]

Thus on scales $k \ll k_J(a)$ the growth is indistinguishable from Einstein--de Sitter evolution.

\subsubsection*{Regulator-Dominated Regime}

For sufficiently large $k$, the regulator term dominates over gravity. The equation becomes
\[
\ddot{\delta}_k
+
2H\dot{\delta}_k
+
\frac{\kappa_{\rm eff} k^4}{a^4}\delta_k
\simeq
0.
\]

Because $\kappa_{\rm eff} k^4/a^4 \propto a^{-4}$ while $H^2 \propto a^{-3}$, the restoring term dominates at early times for fixed $k$. Growth is therefore suppressed once the regulator overtakes gravitational instability.

Define $a_{\rm on}(k)$ as the scale factor at which
\[
\frac{\kappa_{\rm eff} k^4}{a_{\rm on}^4}
=
4\pi G \bar\rho(a_{\rm on}).
\]

Using $\bar\rho \propto a^{-3}$ in matter domination, we obtain
\[
\frac{\kappa_{\rm eff} k^4}{a_{\rm on}^4}
\propto
a_{\rm on}^{-3},
\]
which implies
\[
a_{\rm on}
\propto
k^{-4}.
\]

Modes that enter the regulator-dominated regime early cease growing at $a_{\rm on}$. Their late-time growth factor therefore satisfies
\[
D(k,a_{\rm today})
\propto
a_{\rm on}(k)
\propto
k^{-4}.
\]

\subsubsection*{Asymptotic Transfer Function}

Define the transfer function relative to standard matter-era growth as
\[
T_\kappa(k)
=
\frac{D_\kappa(k,a_{\rm today})}{a_{\rm today}}.
\]

Then asymptotically,
\[
T_\kappa(k)
\to
1
\quad
\text{for }
k \ll k_\kappa,
\]
and
\[
T_\kappa(k)
\propto
k^{-4}
\quad
\text{for }
k \gg k_\kappa,
\]
where $k_\kappa$ is the regulator scale determined by $\kappa_{\rm eff}$.

A smooth analytic interpolation consistent with these limits is
\[
T_\kappa(k)
\simeq
\left(
1 + \frac{k^4}{k_\kappa^4}
\right)^{-1}.
\]

The matter power spectrum therefore becomes
\[
P_\kappa(k,a)
=
P_{\rm prim}(k)
\, T_\kappa(k)^2
\, a^2,
\]
which yields the asymptotic behavior
\[
P_\kappa(k)
\propto
k^{n_s}
\quad
(k \ll k_\kappa),
\]
and
\[
P_\kappa(k)
\propto
k^{n_s - 8}
\quad
(k \gg k_\kappa).
\]

\subsubsection*{Physical Interpretation}

The gradient regulator introduces a calculable high-$k$ suppression that steepens the power spectrum by an additional $k^{-8}$ factor relative to the primordial tilt in the conservative matter-era approximation. This suppression arises from a scale-dependent effective sound speed
\[
c_{\rm eff}^2(k,a)
=
\kappa_{\rm eff}\frac{k^2}{a^2},
\]
which defines a modified Jeans criterion
\[
\frac{c_{\rm eff}^2 k^2}{a^2}
\simeq
4\pi G \bar\rho.
\]

The resulting cutoff scale is therefore directly tied to the regulator parameter $\kappa_{\rm eff}$ and admits an explicit parameter-to-spectrum mapping.

\subsection{Gauge-Invariant Treatment via the Mukhanov--Sasaki Variable}

We now recast the regulated scalar sector in a manifestly gauge-invariant form. This serves two purposes. First, it eliminates any ambiguity about whether the earlier longitudinal-gauge calculation has inadvertently conflated gauge artifacts with physical growth. Second, it identifies precisely where the gradient regulator enters the canonical action and therefore how it must modify the mode equation for scalar perturbations.

We work on a spatially flat FRW background
\[
ds^2 = -dt^2 + a(t)^2 d\vec{x}^{\,2},
\]
and we consider scalar perturbations of a single effective fluid degree of freedom whose microphysics is encoded by a pressure functional that depends not only on the density but also on spatial gradients of a conserved scalar, such as number density $n$ or energy density $\rho$. In the conservative Korteweg closure, this dependence produces, at linear order, a scale-dependent pressure perturbation of the schematic form
\begin{equation}
\delta p_k
=
c_s^2\,\delta\rho_k
+
\frac{\kappa_{\rm eff}}{a^2}\,k^2\,\delta\rho_k,
\label{eq:delta_p_korteweg}
\end{equation}
where $c_s^2$ is the usual adiabatic sound speed and $\kappa_{\rm eff}>0$ is the regulator parameter (with dimensions of length squared) descending from the gradient-energy coefficient in the underlying coarse-grained functional.

\subsubsection*{Gauge-invariant curvature perturbation and effective sound speed}

Let $\mathcal{R}$ denote the comoving curvature perturbation. For a single adiabatic degree of freedom and negligible anisotropic stress at leading order, the standard quadratic action can be written in conformal time $\eta$ as
\begin{equation}
S^{(2)}
=
\frac{1}{2}
\int d\eta\, d^3x
\,
z(\eta)^2
\left[
(\mathcal{R}')^2
-
c_{\rm eff}^2(\nabla\mathcal{R})^2
\right],
\label{eq:MS_action}
\end{equation}
where primes denote $d/d\eta$ and
\begin{equation}
z^2
=
\frac{2a^2\epsilon}{c_{\rm ad}^2}
\quad\text{with}\quad
\epsilon
=
-\frac{\dot H}{H^2},
\label{eq:z_def}
\end{equation}
in the canonical single-field case. In the present effective-fluid setting, it is more robust to regard $z(\eta)$ as the standard background function determined by $\bar\rho+\bar p$ and the adiabatic sound speed $c_{\rm ad}^2=\dot{\bar p}/\dot{\bar\rho}$, while the regulator modifies the gradient sector by replacing the constant sound speed with a scale-dependent effective sound speed.

To see this explicitly, note that \eqref{eq:delta_p_korteweg} implies, for Fourier modes, the closure
\begin{equation}
\delta p_k
=
c_{\rm eff}^2(k,\eta)\,\delta\rho_k,
\qquad
c_{\rm eff}^2(k,\eta)
=
c_s^2(\eta) + \kappa_{\rm eff}\frac{k^2}{a(\eta)^2}.
\label{eq:ceff}
\end{equation}
The key point is that the regulator contributes only through the spatial-gradient sector of the quadratic action; it does not alter the definition of $\mathcal{R}$, and it does not require any gauge choice to be stated.

\subsection{Covariant Scalar Perturbations with Gradient Energy}

We now construct the scalar perturbation sector directly from the covariant stress--energy tensor derived earlier, in order to isolate precisely how the gradient regulator enters linear cosmological dynamics.

We work on a spatially flat Friedmann--Robertson--Walker background
\[
ds^2 = -dt^2 + a(t)^2 \delta_{ij} dx^i dx^j,
\qquad
u^\mu = (1,0,0,0),
\]
and perturb both the metric and the conserved density $n$ by scalar modes.

The perturbed metric in general scalar form may be written
\[
ds^2
=
-(1+2A)dt^2
+2a(t)\partial_i B\, dt\,dx^i
+a(t)^2\left[(1-2\psi)\delta_{ij}+2\partial_i\partial_j E\right]dx^i dx^j.
\]

The conserved density decomposes as
\[
n(t,\vec x) = \bar n(t) + \delta n(t,\vec x),
\]
and the four-velocity acquires a scalar perturbation through
\[
u_\mu = (-1-A,\, a\,\partial_i v).
\]

The stress--energy tensor derived from the regulated Lagrangian contains two contributions: the perfect-fluid part and the gradient-energy (Korteweg) sector. At linear order, the density and pressure perturbations take the form
\[
\delta \rho = \varepsilon'(\bar n)\,\delta n,
\qquad
\delta p = \bar n \varepsilon''(\bar n)\,\delta n,
\]
while the gradient contribution yields an additional term proportional to spatial derivatives of $\delta n$.

Explicitly, from
\[
\mathcal{L}
=
-\varepsilon(n)
-
\frac{\kappa_{\rm eff}}{2}
h^{\mu\nu}\nabla_\mu n \nabla_\nu n,
\]
the linearized pressure perturbation becomes
\[
\delta p_k
=
c_s^2\,\delta \rho_k
+
\frac{\kappa_{\rm eff}}{a^2}k^2\,\delta \rho_k,
\]
where
\[
c_s^2
=
\frac{\partial p}{\partial \rho}
=
\frac{\bar n \varepsilon''(\bar n)}{\varepsilon'(\bar n)}.
\]

The regulator therefore enters exclusively through a $k^2$-dependent correction to the pressure perturbation. No additional propagating field appears, and the number of scalar degrees of freedom remains unchanged. The effect is entirely constitutive.

Furthermore, because the gradient energy arises from a covariant Lagrangian, the modified pressure sector respects energy--momentum conservation,
\[
\nabla_\mu T^{\mu\nu}=0,
\]
ensuring that the perturbation equations derived from the Einstein equations are mutually consistent.

At this stage, the regulator has been shown to modify the linear scalar sector through a scale-dependent pressure term without introducing gauge artifacts or additional dynamical variables. We may now pass to a manifestly gauge-invariant description to identify the canonical variable governing scalar perturbations.

\subsubsection*{Mukhanov--Sasaki equation with regulator}

Define the Mukhanov--Sasaki variable $v=z\mathcal{R}$. Varying \eqref{eq:MS_action} yields the mode equation
\begin{equation}
v_k''
+
\left(
c_{\rm eff}^2(k,\eta)\,k^2
-
\frac{z''}{z}
\right)
v_k
=
0,
\label{eq:MS_mode}
\end{equation}
with $c_{\rm eff}^2$ given by \eqref{eq:ceff}. Substituting \eqref{eq:ceff} into \eqref{eq:MS_mode} exhibits the regulator as an explicit $k^4$ contribution:
\begin{equation}
v_k''
+
\left(
c_s^2(\eta)\,k^2
+
\kappa_{\rm eff}\frac{k^4}{a(\eta)^2}
-
\frac{z''}{z}
\right)
v_k
=
0.
\label{eq:MS_mode_k4}
\end{equation}
Equation \eqref{eq:MS_mode_k4} is the gauge-invariant counterpart of the growth equation derived earlier in the subhorizon matter-era approximation. The new content here is not the functional form, which is the same, but the status: the $k^4$ term is now manifestly physical because it appears in the canonical action for the gauge-invariant mode.

\subsubsection*{Hamiltonian Positivity and Absence of Ghosts}

Before analyzing subhorizon behavior, it is important to verify that the regulator does not introduce pathological degrees of freedom at the quadratic level. Because the modification enters through the spatial-gradient sector of the action \eqref{eq:MS_action}, the kinetic structure remains unchanged.

From \eqref{eq:MS_action} with scale-dependent sound speed,
\[
S^{(2)}
=
\frac{1}{2}
\int d\eta\, d^3x
\,
\left[
v'^2
-
\left(
c_s^2(\eta)(\nabla v)^2
+
\kappa_{\rm eff}\frac{(\nabla^2 v)^2}{a(\eta)^2}
-
\frac{z''}{z}v^2
\right)
\right],
\]
where we have used $v=z\mathcal{R}$ and integrated by parts to make the spatial operators explicit.

The canonical momentum is
\[
\pi_v = v',
\]
so the Hamiltonian density becomes
\[
\mathcal{H}
=
\frac{1}{2}\pi_v^2
+
\frac{1}{2}
\left(
c_s^2(\eta)(\nabla v)^2
+
\kappa_{\rm eff}\frac{(\nabla^2 v)^2}{a(\eta)^2}
-
\frac{z''}{z}v^2
\right).
\]

The regulator contributes a positive-definite quartic-gradient term provided
\[
\kappa_{\rm eff} > 0.
\]

Thus the highest-derivative spatial operator in the quadratic Hamiltonian is strictly positive. In Fourier space the gradient sector contributes
\[
\frac{1}{2}
\left(
c_s^2 k^2
+
\kappa_{\rm eff}\frac{k^4}{a^2}
\right)
|v_k|^2,
\]
which grows as $k^4$ for large $k$.

Two consequences follow immediately.

First, no Ostrogradsky instability arises, since the action remains first order in time derivatives and higher order only in spatial derivatives. The kinetic term is canonical and positive-definite.

Second, the ultraviolet dispersion relation satisfies
\[
\omega_k^2 \sim \kappa_{\rm eff}\frac{k^4}{a^2}
\quad\text{as}\quad k\to\infty,
\]
so $\omega_k^2 \to +\infty$ in the ultraviolet. The regulator therefore enforces hyperbolic well-posedness and prevents runaway growth at arbitrarily small scales.

Any instability that occurs must therefore arise from intermediate scales where the effective mass term or $c_s^2$ becomes negative, rather than from the highest-derivative operator. The theory remains ultraviolet stable by construction.


\section{Assumptions and Regime of Validity}

The construction presented above defines a regulated scalar sector embedded in a relativistic fluid background. In order to prevent overinterpretation of the linear analysis, we collect here the precise assumptions under which the derivations hold.

The free-energy functional and its gradient regularization are treated as an effective description valid at scales larger than a microphysical correlation length $\ell$, with $\kappa_{\rm eff} \sim \ell^2$ arising from coarse-graining. The derivative expansion is truncated at fourth order in spatial gradients, corresponding to the $k^4$ term in Fourier space. No higher-derivative operators are retained in the minimal core.

The background spacetime is taken to be spatially flat FRW with metric

ds^2 = -dt^2 + a(t)^2 d\vec{x}^{\,2}, 

The linear perturbation analysis assumes that the amplitude of scalar perturbations remains sufficiently small that quadratic truncation is valid. The WKB interpretation of the mode equation applies when the effective frequency $\omega_k(\eta)$ varies slowly compared to the mode period, i.e.

\left|\frac{\omega_k'}{\omega_k^2}\right| \ll 1. 

When a finite spinodal interval $a_{\rm on}<a<a_{\rm off}$ is invoked, the sign of $c_s^2(a)$ is assumed to be negative only within that interval and positive outside it. The regulator parameter $\kappa_{\rm eff}$ is assumed positive and slowly varying relative to the Hubble timescale.

Radiation, multi-fluid couplings, baryonic acoustic physics, and non-linear mode coupling are not included in the minimal conservative core. These effects may modify quantitative predictions but do not alter the structural band-limitation mechanism derived from the $k^4$ regulator term.

\subsubsection*{Subhorizon limit and matching to the matter-era growth equation}

On subhorizon scales, and in epochs where $z''/z$ varies slowly compared to the mode frequency, \eqref{eq:MS_mode_k4} admits a WKB interpretation with effective frequency
\begin{equation}
\omega_k(\eta)^2
=
c_s^2(\eta)\,k^2
+
\kappa_{\rm eff}\frac{k^4}{a(\eta)^2}
-
\frac{z''}{z}.
\label{eq:omega}
\end{equation}
In the dust-like matter era, one typically has $c_s^2\simeq 0$ and $z''/z \sim a^2H^2$ (up to order-one factors). Thus the regulator dominates the gradient sector at sufficiently large $k$, producing a scale-dependent stiffening that prevents gravitational growth. In the complementary regime of small $k$, the $k^4$ term is negligible and one recovers standard growth.

If one wishes to connect directly to density contrast evolution, one may use the usual subhorizon relations between $\mathcal{R}$, the Newtonian potential, and $\delta$ in a specified matter model. The important structural claim, however, is already contained in \eqref{eq:MS_mode_k4}: the regulator induces a $k^4$ term in a gauge-invariant equation of motion, and therefore any high-$k$ suppression or band-limitation derived from it is not a gauge artifact.

\subsubsection*{Mapping to Density Contrast in the Subhorizon Regime}

Although equation \eqref{eq:MS_mode_k4} is written for the gauge-invariant variable $v=z\mathcal{R}$, the observable matter power spectrum is expressed in terms of the density contrast $\delta$. It is therefore useful to make explicit the relation between $\mathcal{R}$ and $\delta$ in the regime relevant for structure formation.

On subhorizon scales during matter domination, the comoving curvature perturbation $\mathcal{R}$ is related to the Newtonian potential $\Phi$ by
\[
\mathcal{R}
\simeq
-\Phi
-
\frac{H}{\dot H}
\left(
\dot\Phi + H\Phi
\right).
\]
In the quasi-static limit appropriate to growing-mode evolution in a dust-like era, $\dot\Phi \simeq 0$ and $\mathcal{R}$ becomes approximately constant for modes well outside the regulator-dominated band. 

For subhorizon modes satisfying $k \gg aH$, the Poisson equation gives
\[
\frac{k^2}{a^2}\Phi_k
\simeq
4\pi G \bar\rho\,\delta_k.
\]
Hence
\[
\delta_k
\simeq
\frac{k^2}{4\pi G \bar\rho\,a^2}\,\Phi_k.
\]

Since $\Phi_k$ is algebraically related to $\mathcal{R}_k$ in this regime, any modification of the mode equation for $\mathcal{R}_k$ induced by the $k^4$ regulator term propagates directly into the evolution equation for $\delta_k$. In particular, when the regulator stiffens the dispersion relation so that
\[
\omega_k^2
=
c_s^2 k^2
+
\kappa_{\rm eff}\frac{k^4}{a^2}
-
\frac{z''}{z}
\]
becomes positive and large for sufficiently high $k$, the associated suppression of $v_k$ implies suppression of $\Phi_k$ and therefore of $\delta_k$.

Thus the scale-dependent regulator imprint in \eqref{eq:MS_mode_k4} is not confined to curvature variables but directly modifies the density-contrast growth law in the subhorizon regime.

\subsubsection*{Interpretive consequence}

The conservative gradient regulator can therefore be stated in the most reviewer-legible way as follows. The instability or growth properties of the scalar sector are encoded by the sign structure of $\omega_k^2$. The regulator ensures ultraviolet well-posedness by enforcing $\omega_k^2 \to +\infty$ as $k\to\infty$ at fixed $\eta$, since the leading term is $\kappa_{\rm eff}k^4/a^2$ with $\kappa_{\rm eff}>0$. Any intermediate-scale growth regime, whether gravitational or spinodal in origin, is therefore necessarily band-limited and cannot run away to arbitrarily high wavenumber within the linear theory.

This completes the gauge-invariant reformulation of the regulated scalar sector and provides a canonical bridge between the fluid-level closure \eqref{eq:delta_p_korteweg} and an explicit, testable modification of the scalar perturbation mode equation \eqref{eq:MS_mode_k4}.

\section{Regulated Transfer Function and High-$k$ Tail in the Linear Power Spectrum}
\label{sec:transfer_tail}

This section derives the simplest observable consequence of the regulated model: an explicit, parameter-controlled suppression of small-scale power. The purpose is not to claim a full \(\Lambda\)CDM-quality Boltzmann solution, but to exhibit a distinctive asymptotic tail that follows directly from the \(\kappa_{\rm eff}\)-regularization and therefore cannot be removed by rephrasing the interpretation. The derivation is carried out under a controlled set of approximations and then translated into a transfer-function statement for \(P(k)\).

\subsection{Set-up: conservative perturbations with a gradient-stress closure}

We work in longitudinal gauge with scalar potentials \(\Phi\) and \(\Psi\), and we assume vanishing vorticity. The bulk velocity is encoded by the usual divergence variable \(\theta \equiv a^{-1}\nabla\cdot \vec v\). The governing linearized fluid system in Fourier space takes the standard form
\begin{align}
\dot\delta_k
&=
-(1+w)\big(\theta_k - 3\dot\Psi_k\big)
-3H\left(\frac{\delta p_k}{\bar\rho}-w\delta_k\right),
\label{eq:cont_longitudinal}
\\
\dot\theta_k
&=
-H(1-3w)\theta_k
+\frac{k^2}{a^2}\left(\Phi_k + \frac{\delta p_k}{(1+w)\bar\rho}\right)
-\frac{k^2}{a^2}\sigma_k,
\label{eq:euler_longitudinal}
\end{align}
where overdots denote derivatives with respect to cosmic time \(t\), \(w=\bar p/\bar\rho\), and \(\sigma_k\) is the scalar anisotropic-stress potential. The conservative regulated closure enters through \(\delta p_k\) (and optionally \(\sigma_k\)) derived from a Korteweg-type gradient energy. In the minimal conservative core one takes \(\sigma_k=0\) and adopts a scale-dependent pressure perturbation
\begin{equation}
\delta p_k
=
c_s^2\,\bar\rho\,\delta_k
+
\frac{\kappa_{\rm eff}}{a^2}\,k^2\,\bar\rho\,\delta_k,
\label{eq:dp_korteweg}
\end{equation}
where \(c_s^2\) is the usual adiabatic sound speed (possibly time-dependent), and \(\kappa_{\rm eff}>0\) is a length-squared parameter encoding the energetic cost of gradients in the conserved density. The sign and magnitude of \(\kappa_{\rm eff}\) are part of the model definition; \(\kappa_{\rm eff}>0\) is the ultraviolet regulator ensuring that the highest-derivative contribution to the linear operator remains stabilizing.

\subsection{Subhorizon reduction and the modified growth equation}

To isolate the small-scale transfer tail, we take the subhorizon quasi-static limit in the matter-dominated era, which is the simplest regime where analytic asymptotics are clean. Concretely, we set \(w\simeq 0\), \(\dot\Phi_k\simeq 0\), \(\Psi_k\simeq \Phi_k\), and use the Poisson equation
\begin{equation}
\frac{k^2}{a^2}\Phi_k \simeq 4\pi G\,\bar\rho\,\delta_k.
\label{eq:poisson_subhorizon}
\end{equation}
With \(\sigma_k=0\) and \eqref{eq:dp_korteweg}, equations \eqref{eq:cont_longitudinal}--\eqref{eq:euler_longitudinal} combine into a second-order equation for \(\delta_k\). Differentiating \eqref{eq:cont_longitudinal} and substituting \eqref{eq:euler_longitudinal} yields, after standard algebra,
\begin{equation}
\ddot\delta_k
+
2H\dot\delta_k
+
\left(
\frac{c_s^2 k^2}{a^2}
+
\frac{\kappa_{\rm eff}k^4}{a^4}
-
4\pi G\bar\rho
\right)\delta_k
\simeq 0.
\label{eq:growth_kappa}
\end{equation}
Equation \eqref{eq:growth_kappa} is the conservative, frame-clean statement of the regulated linear theory. The \(k^4\)-term is not introduced as diffusion; it is the linear imprint of a gradient-energy contribution to the stress. It therefore modifies the growth of density fluctuations by introducing a scale-dependent restoring force that becomes dominant at sufficiently large comoving \(k\).

\subsection{A modified Jeans scale and a finite instability window}

A convenient way to summarize the competition between gravity and gradient regularization is via an effective squared sound speed,
\begin{equation}
c_{\rm eff}^2(k,a)
=
c_s^2
+
\kappa_{\rm eff}\frac{k^2}{a^2}.
\label{eq:ceff_def}
\end{equation}
The instantaneous Jeans criterion compares the pressure-restoring term to gravity:
\begin{equation}
\frac{c_{\rm eff}^2(k,a)\,k^2}{a^2}
\gtrsim
4\pi G\bar\rho.
\label{eq:jeans_criterion}
\end{equation}
Defining the critical comoving Jeans wavenumber \(k_J(a)\) by equality gives an explicit quadratic equation in \(k_J^2/a^2\),
\begin{equation}
c_s^2 \frac{k_J^2}{a^2}
+
\kappa_{\rm eff}\frac{k_J^4}{a^4}
=
4\pi G\bar\rho,
\label{eq:jeans_quadratic}
\end{equation}
whose physically relevant solution is
\begin{equation}
\frac{k_J^2(a)}{a^2}
=
\frac{-c_s^2 + \sqrt{c_s^4 + 16\pi G\bar\rho\,\kappa_{\rm eff}}}{2\kappa_{\rm eff}}.
\label{eq:kJ_solution}
\end{equation}
When \(c_s^2\ge 0\), \eqref{eq:kJ_solution} simply defines the scale above which gravitational growth is suppressed. When a finite spinodal epoch is permitted, meaning that \(c_s^2(a)<0\) for \(a_{\rm on}<a<a_{\rm off}\), the same expression shows a structurally important feature: ultraviolet stability is preserved by \(\kappa_{\rm eff}>0\), but an intermediate band of modes can become unstable because the negative \(c_s^2\)-term drives growth at moderate \(k\) while the positive \(k^4\)-term stabilizes sufficiently large \(k\). In other words, the regulated theory can support an ``irruption band'' without backward-parabolic pathology; the band edges are determined by \eqref{eq:jeans_quadratic} and are finite whenever \(\kappa_{\rm eff}>0\).

\subsection{Matter-dominated high-$k$ asymptotics and power suppression}

We now extract the high-\(k\) behavior of solutions to \eqref{eq:growth_kappa} during matter domination. In an Einstein--de Sitter background one has
\begin{equation}
a(t) \propto t^{2/3},
\qquad
H=\frac{2}{3t},
\qquad
\bar\rho=\frac{1}{6\pi G t^2}.
\label{eq:eds_background}
\end{equation}
For sufficiently large comoving \(k\), and outside any finite spinodal interval, the regulator term dominates the bracket in \eqref{eq:growth_kappa}:
\begin{equation}
\ddot\delta_k + 2H\dot\delta_k + \frac{\kappa_{\rm eff}k^4}{a^4}\,\delta_k \simeq 0.
\label{eq:highk_approx_t}
\end{equation}
It is convenient to pass to conformal time \(\eta\) defined by \(d\eta=dt/a(t)\), and to denote \(d/d\eta\) by primes. Using \(\mathcal{H}=a'/a\) and \(\dot{(\cdot)} = a^{-1}(\cdot)'\), equation \eqref{eq:highk_approx_t} becomes
\begin{equation}
\delta_k'' + \mathcal{H}\delta_k' + \kappa_{\rm eff}\frac{k^4}{a^2}\,\delta_k \simeq 0.
\label{eq:highk_approx_eta}
\end{equation}
In matter domination one has \(a(\eta)\propto \eta^2\) and \(\mathcal{H}=2/\eta\). Writing \(a(\eta)=a_\star(\eta/\eta_\star)^2\) gives
\begin{equation}
\delta_k'' + \frac{2}{\eta}\delta_k'
+
\kappa_{\rm eff}\frac{k^4}{a_\star^2}\left(\frac{\eta_\star}{\eta}\right)^4
\delta_k
\simeq 0.
\label{eq:highk_eta_powerlaw}
\end{equation}
Define the rescaled variable \(\delta_k(\eta)=\eta^{-1}u_k(\eta)\). A direct computation yields
\begin{equation}
u_k'' + \left(
\kappa_{\rm eff}\frac{k^4}{a_\star^2}\left(\frac{\eta_\star}{\eta}\right)^4
-
\frac{2}{\eta^2}
\right)u_k \simeq 0.
\label{eq:u_equation}
\end{equation}
For large \(k\) the \(\eta^{-4}\) term dominates over \(2/\eta^2\) except at very late times. In that regime the leading-order equation
\begin{equation}
u_k'' + \Omega_k^2(\eta)\,u_k \simeq 0,
\qquad
\Omega_k(\eta)=
\sqrt{\kappa_{\rm eff}}\frac{k^2}{a_\star}\left(\frac{\eta_\star}{\eta}\right)^2
\label{eq:omega_def}
\end{equation}
is an adiabatically varying oscillator. The WKB approximation applies, giving
\begin{equation}
u_k(\eta)
\simeq
\frac{1}{\sqrt{\Omega_k(\eta)}}
\left[
C_1 \exp\!\left(i\int^\eta \Omega_k(\tilde\eta)\,d\tilde\eta\right)
+
C_2 \exp\!\left(-i\int^\eta \Omega_k(\tilde\eta)\,d\tilde\eta\right)
\right].
\label{eq:wkb_u}
\end{equation}
Since \(\Omega_k(\eta)\propto \eta^{-2}\), the phase integral converges:
\begin{equation}
\int^\eta \Omega_k(\tilde\eta)\,d\tilde\eta
=
\sqrt{\kappa_{\rm eff}}\frac{k^2}{a_\star}\eta_\star^2
\int^\eta \tilde\eta^{-2}\,d\tilde\eta
=
\sqrt{\kappa_{\rm eff}}\frac{k^2}{a_\star}\eta_\star^2
\left(\frac{1}{\eta_{\rm in}}-\frac{1}{\eta}\right),
\label{eq:phase_converges}
\end{equation}
for some initial time \(\eta_{\rm in}\). The amplitude factor satisfies \(\Omega_k^{-1/2}\propto \eta\), hence \(u_k(\eta)\) grows at most linearly in \(\eta\), while \(\delta_k(\eta)=\eta^{-1}u_k(\eta)\) approaches a bounded oscillatory profile. This establishes the essential tail statement: sufficiently high-\(k\) modes do not undergo gravitational growth in the matter era; their growth is shut off by the regulator, and their amplitude is asymptotically bounded (up to slow envelope corrections coming from the neglected \(-2/\eta^2\) term and from any small physical viscosity).

\subsection{Transfer-function form and the asymptotic suppression law}

To translate this into a power-spectrum statement, define the \(k\)-dependent growth factor \(D_\kappa(k,a)\) as the amplitude of the growing mode of \(\delta_k\) normalized to unity at some early reference scale factor \(a_i\) in matter domination. Then the linear matter power spectrum may be written as
\begin{equation}
P_\kappa(k,a)
=
P_{\rm prim}(k)\,|T_\kappa(k)|^2\,D_\kappa^2(k,a),
\label{eq:Pk_factorization}
\end{equation}
where \(T_\kappa(k)\) contains pre-matter-era processing (or, in a minimal comparison, may be set to unity if one begins evolution at \(a_i\) deep in matter domination). The novel content in the regulated model is that \(D_\kappa(k,a)\) becomes scale dependent even in matter domination.

Equation \eqref{eq:growth_kappa} and the asymptotics above imply the following robust two-regime behavior. For \(k\ll k_J(a)\), the regulator is negligible and one recovers the standard growing mode \(D_\kappa(k,a)\simeq a/a_i\) up to small pressure corrections. For \(k\gg k_J(a)\), the regulator dominates and the growth factor saturates,
\begin{equation}
D_\kappa(k,a)
\simeq
D_\kappa(k,a_{\rm sat})
\quad\text{for}\quad k\gg k_J(a),
\label{eq:D_saturates}
\end{equation}
where \(a_{\rm sat}\) is the epoch at which the \(k^4\)-term overtakes the gravitational term in \eqref{eq:growth_kappa} for that mode. As a consequence, relative to an unregulated matter-growth law \(D_{\rm EdS}(a)=a/a_i\), one obtains a high-\(k\) suppression
\begin{equation}
\frac{P_\kappa(k,a)}{P_{\rm EdS}(k,a)}
\sim
\left(\frac{D_\kappa(k,a)}{a/a_i}\right)^2
\ll 1
\quad\text{for}\quad k\gg k_J(a),
\label{eq:Pk_suppression_ratio}
\end{equation}
with \(k_J(a)\) given explicitly by \eqref{eq:kJ_solution}. This is the key reviewer-legible wedge: the model predicts a regulator-controlled cutoff whose functional form is set by \(\kappa_{\rm eff}\) through \eqref{eq:kJ_solution}, producing a characteristic small-scale suppression distinct from collisionless CDM unless additional microphysical parameters are introduced to mimic the same \(k\)-dependence.

In the next section we will treat the finite spinodal interval \(a_{\rm on}<a<a_{\rm off}\) in which \(c_s^2(a)\) becomes negative while \(\kappa_{\rm eff}>0\), and we will show that the same conservative scaffold yields a bounded instability band with calculable band edges and a fastest-growing comoving wavenumber. That refinement turns ``suppression'' into ``irruption'' in the strict dynamical sense, while preserving well-posedness and ultraviolet control.

\section{Finite Spinodal Window and Band-Limited Irruption}

We now consider a finite interval in which the effective adiabatic sound speed becomes negative while the regulator remains strictly positive. Concretely, let
\begin{equation}
c_s^2(a) < 0
\quad \text{for} \quad
a_{\rm on} < a < a_{\rm off},
\qquad
\kappa_{\rm eff} > 0 \ \text{for all } a.
\end{equation}
Outside this interval the system is linearly stable (modulo gravitational Jeans growth). Inside the interval the compressibility becomes negative and a regulated instability can occur.

Starting from the conservative growth equation obtained previously in matter domination,
\begin{equation}
\ddot{\delta}_k + 2H \dot{\delta}_k
+ \left(
\frac{c_s^2(a)\,k^2}{a^2}
+ \frac{\kappa_{\rm eff} k^4}{a^4}
- 4\pi G \bar\rho
\right)\delta_k = 0,
\label{eq:growth_full}
\end{equation}
we first isolate the intrinsic instability by neglecting the gravitational term and restricting attention to subhorizon modes during the spinodal window. Equation \eqref{eq:growth_full} reduces to
\begin{equation}
\ddot{\delta}_k + 2H \dot{\delta}_k
+ \left(
\frac{c_s^2(a)\,k^2}{a^2}
+ \frac{\kappa_{\rm eff} k^4}{a^4}
\right)\delta_k = 0.
\label{eq:growth_spinodal}
\end{equation}

Define
\begin{equation}
A(a) := -c_s^2(a) > 0
\quad \text{on} \quad (a_{\rm on},a_{\rm off}).
\end{equation}
Then the effective squared frequency is
\begin{equation}
\omega_k^2(a)
=
-\frac{A(a)\,k^2}{a^2}
+ \frac{\kappa_{\rm eff} k^4}{a^4}.
\end{equation}

Instability occurs when $\omega_k^2(a)<0$, i.e.
\begin{equation}
\frac{\kappa_{\rm eff} k^4}{a^4}
<
\frac{A(a)\,k^2}{a^2}.
\end{equation}
Cancelling a factor of $k^2/a^2$ gives the band condition
\begin{equation}
0 < \frac{k^2}{a^2} < \frac{A(a)}{\kappa_{\rm eff}}.
\end{equation}
Hence unstable modes satisfy
\begin{equation}
0 < k < k_{\rm max}(a),
\qquad
k_{\rm max}^2(a)
=
\frac{A(a)}{\kappa_{\rm eff}}\,a^2.
\end{equation}

The ultraviolet sector remains stable because the $k^4$ term dominates for sufficiently large $k$.

To determine the fastest-growing mode, approximate the growth rate (ignoring Hubble friction when subdominant) by
\begin{equation}
\gamma_k^2(a)
\approx
\frac{A(a)\,k^2}{a^2}
- \frac{\kappa_{\rm eff} k^4}{a^4}.
\end{equation}
Maximizing with respect to $k$,
\begin{equation}
\frac{\partial}{\partial k}
\left(
\frac{A k^2}{a^2}
- \frac{\kappa_{\rm eff} k^4}{a^4}
\right)
=
\frac{2A k}{a^2}
- \frac{4\kappa_{\rm eff} k^3}{a^4}
= 0,
\end{equation}
which for $k\neq 0$ yields
\begin{equation}
k_*^2(a)
=
\frac{A(a)}{2\kappa_{\rm eff}}\,a^2.
\end{equation}

The corresponding physical wavenumber is
\begin{equation}
k_{\rm phys,*}^2(a)
=
\frac{k_*^2}{a^2}
=
\frac{A(a)}{2\kappa_{\rm eff}}.
\end{equation}
If $A(a)$ is approximately constant across the interval, the instability selects a fixed physical length scale
\begin{equation}
\lambda_*
=
\frac{2\pi}{k_{\rm phys,*}}
=
2\pi
\sqrt{\frac{2\kappa_{\rm eff}}{A}}.
\end{equation}

The finite duration of the spinodal window implies that only modes satisfying the band condition for a nonzero time interval undergo amplification. The approximate accumulated growth exponent is
\begin{equation}
\Gamma_k
=
\int_{t_{\rm on}}^{t_{\rm off}}
\left[
\sqrt{
\frac{A(a)\,k^2}{a^2}
- \frac{\kappa_{\rm eff} k^4}{a^4}
}
- H
\right] dt,
\end{equation}
so that modes near $k_*$ receive maximal enhancement, while modes near the band edges are marginal.

The resulting late-time power spectrum therefore acquires a feature centered near $k_*$ whose width and amplitude depend on the depth and duration of the negative-$c_s^2$ interval. Throughout the evolution the highest spatial derivative remains fourth order with positive coefficient $\kappa_{\rm eff}$, ensuring well-posedness and ultraviolet control.

\section{Combined Jeans--Spinodal Instability Structure}

We now restore the gravitational sector and analyze the full linear growth equation during the finite spinodal interval. Starting from
\begin{equation}
\ddot{\delta}_k + 2H \dot{\delta}_k
+
\left(
\frac{c_s^2(a)\,k^2}{a^2}
+
\frac{\kappa_{\rm eff} k^4}{a^4}
-
4\pi G \bar\rho
\right)\delta_k
=
0,
\label{eq:full_growth_grav}
\end{equation}
we again define
\begin{equation}
A(a) := -c_s^2(a) > 0
\qquad \text{for} \qquad
a_{\rm on} < a < a_{\rm off}.
\end{equation}

Inside the spinodal window, the effective frequency becomes
\begin{equation}
\omega_k^2(a)
=
-
\frac{A(a)\,k^2}{a^2}
+
\frac{\kappa_{\rm eff} k^4}{a^4}
-
4\pi G \bar\rho.
\end{equation}

Instability occurs when $\omega_k^2(a) < 0$, i.e.
\begin{equation}
\frac{\kappa_{\rm eff} k^4}{a^4}
-
\frac{A(a)\,k^2}{a^2}
-
4\pi G \bar\rho
<
0.
\end{equation}

Multiplying through by $a^4$ gives the quartic inequality
\begin{equation}
\kappa_{\rm eff} k^4
-
A(a)\,a^2 k^2
-
4\pi G \bar\rho\,a^4
<
0.
\end{equation}

Define $x := k^2$. The instability condition becomes a quadratic inequality
\begin{equation}
\kappa_{\rm eff} x^2
-
A(a)\,a^2 x
-
4\pi G \bar\rho\,a^4
<
0.
\end{equation}

The corresponding quadratic equation
\begin{equation}
\kappa_{\rm eff} x^2
-
A(a)\,a^2 x
-
4\pi G \bar\rho\,a^4
=
0
\end{equation}
has roots
\begin{equation}
x_{\pm}(a)
=
\frac{
A(a)\,a^2
\pm
\sqrt{
A(a)^2 a^4
+
16\pi G \bar\rho\,\kappa_{\rm eff}\,a^4
}
}{
2\kappa_{\rm eff}
}.
\end{equation}

Since $x = k^2 \ge 0$, instability occurs for
\begin{equation}
x_-(a) < k^2 < x_+(a).
\end{equation}

One finds that $x_-(a)$ is negative (because the constant term is negative), so the physical instability band is
\begin{equation}
0 < k^2 < x_+(a).
\end{equation}

Thus the upper band edge is
\begin{equation}
k_{\rm max}^2(a)
=
\frac{
A(a)\,a^2
+
\sqrt{
A(a)^2 a^4
+
16\pi G \bar\rho\,\kappa_{\rm eff}\,a^4
}
}{
2\kappa_{\rm eff}
}.
\end{equation}

Two limiting regimes are instructive.

\subsection*{Weak Gravity Limit}

If
\begin{equation}
16\pi G \bar\rho\,\kappa_{\rm eff}
\ll
A(a)^2,
\end{equation}
then
\begin{equation}
k_{\rm max}^2(a)
\approx
\frac{A(a)}{\kappa_{\rm eff}}\,a^2,
\end{equation}
which reproduces the purely spinodal band derived previously.

\subsection*{Strong Gravity Limit}

If instead
\begin{equation}
16\pi G \bar\rho\,\kappa_{\rm eff}
\gg
A(a)^2,
\end{equation}
then
\begin{equation}
k_{\rm max}^2(a)
\approx
a^2
\sqrt{
\frac{4\pi G \bar\rho}{\kappa_{\rm eff}}
}.
\end{equation}

In this regime, gravity enlarges the unstable band relative to the pure spinodal case.

\subsection*{Fastest Growing Mode}

To identify the fastest growing mode, approximate the instantaneous growth rate (neglecting Hubble friction for subhorizon modes) as
\begin{equation}
\gamma_k^2(a)
=
4\pi G \bar\rho
+
\frac{A(a)\,k^2}{a^2}
-
\frac{\kappa_{\rm eff} k^4}{a^4}.
\end{equation}

Differentiating with respect to $k$ and setting to zero yields
\begin{equation}
\frac{\partial \gamma_k^2}{\partial k}
=
\frac{2A(a)k}{a^2}
-
\frac{4\kappa_{\rm eff}k^3}{a^4}
=
0,
\end{equation}
so that
\begin{equation}
k_*^2(a)
=
\frac{A(a)}{2\kappa_{\rm eff}}\,a^2.
\end{equation}

Remarkably, the fastest-growing comoving mode is independent of the gravitational term at leading order; gravity shifts the band edges but does not shift the location of the maximum of the quadratic-minus-quartic structure.

The corresponding physical wavenumber is
\begin{equation}
k_{\rm phys,*}^2
=
\frac{A(a)}{2\kappa_{\rm eff}},
\end{equation}
so a constant negative compressibility across the spinodal interval produces a fixed physical irruption scale.

\subsection*{Interpretation}

The combined system therefore exhibits a bounded instability band whose upper edge is modified by gravitational coupling but whose intrinsic scale selection is governed by the ratio $A(a)/\kappa_{\rm eff}$. Ultraviolet stability is preserved by the positive fourth-order term, while infrared behavior is controlled by gravitational amplification. The instability is therefore neither pure Jeans collapse nor pure phase separation, but a regulated hybrid mechanism with calculable band structure and a well-defined fastest-growing mode.


\section{Dimensional Analysis and Characteristic Scales}

The regulator parameter $\kappa_{\rm eff}$ has dimensions of length squared. In natural units with $c=1$,

[\kappa_{\rm eff}] = L^2. \ell \sim \sqrt{\kappa_{\rm eff}}. 

In the presence of an effective sound speed $c_s^2$, the characteristic physical wavenumber at which the gradient regulator balances the pressure sector is determined by

c_s^2 k_{\rm phys}^2 \sim \kappa_{\rm eff} k_{\rm phys}^4, k_{\rm phys,*}^2 \sim \frac{|c_s^2|}{\kappa_{\rm eff}}. \lambda_* \sim \frac{2\pi}{k_{\rm phys,*}} \sim 2\pi \sqrt{\frac{\kappa_{\rm eff}}{|c_s^2|}}. 

During a spinodal epoch in which $c_s^2<0$, this same relation determines the fastest-growing physical mode in linear theory. In the conservative FRW embedding, the comoving fastest mode evolves according to

k_*^2(a) = a^2 \frac{|c_s^2(a)|}{2\kappa_{\rm eff}}, 

The modified Jeans condition,

c_s^2 \frac{k^2}{a^2} + \kappa_{\rm eff}\frac{k^4}{a^4} = 4\pi G \bar\rho, \omega_k^2 \sim \kappa_{\rm eff}\frac{k^4}{a^2} \to +\infty, 

These relations demonstrate that the instability scale is not arbitrary but controlled by the ratio $|c_s^2|/\kappa_{\rm eff}$ and by background density. Observable structure therefore encodes regulator and compressibility parameters directly.

\section{Phase Structure and Transition to Lamphron--Lamphrodyne Regimes}

The regulated spinodal analysis of the preceding section establishes that scalar growth arises from a finite instability band determined by the curvature of the coarse-grained free-energy functional and the ultraviolet regulator $\kappa_{\rm eff}$. In particular, the instability condition is governed by the sign of the second variation of the effective thermodynamic potential and remains dynamically well-posed due to the fourth-order stabilizing term. The fastest-growing mode is fixed by the ratio $A(a)/\kappa_{\rm eff}$, and the ultraviolet sector remains damped for all parameter values with $\kappa_{\rm eff}>0$.

The lamphron–lamphrodyne refinement that follows does not introduce a new dynamical ingredient but rather recasts the previously derived instability structure in a local geometric formulation. Specifically, the quantity
\[
\chi := -\beta \Delta_g S
\]
is a curvature-dependent effective mass contribution that can be interpreted as the local projection of the second variation of the entropy functional onto the scalar sector. Within the regulated framework, $\chi$ encodes the sign of the entropic compressibility at a given spacetime point.

During epochs in which the coarse-grained compressibility is positive, corresponding to $f''>0$ in the density formulation or equivalently to $\chi<0$ in the curvature formulation, the system lies in the lamphron regime. In this phase, the principal part of the effective linear operator remains parabolic, and all Fourier modes outside the gravitationally allowed Jeans window are damped. The regulator enforces ultraviolet control, and the homogeneous configuration is linearly stable.

When the system enters the finite spinodal interval $a_{\rm on}<a<a_{\rm off}$, the compressibility becomes negative in a bounded epoch. In the curvature formulation this corresponds to the emergence of regions in which $\chi>0$ on a set of nonzero measure. These regions define lamphrodyne domains. The local operator
\[
\mathcal{L}_{\text{eff}} = c^2 \Delta_g + \alpha + \chi
\]
then acquires an indefinite second variation in those domains, but the highest-order term remains elliptic because $c^2>0$ and $\kappa_{\rm eff}>0$ at the level of the conserved density dynamics. Consequently, the transition is not a literal backward-parabolic pathology; it is a controlled sign inversion in the lower-order sector embedded within a fourth-order parabolic scaffold.

The distinction between lamphron and lamphrodyne states therefore corresponds precisely to whether the local entropy curvature drives the system toward convex free-energy behavior or into a concave spinodal regime. In the density formulation this is expressed by the sign of $f''(\bar n)$; in the curvature formulation it is encoded by the sign of $\chi$. The two descriptions are mathematically equivalent provided $S=S[\rho]$ is understood as a coarse-grained functional of a conserved density.

This reformulation permits the lamphron--lamphrodyne dichotomy to be interpreted as a phase transition in the effective thermodynamic geometry of the scalar--vector--entropy system. The band-limited instability derived earlier supplies the global spectral structure, while the lamphrodyne definition identifies the local geometric criterion for the onset of growth. The regulator scale $\kappa_{\rm eff}$ ensures that any such transition preserves well-posedness and prevents ultraviolet divergence.

We now introduce the lamphron and lamphrodyne states explicitly as dynamical phases of the regulated scalar--entropy system.

\section{Equivalence Between Scalar--Entropy and Conserved Density Formulations}

The regulated spinodal construction was formulated above in terms of a conserved scalar density $n$ (or $\rho$) with a coarse--grained free--energy functional
\[
\mathcal{F}[n]
=
\int_{\Sigma_t} \sqrt{\gamma}
\left(
f(n)
+
\frac{\kappa}{2} D_i n D^i n
\right)
\, d^3x,
\]
with associated chemical potential
\[
\mu = \frac{\delta \mathcal{F}}{\delta n}
=
f'(n) - \kappa D^2 n.
\]
The instability criterion was shown to be governed by the sign of the second variation,
\[
\delta^2 \mathcal{F}
=
\int
\left(
f''(\bar n)\, (\delta n)^2
+
\kappa D_i \delta n D^i \delta n
\right),
\]
with spinodal growth occurring when $f''(\bar n)<0$ while $\kappa>0$ maintains ultraviolet control.

We now demonstrate that the scalar--entropy formulation introduced later is not an independent dynamical hypothesis but a reparameterization of this same structure.

\subsection{Scalar Reparameterization}

Let $\phi$ denote a monotone function of the conserved density,
\[
\phi = \Phi(n),
\qquad
\Phi'(n)>0.
\]
Under such a transformation, the gradient term becomes
\[
D_i n D^i n
=
\frac{1}{(\Phi'(n))^2}
D_i \phi D^i \phi,
\]
and the free--energy density transforms as
\[
f(n)
=
\tilde f(\phi),
\qquad
\tilde f'(\phi)
=
\frac{f'(n)}{\Phi'(n)}.
\]
The functional can therefore be rewritten entirely in terms of $\phi$:
\[
\mathcal{F}[\phi]
=
\int_{\Sigma_t} \sqrt{\gamma}
\left(
\tilde f(\phi)
+
\frac{\tilde\kappa(\phi)}{2}
D_i \phi D^i \phi
\right)
\, d^3x,
\]
with
\[
\tilde\kappa(\phi)
=
\frac{\kappa}{(\Phi'(n))^2}.
\]

Thus the scalar field $\phi$ inherits both the metastable curvature structure and the regulator from the density formulation.

\subsection{Entropy Functional Interpretation}

If one introduces a coarse--grained entropy density $S(\phi)$ and defines an effective Helmholtz functional
\[
\mathcal{F}[\phi]
=
\int
\left(
U(\phi) - T S(\phi)
+
\frac{\kappa}{2} D_i \phi D^i \phi
\right),
\]
then
\[
f(\phi) = U(\phi) - T S(\phi),
\]
and the instability criterion becomes
\[
f''(\bar\phi)
=
U''(\bar\phi) - T S''(\bar\phi) < 0.
\]
The sign reversal responsible for irruption is therefore the negativity of the entropy--modified compressibility,
not a reversal of the highest--order spatial operator.

\subsection{Equivalence of Linearized Growth Rates}

Linearizing either formulation about a homogeneous background yields
\[
\partial_t \delta\phi_k
=
M
\left(
|f''(\bar\phi)|\frac{k^2}{a^2}
-
\kappa \frac{k^4}{a^4}
\right)
\delta\phi_k,
\]
which is identical to the dispersion relation obtained from the density formulation.

The lamphron--lamphrodyne refinement that follows should therefore be understood as a geometric re-expression of the same regulated spinodal structure, not as the introduction of new degrees of freedom.

\section{Korteweg Stress and Conservative Gradient Energy}

The regulator introduced through the gradient term in the free-energy functional must appear consistently in the stress--energy tensor if the model is to remain conservative and covariantly closed. In this section we make that identification explicit and derive the corresponding modification of the scalar perturbation sector.

We begin from the spatially projected free-energy functional on hypersurfaces orthogonal to $u^\mu$,
\[
\mathcal{F}[n]
=
\int_{\Sigma_t} \sqrt{\gamma}
\left(
f(n)
+
\frac{\kappa}{2} D_i n D^i n
\right)
\, d^3x,
\]
where $D_i$ is the spatial covariant derivative defined through the projector
\[
h_{\mu\nu} = g_{\mu\nu} + u_\mu u_\nu.
\]

To incorporate the gradient regulator at the level of gravity coupling, we promote the gradient term to a covariant contribution in the matter Lagrangian density,
\[
\mathcal{L}
=
\sqrt{-g}
\left[
\mathcal{P}(n)
-
\frac{\kappa}{2}
h^{\mu\nu}
\nabla_\mu n
\nabla_\nu n
\right].
\]

Varying with respect to $g_{\mu\nu}$ yields a stress--energy tensor of Korteweg type,
\[
T^{\mu\nu}
=
(\varepsilon + p) u^\mu u^\nu
+
p g^{\mu\nu}
+
\kappa
\left(
\nabla^\mu n \nabla^\nu n
-
\frac{1}{2}
g^{\mu\nu}
\nabla_\alpha n \nabla^\alpha n
\right)
+
\Pi^{\mu\nu}_{\text{proj}},
\]
where $\Pi^{\mu\nu}_{\text{proj}}$ enforces orthogonality to $u^\mu$ if one insists on strictly spatial gradients in the comoving frame. The thermodynamic scalars $\varepsilon(n)$ and $p(n)$ are derived from the same effective potential $f(n)$ that determines the chemical potential.

This expression makes a single, non-negotiable commitment: the regulator carries an energetic cost. It therefore contributes to gravitational sourcing and to the anisotropic stress sector. Covariant conservation,
\[
\nabla_\mu T^{\mu\nu} = 0,
\]
is then part of the model definition rather than an auxiliary assumption.

Linearizing around an FRW background with $n = \bar n(t) + \delta n(t,\vec{x})$, one finds that the gradient term modifies the pressure perturbation as
\[
\delta p_k
=
c_s^2 \bar\rho \delta_k
+
\frac{\kappa_{\rm eff}}{a^2}
k^2
\bar\rho
\delta_k,
\]
where $\kappa_{\rm eff}$ absorbs background factors from the conversion between $n$ and $\rho$ and from the projection structure.

The anisotropic stress arising from the traceless part of the Korteweg tensor scales as
\[
\sigma_k
\propto
\frac{\kappa_{\rm eff}}{a^2}
k^2
\delta_k,
\]
and therefore becomes relevant only at sufficiently large comoving wavenumber.

The crucial structural consequence is that the regulator modifies the effective squared sound speed according to
\[
c_{\rm eff}^2(k,a)
=
c_s^2(a)
+
\kappa_{\rm eff}
\frac{k^2}{a^2}.
\]

The $k^4$ term in the growth equation therefore arises not from diffusion but from conservative gradient energy. The ultraviolet sector is stabilized by construction because $\kappa_{\rm eff} > 0$ ensures positivity of the highest-order spatial operator. No backward-parabolic pathology is present at the level of the principal symbol.

The lamphron--lamphrodyne refinement that follows should therefore be understood as a geometric restatement of this stability structure. It does not introduce a new field nor a new dynamical degree of freedom; rather, it expresses the sign structure of the effective mass and curvature terms in the scalar sector using the language of entropy curvature and vacuum response.

\section{Entropy Curvature and Effective Compressibility}

The previous section established that the regulator modifies the effective squared sound speed according to
\[
c_{\rm eff}^2(k,a)
=
c_s^2(a)
+
\kappa_{\rm eff}
\frac{k^2}{a^2}.
\]
The ultraviolet stability of the system is therefore controlled by $\kappa_{\rm eff} > 0$, while the large-scale stability is governed by the sign of the background compressibility encoded in $c_s^2(a)$.

We now make explicit how this effective sound speed arises from the curvature of the coarse-grained free-energy density and how a sign change in $c_s^2$ corresponds to a geometric change in entropy curvature.

Recall that the chemical potential derived from the regulated free-energy functional is
\[
\mu = f'(n) - \kappa D^2 n.
\]
Linearizing about a homogeneous background $n=\bar n$ gives
\[
\delta \mu
=
f''(\bar n)\,\delta n
-
\kappa D^2 \delta n.
\]

The effective adiabatic sound speed follows from the thermodynamic relation
\[
c_s^2
=
\left.\frac{\partial p}{\partial \rho}\right|_{\bar n}
=
\frac{\bar n}{\bar \rho}
f''(\bar n),
\]
up to background conversion factors between $n$ and $\rho$. Thus the sign of $c_s^2$ is determined directly by the second variation of the coarse-grained free-energy density.

In particular, the homogeneous state is linearly stable when
\[
f''(\bar n) > 0,
\]
and becomes unstable when
\[
f''(\bar n) < 0.
\]
This is precisely the spinodal condition in non-equilibrium thermodynamics: the instability is not a reversal of diffusion but a local indefiniteness of the second variation of the entropy (or free-energy) functional.

It is therefore natural to define an entropy-curvature scalar that captures this structure geometrically. Let $S$ denote the coarse-grained entropy density associated with $n$, and define its spatial curvature through the projected Laplacian,
\[
\Delta_g S := h^{\mu\nu}\nabla_\mu\nabla_\nu S.
\]
The sign structure of $f''(\bar n)$ may be equivalently expressed through the curvature of $S$ because
\[
\delta^2 \mathcal{F}
=
\int_{\Sigma_t}
\sqrt{\gamma}
\left(
f''(\bar n)(\delta n)^2
+
\kappa D_i \delta n D^i \delta n
\right)d^3x,
\]
and the gradient term enforces convexity at high wavenumber while the sign of $f''$ governs convexity at long wavelength.

The transition between stabilizing and destabilizing regimes is therefore equivalently described as a change in entropy curvature at the background level. When the second variation becomes indefinite, the system enters a regime in which long-wavelength modes experience effective negative compressibility, while short-wavelength modes remain stabilized by the gradient regulator.

This geometric reformulation prepares the introduction of lamphron and lamphrodyne states. The dichotomy will not introduce new degrees of freedom; rather, it will express the sign structure of $f''(\bar n)$ and the induced curvature of $S$ in a local geometric language that makes the instability domain manifest.

\section{Lamphron and Lamphrodyne States}

We now introduce a refinement of the scalar--entropy coupling in terms of dual thermodynamic regimes, termed lamphron and lamphrodyne states. These are not additional fields but distinct dynamical phases of the scalar--vector--entropy system.

Let $\phi$ denote the primary scalar density and introduce an effective vacuum response field $\chi$ defined implicitly through the entropy curvature:

\[
\chi := - \beta \Delta_g S.
\]

The evolution equation \eqref{eq:evolution} may then be rewritten as

\[
\partial_t \phi
=
c^2 \Delta_g \phi
- \nabla \cdot (\phi \mathbf{v})
+ \alpha \phi
+ \chi \phi.
\]

We define a lamphron state as a regime in which $\chi < 0$ almost everywhere, so that entropy curvature enhances smoothing and suppresses amplification. Conversely, a lamphrodyne state occurs when $\chi > 0$ on a region of nonzero measure, so that entropy curvature contributes to scalar growth.

\begin{definition}
A lamphrodyne domain $U \subset \mathcal{M}$ is an open set such that $\chi(x) > 0$ for all $x \in U$.
\end{definition}

In lamphron regions, the effective mass term remains stabilizing. In lamphrodyne regions, the effective mass becomes negative in the diffusive sector, inducing scalar irruption.

To formalize this dichotomy, define the effective linear operator

\[
\mathcal{L}_{\text{eff}} = c^2 \Delta_g + \alpha + \chi.
\]

If the principal symbol of $\mathcal{L}_{\text{eff}}$ changes sign in a region, the PDE transitions from parabolic smoothing to backward parabolic amplification. Thus lamphrodyne states correspond precisely to local sign inversion of the principal symbol.

\section{Principal Symbol and Second Variation of the Effective Operator}

The lamphron--lamphrodyne distinction was introduced above through the sign of the effective curvature term
\[
\chi = -\beta \Delta_g S,
\]
and through the effective linear operator
\[
\mathcal{L}_{\mathrm{eff}} = c^2 \Delta_g + \alpha + \chi.
\]

It is essential to clarify that lamphrodyne behavior does not correspond to a reversal of the highest--order spatial operator in the sense of a change from parabolic to backward--parabolic type. The principal symbol of $\mathcal{L}_{\mathrm{eff}}$ remains that of $c^2 \Delta_g$, which is elliptic and therefore generates parabolic smoothing in the evolution equation.

The instability instead arises from the sign of the second variation of the effective free--energy functional.

\subsection{Second Variation and Stability}

Let the scalar sector be governed by a functional of the form
\[
\mathcal{F}[\phi]
=
\int_{\Sigma_t}
\sqrt{\gamma}
\left(
f(\phi)
+
\frac{\kappa}{2} D_i \phi D^i \phi
\right)
\, d^3x.
\]

Linearizing about a homogeneous background $\bar\phi$ yields the quadratic form
\[
\delta^2 \mathcal{F}
=
\int
\left(
f''(\bar\phi)\, (\delta\phi)^2
+
\kappa D_i \delta\phi D^i \delta\phi
\right).
\]

In Fourier space, this becomes
\[
\delta^2 \mathcal{F}
=
\int
\left(
f''(\bar\phi)
+
\kappa \frac{k^2}{a^2}
\right)
|\delta\phi_k|^2.
\]

Stability requires
\[
f''(\bar\phi)
+
\kappa \frac{k^2}{a^2}
> 0
\quad
\text{for all } k.
\]

When $f''(\bar\phi)<0$, the quadratic form becomes indefinite over a finite band of wavenumbers,
\[
0 < \frac{k^2}{a^2} < \frac{|f''(\bar\phi)|}{\kappa}.
\]

The ultraviolet sector remains stable because the $k^2$ term in the quadratic form dominates at large $k$.

\subsection{Relation to the Effective Operator}

The lamphrodyne condition $\chi>0$ corresponds, at linear order, to the regime in which
\[
\alpha + \chi
<
0
\]
after absorbing background terms into the definition of $f''(\bar\phi)$.

The growth of perturbations is therefore governed by the sign of the second variation of $\mathcal{F}$ rather than by a literal inversion of the Laplacian in the principal symbol.

The evolution equation remains of regulated parabolic type; the instability is band--limited and controlled by the gradient penalty coefficient $\kappa$.

\subsection{Geometric Interpretation}

In this formulation, lamphron regions correspond to points in field space where the effective compressibility
\[
\frac{\partial^2 f}{\partial \phi^2}
\]
is positive, while lamphrodyne regions correspond to those where it becomes negative.

The instability boundary is therefore defined by
\[
f''(\phi) = 0,
\]
which is the geometric locus at which the free--energy landscape transitions from convex to locally concave.

The lamphron--lamphrodyne refinement is thus a geometric characterization of a regulated spinodal transition in field space rather than a change in the parabolic character of the highest--order operator.

\section{Spectral Decomposition and Resonant Structure Formation}

To analyze spatial patterning induced by scalar irruption, expand $\phi$ in eigenfunctions of the Laplace--Beltrami operator:

\[
\phi(x,t) = \sum_{k=0}^{\infty} a_k(t) \psi_k(x),
\qquad
\Delta_g \psi_k = -\lambda_k \psi_k,
\]

with $0 = \lambda_0 < \lambda_1 \le \lambda_2 \le \cdots$.

Substituting into the linearized irruption regime yields

\[
\dot{a}_k
=
\left(
- c^2 \lambda_k + \alpha + \chi_k
\right)
a_k,
\]

where $\chi_k$ denotes the spectral projection of $\chi$ onto $\psi_k$.

Growth occurs when

\[
\alpha + \chi_k > c^2 \lambda_k.
\]

Thus only a band of modes satisfying

\[
\lambda_k < \frac{\alpha + \chi_k}{c^2}
\]

are unstable. This band-limited instability naturally produces quasi-periodic structure formation analogous to baryon acoustic oscillation--like resonances, without invoking metric expansion.

\begin{proposition}
If $\chi$ is spatially localized and positive in a bounded region, then the unstable spectrum is discrete and finite.
\end{proposition}

\begin{proof}
Since $\lambda_k \to \infty$ as $k \to \infty$ and $\chi_k$ is bounded by $\norm{\chi}_{L^2}$, there exists $K$ such that for all $k>K$, $c^2 \lambda_k > \alpha + \chi_k$. Hence only finitely many modes are unstable.
\end{proof}

This explains why scalar irruption yields structured condensation rather than runaway ultraviolet divergence.

\section{Quantized Scalar Irruption Operator}

We now sketch a semiclassical quantization of scalar irruption.

Promote $\phi$ to an operator-valued field $\hat{\phi}$ on a Hilbert space $\mathcal{H}$. The classical Hamiltonian density corresponding to the Lagrangian in \eqref{eq:action} is

\[
\mathcal{H}
=
\frac{1}{2} \pi^2
+
\frac{c^2}{2} \norm{\nabla \phi}^2
+
U(\phi,S),
\]

where $\pi = \partial_t \phi$ is the conjugate momentum.

Quantization imposes canonical commutation relations

\[
[\hat{\phi}(x), \hat{\pi}(y)] = i\hbar \delta(x-y).
\]

In lamphrodyne regimes where $\chi>0$, the quadratic part of the Hamiltonian acquires negative eigenvalues, producing inverted harmonic oscillator sectors:

\[
\hat{H}_k
=
\frac{1}{2} \hat{\pi}_k^2
-
\frac{1}{2} \omega_k^2 \hat{\phi}_k^2.
\]

Such sectors exhibit exponential amplification of vacuum fluctuations, analogous to particle production in time-dependent backgrounds. However, here the trigger is entropic curvature rather than metric expansion.

\begin{theorem}
In a lamphrodyne domain, the vacuum state is dynamically unstable under the quantized Hamiltonian, leading to exponential growth of mode occupation numbers.
\end{theorem}

\begin{proof}
For an inverted harmonic oscillator,
\[
\ddot{\hat{\phi}}_k = \omega_k^2 \hat{\phi}_k,
\]
whose solutions grow exponentially. The number operator expectation value diverges as $e^{2\omega_k t}$.
\end{proof}

Thus scalar irruption admits a consistent quantum interpretation as entropically driven mode excitation.

\section{Observable Consequences in a Non-Expanding Plenum}

Since the metric is static, observable signatures arise from redistribution rather than expansion. Consider a two-point correlation function

\[
C(r) = \langle \phi(x)\phi(x+r)\rangle.
\]

In the linear regime, unstable modes imprint oscillatory features:

\[
C(r) \sim \sum_{k \in \mathcal{U}} e^{2\gamma_k t} \psi_k(x)\psi_k(x+r),
\]

where $\mathcal{U}$ denotes unstable modes.

These correlations mimic acoustic-like resonances in spatial power spectra while preserving constant global volume. Hence BAO-like signatures can emerge purely from entropic bifurcation.

\section{Entropy Conservation and Global Constraints}

Despite local amplification, global scalar mass is conserved under zero-flux boundary conditions:

\[
\frac{d}{dt} \int_{\mathcal{M}} \phi \, d\mu_g = 0.
\]

\begin{proof}
Integrating \eqref{eq:evolution} over $\mathcal{M}$ and applying the divergence theorem yields zero net flux provided $\mathbf{v}\cdot \mathbf{n}=0$ and $\nabla \phi \cdot \mathbf{n}=0$ on $\partial \mathcal{M}$.
\end{proof}

Scalar irruption therefore redistributes rather than creates total scalar density. Structure formation is internally reorganizational.

\section{Synthesis: Scalar Irruption as Entropic Phase Transition}

Scalar irruption is now seen to possess four equivalent formulations. In PDE language it is a sign reversal of effective diffusion. In spectral language it is band-limited exponential growth of Laplacian modes. In thermodynamic language it is curvature-induced entropy release. In quantum language it is an inverted oscillator instability induced by entropic geometry.

The mechanism requires no expanding metric, no vacuum tunneling, and no external inflaton potential. It emerges from internal entropy differentials within a non-expanding plenum.

\section{Variational Degeneracy and the Geometry of the Critical Locus}

The preceding synthesis exhibits scalar irruption in four equivalent classical languages. 
However, the unifying feature behind these formulations has not yet been made explicit. 
The instability does not arise from the introduction of new degrees of freedom but from a change in the structure of the critical locus of the action functional.

Let
\[
S_{\mathrm{cl}}[\phi]
=
\int_{\mathcal{M}}
\left(
\frac{c^2}{2} \norm{\nabla \phi}^2
+
\frac{\alpha}{2} \phi^2
+
\frac{\beta}{2} \phi^2 \Delta_g S
\right)
d\mu_g.
\]
The Euler–Lagrange equation derived from $S_{\mathrm{cl}}$ reproduces the effective linear operator
\[
\mathcal{L}_{\mathrm{eff}}
=
c^2 \Delta_g + \alpha + \chi,
\qquad
\chi := -\beta \Delta_g S.
\]

In the lamphron regime, the second variation
\[
\delta^2 S_{\mathrm{cl}}[\phi]
=
\int_{\mathcal{M}}
\delta \phi \, \mathcal{L}_{\mathrm{eff}} \delta \phi \, d\mu_g
\]
is positive definite and the classical critical point is isolated and stable.

In the lamphrodyne regime, the second variation becomes indefinite on a finite spectral band. 
The critical locus ceases to be isolated and instead acquires unstable directions. 
This is precisely the variational signature of a phase transition: the Hessian of the action changes signature, and the Morse index of the critical point increases.

Crucially, global conservation demonstrated in the previous section implies that the instability redistributes scalar density within a constrained manifold of fixed total mass. 
The transition is therefore not explosive but reorganizational: the configuration space develops new saddle directions while remaining confined to a conserved hypersurface.

From this perspective, scalar irruption is equivalently characterized as:

\begin{enumerate}
\item a spectral band crossing zero in the principal symbol,
\item a sign change in the second variation of the action,
\item a change in the Morse index of the classical solution.
\end{enumerate}

The natural mathematical language for handling such degeneracies is not merely classical PDE analysis but the geometry of derived critical loci. 
When the Hessian becomes degenerate, the classical critical set must be replaced by a derived space that keeps track of unstable directions and gauge redundancies.

This observation motivates the transition to the Batalin–Vilkovisky formalism. 
The BV extension does not introduce new physical instabilities; rather, it encodes the geometry of the classical degeneracy in a graded symplectic framework that keeps the gauge structure explicit and ensures that the instability is consistent with diffeomorphism invariance.

In this sense, the AKSZ/BV reformulation is not an embellishment of the instability but its natural geometric completion: 
the lamphron–lamphrodyne transition corresponds to a change in the homotopy type of the derived critical locus of the action functional.

\section{AKSZ/BV Formulation of Scalar Irruption}

We now reformulate scalar irruption within the Batalin--Vilkovisky (BV) and Alexandrov--Kontsevich--Schwarz--Zaboronsky (AKSZ) framework in order to exhibit its gauge structure and derived geometric consistency.

Let $\mathcal{M}$ be a smooth compact manifold and consider the graded manifold
\[
\mathcal{F} = T^*[-1]\mathrm{Map}(T[1]\mathcal{M}, \mathbb{R}),
\]
whose degree-zero component corresponds to the scalar field $\phi$. Introduce ghosts $c$ for diffeomorphism symmetry and antifields $\phi^*$ and $c^*$ of opposite degree.

The classical action functional for the scalar--entropy system may be written in AKSZ form as
\[
S_{\mathrm{cl}}[\phi]
=
\int_{\mathcal{M}}
\left(
\frac{c^2}{2} \norm{\nabla \phi}^2
+
\frac{\alpha}{2} \phi^2
+
\frac{\beta}{2} \phi^2 \Delta_g S
\right)
d\mu_g.
\]

The BV extension introduces antifields and ghost couplings:
\[
S_{\mathrm{BV}}
=
S_{\mathrm{cl}}
+
\int_{\mathcal{M}}
\phi^* \mathcal{L}_c \phi
+
c^* \frac{1}{2}[c,c],
\]
where $\mathcal{L}_c$ denotes the Lie derivative along the ghost vector field.

The BV bracket is defined by
\[
\{F,G\}
=
\int_{\mathcal{M}}
\left(
\frac{\delta_r F}{\delta \phi}
\frac{\delta_l G}{\delta \phi^*}
-
\frac{\delta_r F}{\delta \phi^*}
\frac{\delta_l G}{\delta \phi}
+
\frac{\delta_r F}{\delta c}
\frac{\delta_l G}{\delta c^*}
-
\frac{\delta_r F}{\delta c^*}
\frac{\delta_l G}{\delta c}
\right).
\]

The classical master equation requires
\[
\{S_{\mathrm{BV}}, S_{\mathrm{BV}}\} = 0.
\]

\begin{theorem}
The BV-extended scalar--entropy action satisfies the classical master equation provided the entropy functional $S[\phi]$ is diffeomorphism-invariant.
\end{theorem}

\begin{proof}
The entropy density $S = -\phi \log \phi$ is a scalar under diffeomorphisms. Therefore its Laplacian $\Delta_g S$ transforms covariantly. The ghost variation of $S_{\mathrm{cl}}$ cancels against the antifield terms by standard AKSZ construction. Hence the BV bracket vanishes.
\end{proof}

In lamphrodyne regimes, the quadratic part of the BV action develops negative eigenvalues in its kinetic operator. The derived critical locus of $S_{\mathrm{BV}}$ therefore acquires additional nontrivial homology in degree zero, corresponding to irruption branches.

Thus scalar irruption corresponds, in derived geometric language, to a change in the homotopy type of the critical locus of the action functional. The instability is not merely analytic but alters the derived stack of classical solutions.

This completes the BV/AKSZ formulation of scalar irruption as an entropically induced derived bifurcation.

\section{From Derived Bifurcation to Spectral Realization}

The BV/AKSZ formulation establishes that scalar irruption corresponds to a change in the homotopy type of the derived critical locus of the action functional. 
At the classical level this transition appears as a change in the signature of the Hessian; at the derived level it appears as a shift in the cohomological structure governing infinitesimal deformations.

To pass from this geometric formulation to a concrete computational realization, one must exhibit the same instability structure in a representation where spectral data are explicit. 
The essential requirement is that the discretization preserve three structural features:

First, the instability must arise from a sign change in the quadratic form determined by the second variation of the action.

Second, the growth must be band-limited rather than ultraviolet divergent.

Third, global scalar conservation must be maintained so that amplification remains reorganizational rather than generative.

These properties are not artifacts of smooth geometry; they depend only on the spectral properties of the Laplacian and on the variational structure of the entropy coupling. 
Consequently they descend naturally to any representation in which the Laplacian admits a well-defined eigenbasis and the action admits a discrete second variation.

Let $\{\psi_k\}$ denote an orthonormal eigenbasis of $-\Delta_g$ with eigenvalues $\lambda_k \ge 0$. 
In the continuum lamphron regime the quadratic part of the action takes the form
\[
\delta^2 S_{\mathrm{cl}} 
=
\sum_k 
\left(
c^2 \lambda_k + \alpha + \chi
\right)
|\psi_k|^2.
\]
Lamphrodyne transition corresponds to the crossing of zero of this coefficient for a finite subset of modes. 
The instability therefore depends only on the spectral data $\{\lambda_k\}$ and on the effective diffusion coefficient.

A discrete Laplacian on a lattice possesses the same spectral structure: it is a self-adjoint operator with nonnegative spectrum. 
The sign reversal condition therefore translates without alteration into a statement about the eigenvalues of $-\Delta_{\Lambda}$. 
The derived bifurcation of the smooth theory thus acquires a purely algebraic form: certain discrete modes cross from positive to negative curvature directions in the quadratic form.

In this sense the lattice realization is not a separate model but a spectral representation of the same variational instability. 
The derived change in the critical locus becomes a finite-dimensional change in the signature of the quadratic form on the space of lattice modes.

This observation justifies constructing a Crystal Plenum discretization. 
It demonstrates that scalar irruption does not depend on smooth manifold structure, coordinate charts, or differential operators per se. 
It depends only on spectral properties of a Laplacian and on the curvature of the entropy functional. 
Those properties are preserved under discretization provided the lattice Laplacian is symmetric and the entropy functional remains local.

We now proceed to construct this discrete realization explicitly.

\section{Crystal Plenum Discretization and Lattice Realization of Scalar Irruption}

We now construct a discrete realization of scalar irruption within the Crystal Plenum framework, in which the manifold $(\mathcal{M},g)$ is replaced by a lattice $\Lambda$ endowed with a discrete Laplacian.

Let $\Lambda$ be a finite or countably infinite lattice with adjacency relation $\sim$. The scalar field becomes a function
\[
\phi : \Lambda \to \mathbb{R},
\]
and the discrete Laplacian is defined by
\[
(\Delta_{\Lambda} \phi)_i = \sum_{j \sim i} (\phi_j - \phi_i).
\]

The entropy density at site $i$ is defined as
\[
S_i = - \phi_i \log \phi_i.
\]

The discrete evolution equation corresponding to \eqref{eq:evolution} becomes
\[
\dot{\phi}_i
=
c^2 (\Delta_{\Lambda} \phi)_i
+
\alpha \phi_i
+
\beta \phi_i (\Delta_{\Lambda} S)_i.
\]

\subsection{Discrete Irruption Criterion}

Let $\phi_i = \phi_0 + \varepsilon \psi_i$ with $\phi_0$ uniform. Linearizing yields
\[
\dot{\psi}_i
=
c^2 (\Delta_{\Lambda} \psi)_i
+
\alpha \psi_i
-
\beta \phi_0 (1+\log \phi_0) (\Delta_{\Lambda} \psi)_i.
\]

Define the effective diffusion constant
\[
D_{\mathrm{eff}} = c^2 - \beta \phi_0 (1+\log \phi_0).
\]

Let $\lambda_k$ denote eigenvalues of $-\Delta_{\Lambda}$. Then growth rates are
\[
\gamma_k = - D_{\mathrm{eff}} \lambda_k + \alpha.
\]

\begin{theorem}
If $D_{\mathrm{eff}}<0$, then the lattice system exhibits scalar irruption through exponential amplification of a finite set of eigenmodes.
\end{theorem}

\begin{proof}
Since the spectrum of $\Delta_{\Lambda}$ is bounded above on a finite lattice, only modes with sufficiently small $\lambda_k$ satisfy $\gamma_k>0$. Hence growth occurs in a controlled band, leading to patterned condensation.
\end{proof}

\subsection{Poincaré-Triggered Lattice Recrystallization}

In the crystal plenum, recurrence phenomena arise through approximate Poincaré return times of lattice configurations. Let $\phi(t)$ evolve under the discrete dynamics. A recurrence time $T$ satisfies
\[
\norm{\phi(T)-\phi(0)} < \varepsilon.
\]

Recurrence can amplify local entropy curvature by concentrating deviations. If at recurrence time $T$,
\[
(\Delta_{\Lambda} S)_i < -\kappa_c
\]
for some site $i$, the irruption condition is triggered.

Thus Poincaré-Triggered Lattice Recrystallization acts as a catalyst for scalar irruption by generating localized curvature spikes in the entropy landscape.

\subsection{Crystallization After Irruption}

Nonlinear saturation of unstable modes yields new equilibrium configurations $\phi_i^*$ satisfying
\[
0 = c^2 (\Delta_{\Lambda} \phi^*)_i + \alpha \phi_i^* + \beta \phi_i^* (\Delta_{\Lambda} S^*)_i.
\]

These equilibria correspond to discrete crystal-like scalar condensates.

\begin{proposition}
If the nonlinear saturation term is positive definite, the post-irruption equilibria minimize a discrete free energy functional.
\end{proposition}

\begin{proof}
Define the discrete free energy
\[
E_{\Lambda}[\phi] =
\sum_i
\left(
\frac{c^2}{2} \sum_{j \sim i} (\phi_j-\phi_i)^2
-
\frac{\alpha}{2}\phi_i^2
-
\frac{\beta}{2}\phi_i^2 (\Delta_{\Lambda} S)_i
\right).
\]
Critical points satisfy the equilibrium equation above, and positive definiteness of the quartic stabilization term ensures local minimality.
\end{proof}

\subsection{Discrete--Continuous Correspondence}

Let lattice spacing be $h$. Then
\[
\Delta_{\Lambda} \phi_i = h^2 \Delta_g \phi(x_i) + O(h^3).
\]

Thus the discrete irruption condition converges to the continuous condition as $h \to 0$.

This demonstrates that scalar irruption is not an artifact of continuum modeling but persists under crystallographic discretization. The Crystal Plenum therefore provides a geometrically concrete realization of entropically driven scalar phase transitions.


\section{Limiting Cases and Structural Degeneracies}

It is instructive to examine several limiting regimes of the regulated scalar sector in order to clarify what is genuinely novel in the construction.

If $\kappa_{\rm eff} \to 0$, the $k^4$ term vanishes and the theory reduces to a conventional fluid with sound speed $c_s^2$. In a regime where $c_s^2<0$, the linear system becomes backward-parabolic and therefore ill-posed. The regulator is thus not decorative; it is the element that restores well-posedness and enforces band limitation.

If $c_s^2 \to 0$ while $\kappa_{\rm eff}>0$, the theory reduces to a scale-dependent pressure model in which high-$k$ modes are stabilized but low-$k$ modes evolve gravitationally as in dust. In this limit, deviations from $\Lambda$CDM appear only beyond a regulator-controlled wavenumber.

If gravitational coupling is formally switched off by taking $G\to 0$, the instability structure reduces to the conservative spinodal form controlled entirely by $c_s^2$ and $\kappa_{\rm eff}$. Growth then reflects internal free-energy curvature rather than gravitational collapse.

If the background expansion is frozen by setting $a(t)=1$, the model reduces to a static plenum with regulated phase separation. The FRW embedding therefore does not generate the instability; it modulates it through redshifting of physical wavenumbers.

Finally, if $\beta\to 0$ in the original entropy-coupled formulation, the effective negative compressibility sector disappears and the lamphrodyne regime collapses to purely stabilizing dynamics. The instability is therefore directly traceable to entropy curvature rather than to the regulator itself.

These limiting cases show that the regulated irruption mechanism occupies a distinct structural position between standard Jeans instability and unregulated backward diffusion. Its characteristic feature is a finite instability band bounded in both infrared and ultraviolet by independently identifiable physical parameters.

\section{Comparisons with Other Frameworks}

The mechanism developed here occupies a distinct structural position relative to several established paradigms of structure formation and instability theory. Although it shares formal similarities with known mechanisms, its ontological commitments and regulator structure differ in essential ways.

\subsection*{Contrast with Inflationary Instability}

In slow-roll inflationary cosmology, structure originates from quantum fluctuations of a scalar inflaton field stretched to superhorizon scales by accelerated metric expansion. The instability mechanism is geometric: mode amplification occurs because physical wavelengths outpace the Hubble radius, freezing curvature perturbations.

By contrast, scalar irruption does not rely on accelerated expansion, horizon crossing, or vacuum energy domination. The instability arises from the internal thermodynamic curvature of a conserved density functional. The regulator enforces ultraviolet control directly through a gradient-energy term rather than through redshifting by expansion. Scale selection is therefore intrinsic and parameter-determined, not horizon-imprinted.

\subsection*{Contrast with Standard Jeans Instability}

The classical Jeans mechanism arises from competition between gravitational attraction and positive pressure support. Instability occurs for wavelengths exceeding the Jeans length, determined by the sound speed and background density.

Scalar irruption modifies this structure in two respects. First, compressibility may become negative in a finite interval, generating an instability independent of gravitational attraction. Second, the presence of a $k^4$ regulator term bounds the unstable band from above. The resulting instability is therefore neither purely infrared (Jeans-type) nor ultraviolet-divergent, but spectrally finite. Gravity shifts band edges but does not determine the intrinsic preferred scale.

\subsection*{Relation to Cahn--Hilliard and Spinodal Decomposition}

Mathematically, the regulated instability resembles spinodal phase separation in a conserved order parameter governed by a Cahn--Hilliard equation. In that setting, instability arises from loss of convexity of a free-energy density, and gradient-energy regularization suppresses short-wavelength divergence.

Scalar irruption may be regarded as a relativistic and cosmologically embedded generalization of this structure. The key difference lies in the gravitational coupling, the cosmological background, and the interpretation of the scalar as a matter density rather than a chemical order parameter. The ultraviolet regulator plays the same structural role as in Cahn--Hilliard theory but is derived from a covariant stress--energy sector rather than imposed phenomenologically.

\subsection*{Relation to Effective Field Theory of Fluids}

In effective field theory treatments of cosmological fluids, higher-derivative corrections are often introduced as gradient expansions encoding short-distance physics. The $\kappa_{\rm eff}$ term may be interpreted in this language as the leading operator in a derivative expansion consistent with conservation and symmetry.

However, in the present framework the regulator is not merely a perturbative correction but an essential structural component. It controls well-posedness, determines the instability band, and fixes the physical irruption scale. Removing it collapses the theory into either ill-posed backward diffusion or trivial stability.

\subsection*{Geometric Interpretation}

Unlike inflationary scenarios that attribute structure to spacetime expansion, or purely mechanical phase transitions that treat curvature as secondary, scalar irruption attributes instability directly to entropy curvature in a fixed geometric background. The lamphron–lamphrodyne distinction formalizes this geometrically: instability corresponds to local sign inversion of an effective curvature-induced mass term within an otherwise elliptic operator.

\subsection*{Summary of Structural Distinctions}

The framework developed here may therefore be characterized as follows. It is conservative rather than source-driven, regulator-controlled rather than ultraviolet-divergent, curvature-triggered rather than expansion-triggered, and band-limited rather than scale-free. Its preferred clustering scale is intrinsic and parameter-determined, and its instability is a phase transition in thermodynamic geometry rather than a consequence of background dilation.

These distinctions clarify that scalar irruption is neither a rephrasing of inflation nor a simple restatement of Jeans collapse, but a hybrid regulated spinodal mechanism embedded in a relativistic fluid with explicit conservation structure.

\section{Future Directions}

The present formulation establishes scalar irruption as a regulated, band-limited instability arising from loss of convexity in a conserved relativistic density functional. Several natural extensions follow from this foundation.

First, the nonlinear regime warrants systematic treatment beyond quadratic stabilization. While the finite spinodal window determines the fastest-growing mode in linear theory, the long-time morphology of condensates depends on higher-order terms in the effective free-energy functional. A fully nonlinear analysis could determine whether late-time equilibria correspond to stable solitonic structures, periodic lattices, or metastable glass-like configurations. In particular, a rigorous existence and regularity theory for global solutions in the presence of the $k^4$ regulator would clarify the admissible phase diagram of the plenum.

Second, the coupling between scalar irruption and vector flow remains only partially explored. The inclusion of vorticity, shear viscosity, and anisotropic stress may modify the instability band or produce secondary instabilities analogous to Kelvin–Helmholtz–type modes in compressible fluids. A consistent treatment within the covariant stress–energy framework would determine whether irruption seeds rotational structure or remains purely compressive.

Third, the gravitational sector may be developed beyond the linear regime. While the present work derives the modified growth equation and transfer function, a full Einstein–Boltzmann treatment would be required to confront precision cosmological data. In particular, the regulator-induced $k^4$ correction predicts scale-dependent gravitational slip and a distinctive high-$k$ suppression that should be propagated through lensing kernels, CMB anisotropies, and nonlinear halo formation models.

Fourth, the derived-geometric formulation suggests a deeper structural question. The lamphron–lamphrodyne transition corresponds to a change in the homotopy type of the classical solution space. A systematic study of this transition within the BV/AKSZ framework could clarify whether irruption corresponds to a derived Morse bifurcation and whether the instability admits a cohomological classification. Such analysis may reveal whether entropic phase transitions possess universal homological signatures.

Fifth, the discrete Crystal Plenum realization invites renormalization analysis. As lattice spacing varies, the effective parameters $\alpha$, $\beta$, and $\kappa_{\rm eff}$ flow under coarse graining. Determining the renormalization group structure of the regulated instability would establish whether scalar irruption is stable under scale transformations or requires fine tuning. In particular, the scaling dimension of the gradient regulator controls whether the $k^4$ term remains relevant in the infrared limit.

Finally, observational discriminants must be sharpened into quantitative forecasts. The regulator fixes a preferred physical clustering scale during the spinodal interval. If such an interval occurred in the cosmological past, residual imprints should survive in the matter power spectrum, weak-lensing shear correlations, and possibly in relic neutrino or gravitational-wave backgrounds. A program of parameter estimation would determine whether the allowed window for $\kappa_{\rm eff}$ is already constrained by existing surveys or whether the mechanism predicts novel small-scale deviations accessible to future instruments.

\section{Conclusion}

Scalar irruption has been formulated as an entropically driven instability in a non-expanding plenum, beginning from a strictly variational construction and proceeding through spectral, cosmological, and derived-geometric reformulations. The central mechanism is the loss of convexity of a coarse-grained free-energy functional governing a conserved scalar density. When the second variation of that functional changes sign, a finite band of modes becomes unstable while the ultraviolet sector remains controlled by an explicit gradient regulator. 

The analysis established that the instability is not a consequence of metric expansion, vacuum tunneling, or inflaton dynamics. Instead, it arises from the sign reversal of an effective diffusive coefficient induced by entropy curvature. The highest-order spatial operator remains positive definite, ensuring well-posedness. Amplification is therefore band-limited and regulator-controlled rather than backward-parabolic or ultraviolet divergent. The preferred physical scale of growth is determined by the ratio of entropy curvature to the regulator parameter, yielding an explicit parameter-to-scale map.

Embedding the regulated fluid into a cosmological background demonstrated that the same structure produces a modified growth equation for scalar perturbations. The regulator generates a $k^4$ contribution to the dispersion relation and induces a scale-dependent effective sound speed. In the matter-dominated regime this produces a calculable small-scale suppression of power, while in a finite spinodal interval a bounded irruption band appears. The fastest-growing mode is determined intrinsically by the compressibility–regulator ratio and is not shifted at leading order by gravitational coupling. The instability thus interpolates between Jeans collapse and spinodal phase separation within a single conservative scaffold.

The gauge-invariant reformulation via the Mukhanov–Sasaki variable confirmed that the $k^4$ regulator term survives at the level of the canonical quadratic action. The ultraviolet control and band-limited instability are therefore not gauge artifacts but physical consequences of the regulated stress–energy sector. Observable implications follow directly: a preferred clustering scale, modified high-$k$ asymptotics in the linear matter power spectrum, and scale-dependent anisotropic stress.

At the level of geometric structure, the lamphron–lamphrodyne distinction was shown to correspond to the sign of local entropy curvature. The transition between these regimes is a change in the signature of the quadratic form governing perturbations. In PDE language it appears as effective diffusion sign inversion; in spectral language as bounded exponential growth of Laplacian modes; in thermodynamic language as a convexity-to-concavity transition of the free-energy density; in cosmological language as a regulated instability window; and in quantum language as the emergence of inverted harmonic-oscillator sectors.

The BV/AKSZ extension exhibited scalar irruption as a derived bifurcation: the instability alters the homotopy type of the critical locus of the action functional. This places the mechanism within a formally consistent gauge-theoretic framework and shows that the transition is not merely analytic but structural at the level of the derived solution space.

Finally, the Crystal Plenum discretization demonstrated that scalar irruption persists under lattice realization. The instability depends only on spectral properties of a Laplacian and on the curvature of an entropy functional. These features survive discretization provided the Laplacian remains self-adjoint and the regulator positive. The discrete system therefore reproduces band-limited growth, finite unstable spectra, and nonlinear saturation into patterned condensates, confirming that the mechanism is not an artifact of continuum modeling.

Taken together, these results establish scalar irruption as a controlled, conservative, and geometrically interpretable phase transition in a non-expanding plenum. Structure emerges from smoothing dynamics when entropy curvature exceeds a regulator-defined threshold. No background dilation, external potential, or ultraviolet pathology is required. The instability is intrinsic, bounded, and parameter-determined.

Future extensions may include fully nonlinear evolution beyond the spinodal window, coupling to vector transport sectors, inclusion of additional conserved charges, and detailed confrontation with observational power spectra. At the formal level, further development within derived geometry, renormalization analysis of the regulator sector, and exploration of multi-field generalizations remain open directions.

The essential conclusion remains invariant across all representations: entropic differential alone, embedded within a variationally consistent and ultraviolet-regulated framework, suffices to generate scale-selected structure in a non-expanding cosmological setting.


\newpage
\section*{Appendices}

\appendix
\section{Functional-Analytic Structure of the Regulated Free Energy}

Let $\Sigma$ be a compact Riemannian 3-manifold with metric $\gamma_{ij}$ and Laplace--Beltrami operator $D^2 = D_i D^i$. Consider the regulated free-energy functional
\[
\mathcal{F}[n]
=
\int_{\Sigma}
\left(
f(n)
+
\frac{\kappa}{2}
D_i n D^i n
\right)
\sqrt{\gamma}\, d^3x,
\qquad
\kappa>0.
\]

\subsection*{Sobolev Setting}

Assume
\[
n \in H^1(\Sigma),
\]
so that $n$ and its first weak derivatives are square-integrable. Then $\mathcal{F}[n]$ is well-defined provided $f$ is $C^2$ with at most polynomial growth.

The first variation is
\[
\delta \mathcal{F}
=
\int_{\Sigma}
\left(
f'(n)\,\delta n
-
\kappa D^2 n\,\delta n
\right)
\sqrt{\gamma}\, d^3x,
\]
after integration by parts.

Hence
\[
\frac{\delta \mathcal{F}}{\delta n}
=
f'(n)
-
\kappa D^2 n.
\]

\subsection*{Second Variation and Convexity}

Let $n = \bar n + \varepsilon \eta$ with $\eta \in H^1(\Sigma)$. The second variation at $\bar n$ is

\[
\delta^2 \mathcal{F}[\bar n](\eta,\eta)
=
\int_{\Sigma}
\left(
f''(\bar n)\eta^2
+
\kappa D_i \eta D^i \eta
\right)
\sqrt{\gamma}\, d^3x.
\]

Let $\{-\lambda_k\}$ be the eigenvalues of $D^2$,
\[
D^2 \psi_k = -\lambda_k \psi_k,
\qquad
\lambda_k \ge 0.
\]

Expanding
\[
\eta = \sum_k a_k \psi_k,
\]
gives
\[
\delta^2 \mathcal{F}
=
\sum_k
\left(
f''(\bar n)
+
\kappa \lambda_k
\right)
a_k^2.
\]

Convexity holds iff
\[
f''(\bar n) + \kappa \lambda_k > 0
\quad
\forall k.
\]

If $f''(\bar n) < 0$, instability occurs only for
\[
0 < \lambda_k < \frac{|f''(\bar n)|}{\kappa}.
\]

Thus the unstable spectrum is finite.

\subsection*{Coercivity at High Frequency}

Since $\lambda_k \to \infty$ as $k\to\infty$, one has
\[
f''(\bar n) + \kappa \lambda_k \to +\infty.
\]

Therefore
\[
\delta^2 \mathcal{F}
\ge
\kappa \| \eta \|_{H^1}^2
-
|f''(\bar n)| \|\eta\|_{L^2}^2.
\]

Using the compactness of $\Sigma$, the embedding
\[
H^1(\Sigma) \hookrightarrow L^2(\Sigma)
\]
is continuous, so the quadratic form is bounded below and coercive on the orthogonal complement of the unstable eigenspaces.

Hence the regulated functional defines a well-posed gradient flow in $H^1(\Sigma)$.

\subsection*{Spectral Gap in the Spinodal Regime}

Define the unstable band
\[
\mathcal{U}
=
\left\{
k \mid
\lambda_k < \frac{|f''(\bar n)|}{\kappa}
\right\}.
\]

Since $\lambda_k \sim k^2$ asymptotically,
\[
|\mathcal{U}| < \infty.
\]

Let
\[
\lambda_{\max}^{\mathrm{unst}}
=
\max_{k \in \mathcal{U}} \lambda_k.
\]

Then the growth rate in linear gradient flow
\[
\partial_t \eta = -\frac{\delta \mathcal{F}}{\delta n}
\]
is
\[
\gamma_k = |f''(\bar n)|\lambda_k - \kappa \lambda_k^2.
\]

The maximal growth occurs at
\[
\lambda_* = \frac{|f''(\bar n)|}{2\kappa}.
\]

This establishes existence of a preferred eigenmode independent of background expansion.

\subsection*{Fourth-Order Operator Structure}

The linearized operator may be written
\[
\mathcal{L}
=
- f''(\bar n) D^2
+
\kappa (D^2)^2.
\]

This is elliptic of order four with principal symbol
\[
\sigma(\mathcal{L})(\xi)
=
\kappa |\xi|^4.
\]

Since $\kappa>0$, the principal symbol is strictly positive. Therefore $\mathcal{L}$ is strongly elliptic, guaranteeing well-posedness of the associated parabolic problem.

No backward-parabolic sector appears because the highest-order term remains positive definite.

\section{Hamiltonian Structure and Absence of Ostrogradsky Instabilities}

We analyze the canonical structure of the regulated scalar sector at quadratic order and demonstrate that the $k^4$ regulator does not introduce higher-time-derivative instabilities.

\subsection*{Quadratic Action}

Recall the gauge-invariant quadratic action
\[
S^{(2)}
=
\frac{1}{2}
\int d\eta\, d^3x \,
\left[
(v')^2
-
c_s^2 (\nabla v)^2
-
\kappa_{\rm eff}\frac{1}{a^2} (\Delta v)^2
+
\frac{z''}{z} v^2
\right],
\]
where $v = z \mathcal{R}$.

The regulator modifies only spatial derivatives:
\[
(\Delta v)^2 = (\partial_i \partial^i v)^2.
\]

No higher time derivatives appear.

\subsection*{Canonical Momentum and Hamiltonian}

The canonical momentum conjugate to $v$ is
\[
\pi = \frac{\partial \mathcal{L}}{\partial v'} = v'.
\]

The Hamiltonian density is
\[
\mathcal{H}
=
\frac{1}{2}
\left[
\pi^2
+
c_s^2 (\nabla v)^2
+
\kappa_{\rm eff}\frac{1}{a^2} (\Delta v)^2
-
\frac{z''}{z} v^2
\right].
\]

In Fourier space,
\[
v(\eta,\vec{x})
=
\int \frac{d^3k}{(2\pi)^3}
v_k(\eta) e^{i\vec{k}\cdot\vec{x}},
\]
the Hamiltonian becomes a sum over decoupled modes:
\[
H
=
\frac{1}{2}
\int d^3k
\left[
|v_k'|^2
+
\omega_k^2(\eta)
|v_k|^2
\right],
\]
with
\[
\omega_k^2(\eta)
=
c_s^2 k^2
+
\kappa_{\rm eff}\frac{k^4}{a^2}
-
\frac{z''}{z}.
\]

\subsection*{Absence of Higher-Time-Derivative Degrees of Freedom}

Ostrogradsky instabilities arise when the Lagrangian contains higher time derivatives such as $v''$ or $(v')^2$ coupled to time derivatives of higher order.

In the present case:

\[
\mathcal{L}
=
\frac{1}{2}
\left[
(v')^2
-
\text{(spatial derivatives only)}
\right].
\]

Therefore:

\[
\frac{\partial^2 \mathcal{L}}{\partial (v'')^2} = 0.
\]

The Euler–Lagrange equation remains second order in time:

\[
v_k''
+
\omega_k^2 v_k
=
0.
\]

Hence no additional canonical momentum is introduced, and the phase space remains two-dimensional per mode.

\subsection*{Energy Positivity at High Wavenumber}

For fixed conformal time $\eta$, as $k\to\infty$,
\[
\omega_k^2(\eta)
\sim
\kappa_{\rm eff}\frac{k^4}{a^2}.
\]

Since $\kappa_{\rm eff}>0$, one has
\[
\omega_k^2 \to +\infty.
\]

Therefore high-$k$ modes have positive definite Hamiltonian contribution:
\[
H_k \sim \frac{1}{2}
\left(
|v_k'|^2
+
\kappa_{\rm eff}\frac{k^4}{a^2} |v_k|^2
\right).
\]

No ultraviolet ghost appears.

\subsection*{Comparison with Genuine Higher-Derivative Gravity}

In contrast, consider a higher-derivative time action of schematic form
\[
\mathcal{L}
=
\frac{1}{2}
\left[
(v'')^2
-
(\nabla v)^2
\right].
\]

Then
\[
\pi_1 = \frac{\partial \mathcal{L}}{\partial v'} - \frac{d}{d\eta}\frac{\partial \mathcal{L}}{\partial v''},
\qquad
\pi_2 = \frac{\partial \mathcal{L}}{\partial v''}.
\]

The phase space doubles, and the Hamiltonian becomes unbounded from below.

No such structure appears in the regulated scalar irruption model.

The regulator introduces higher spatial derivatives only. The time evolution remains second order. The Hamiltonian is bounded from below at high wavenumber, provided $\kappa_{\rm eff}>0$.

Therefore the regulated scalar sector is free of Ostrogradsky instabilities and remains a conventional canonical field theory with modified dispersion relation:
\[
\omega_k^2
=
c_s^2 k^2
+
\kappa_{\rm eff}\frac{k^4}{a^2}
-
\frac{z''}{z}.
\]

\section{Renormalization-Group Interpretation and Effective Field-Theory Domain}

\subsection*{Dimensional Analysis}

Consider the quadratic action for the Mukhanov--Sasaki variable:
\[
S^{(2)}
=
\frac{1}{2}
\int d\eta\, d^3x
\left[
(v')^2
-
c_s^2 (\nabla v)^2
-
\kappa_{\rm eff}\frac{1}{a^2} (\Delta v)^2
+
\frac{z''}{z} v^2
\right].
\]

Assign scaling dimensions in comoving coordinates:
\[
[x] = L,
\quad
[\eta] = L,
\quad
[v] = L^{-1}.
\]

Then
\[
[(\nabla v)^2] = L^{-4},
\quad
[(\Delta v)^2] = L^{-6}.
\]

To maintain action dimensionless (in units where $\hbar=1$), one finds
\[
[\kappa_{\rm eff}] = L^2.
\]

Thus $\kappa_{\rm eff}$ defines a squared length scale.

\subsection*{Associated Physical Cutoff Scale}

Define the physical wavenumber
\[
k_{\rm phys} = \frac{k}{a}.
\]

The dispersion relation is
\[
\omega_k^2
=
c_s^2 k_{\rm phys}^2
+
\kappa_{\rm eff} k_{\rm phys}^4
-
\frac{z''}{z}.
\]

The crossover between quadratic and quartic behavior occurs at
\[
k_{\rm phys}^2 \sim \frac{c_s^2}{\kappa_{\rm eff}}.
\]

Define the regulator scale
\[
\Lambda_\kappa^2 := \frac{1}{\kappa_{\rm eff}}.
\]

Then
\[
\omega_k^2
=
c_s^2 k_{\rm phys}^2
+
\frac{k_{\rm phys}^4}{\Lambda_\kappa^2}
-
\frac{z''}{z}.
\]

Hence $\Lambda_\kappa$ plays the role of a physical ultraviolet scale.

\subsection*{Wilsonian Interpretation}

Treat the theory as an effective field theory obtained after integrating out microscopic degrees of freedom above some scale $\Lambda_{\rm UV}$.

The most general local quadratic spatial operator consistent with isotropy is an expansion in even powers of $k_{\rm phys}$:

\[
\omega_k^2
=
c_s^2 k_{\rm phys}^2
+
\frac{\alpha_4}{\Lambda_{\rm UV}^2} k_{\rm phys}^4
+
\frac{\alpha_6}{\Lambda_{\rm UV}^4} k_{\rm phys}^6
+
\cdots.
\]

Truncating at fourth order corresponds to retaining the leading irrelevant operator.

Identifying
\[
\kappa_{\rm eff} = \frac{\alpha_4}{\Lambda_{\rm UV}^2}
\]
shows that the regulator term is the lowest-order higher-gradient correction in a Wilsonian derivative expansion.

\subsection*{Scaling Toward the Ultraviolet}

Under spatial rescaling
\[
x \to b x,
\quad
k \to \frac{k}{b},
\]
the operators scale as
\[
k^2 \to b^{-2} k^2,
\quad
k^4 \to b^{-4} k^4.
\]

Thus the quartic operator is irrelevant in the infrared but dominant in the ultraviolet.

In particular, as $k_{\rm phys} \to \infty$,
\[
\omega_k^2 \sim \kappa_{\rm eff} k_{\rm phys}^4,
\]
ensuring ultraviolet stiffening.

\subsection*{EFT Validity Domain}

The derivative expansion is valid provided
\[
\frac{k_{\rm phys}^2}{\Lambda_\kappa^2} \ll 1
\]
for scales where neglected $k^6$ and higher operators remain small.

Thus the conservative scalar irruption theory is an EFT valid for
\[
k_{\rm phys} \ll \Lambda_\kappa.
\]

Within this regime, truncation at fourth order is controlled.

\subsection*{Relation to Lifshitz-Type Scaling}

The quartic dispersion resembles $z=2$ Lifshitz scaling:
\[
\omega \sim k_{\rm phys}^2.
\]

If $c_s^2$ is negligible and $\kappa_{\rm eff}>0$, the high-$k$ fixed point exhibits anisotropic scaling symmetry
\[
\eta \to b^2 \eta,
\quad
x \to b x.
\]

Thus the ultraviolet sector is governed by a Lifshitz-type quadratic-in-time, quartic-in-space operator.

\subsection*{Absence of Fine-Tuning Requirement}

Because $\kappa_{\rm eff}$ is associated with an irrelevant operator, its presence does not require fine tuning to maintain stability. Any positive $\kappa_{\rm eff}$ produces ultraviolet regularization. Only the sign is dynamically essential.

\subsection*{Corollary: Bounded Instability Band}

Suppose $c_s^2(a)<0$ in a finite interval. Then instability requires
\[
|c_s^2| k_{\rm phys}^2
>
\kappa_{\rm eff} k_{\rm phys}^4.
\]

Thus
\[
0 < k_{\rm phys}^2 < \frac{|c_s^2|}{\kappa_{\rm eff}}.
\]

The existence of $\Lambda_\kappa$ guarantees finiteness of this band independent of gravitational contributions.

The regulator term:

\[
\kappa_{\rm eff}(\Delta v)^2
\]

is the leading irrelevant spatial operator in a Wilsonian expansion. It defines a UV stiffness scale $\Lambda_\kappa$, ensures positivity of $\omega_k^2$ at large $k$, and preserves EFT control provided $k_{\rm phys} \ll \Lambda_\kappa$.

No additional degrees of freedom are introduced; the theory remains second order in time and canonically well-defined.

\section{Spectral Theory of Fourth-Order Elliptic Operators on Compact Manifolds}

\subsection*{Ellipticity and Principal Symbol}

Let $(\mathcal{M},g)$ be a smooth compact Riemannian manifold without boundary. Consider the linear operator
\[
\mathcal{L}_\kappa
=
-\kappa_{\rm eff}\Delta_g^2
+
c_s^2 \Delta_g
+
V(x),
\]
acting on $C^\infty(\mathcal{M})$, where $\kappa_{\rm eff}>0$ and $V(x)\in C^\infty(\mathcal{M})$.

The principal symbol of $\mathcal{L}_\kappa$ is
\[
\sigma_{\rm pr}(\mathcal{L}_\kappa)(x,\xi)
=
\kappa_{\rm eff} |\xi|_g^4.
\]

Since $\kappa_{\rm eff}>0$ and $|\xi|_g^4>0$ for $\xi\neq 0$, the operator is strongly elliptic of order four.

\subsection*{Self-Adjointness}

Define the domain
\[
D(\mathcal{L}_\kappa)
=
H^4(\mathcal{M}),
\]
with $\mathcal{L}_\kappa$ acting on $L^2(\mathcal{M})$.

Because $\Delta_g$ is self-adjoint and $V(x)$ is multiplication by a real-valued function, the operator $\mathcal{L}_\kappa$ is symmetric on $C^\infty(\mathcal{M})$ and extends to a self-adjoint operator on $H^4(\mathcal{M})$.

\subsection*{Compact Resolvent}

Since $\mathcal{M}$ is compact and $\mathcal{L}_\kappa$ is elliptic of positive order, the resolvent
\[
(\mathcal{L}_\kappa - \lambda I)^{-1}
\]
is compact for $\lambda$ not in the spectrum.

Hence the spectrum is discrete:
\[
\mathrm{Spec}(\mathcal{L}_\kappa)
=
\{\lambda_0 \le \lambda_1 \le \lambda_2 \le \cdots\},
\]
with $\lambda_n \to +\infty$ as $n\to\infty$.

\subsection*{Spectral Asymptotics}

By Weyl's law for fourth-order elliptic operators,
\[
N(\Lambda)
=
\#\{\lambda_n \le \Lambda\}
\sim
C_{\mathcal{M}}
\, \Lambda^{\frac{d}{4}},
\qquad
\Lambda\to\infty,
\]
where $d=\dim \mathcal{M}$.

Thus
\[
\lambda_n \sim C\, n^{4/d}
\quad\text{as}\quad n\to\infty.
\]

\subsection*{Boundedness of the Unstable Spectrum}

Suppose the linearized growth operator takes the form
\[
\mathcal{G}
=
-\mathcal{L}_\kappa
+
\alpha I,
\]
so that growth rates correspond to eigenvalues
\[
\gamma_n = \alpha - \lambda_n.
\]

Instability occurs when $\gamma_n>0$, i.e.
\[
\lambda_n < \alpha.
\]

Since $\lambda_n\to\infty$, only finitely many indices satisfy this inequality.

\begin{theorem}
The unstable spectrum of the regulated scalar sector is finite on compact manifolds.
\end{theorem}

\begin{proof}
Because $\lambda_n\to\infty$, there exists $N$ such that for all $n>N$, $\lambda_n>\alpha$. Hence $\gamma_n<0$ for $n>N$. Only finitely many modes can satisfy $\gamma_n>0$.
\end{proof}

\subsection*{Energy Coercivity}

Define the quadratic form
\[
Q[u]
=
\int_{\mathcal{M}}
\left(
\kappa_{\rm eff} |\Delta_g u|^2
-
c_s^2 |\nabla u|^2
+
V(x) u^2
\right)
d\mu_g.
\]

For $\kappa_{\rm eff}>0$, the leading term satisfies
\[
\kappa_{\rm eff} \|\Delta_g u\|_{L^2}^2
\ge
C \|u\|_{H^2}^2 - C'\|u\|_{L^2}^2,
\]
by elliptic regularity.

Hence
\[
Q[u] \ge
C \|u\|_{H^2}^2 - C''\|u\|_{L^2}^2,
\]
establishing coercivity modulo compact perturbations.

\subsection*{Band-Limited Instability on $\mathbb{R}^d$}

If $\mathcal{M}=\mathbb{R}^d$, Fourier transform yields symbol
\[
\sigma(\mathcal{L}_\kappa)(k)
=
\kappa_{\rm eff} |k|^4
-
c_s^2 |k|^2
+
\widehat{V}.
\]

Instability requires
\[
\kappa_{\rm eff} |k|^4
-
c_s^2 |k|^2
+
\widehat{V}
<
0.
\]

This inequality defines either an empty set or a bounded interval in $|k|$ whenever $\kappa_{\rm eff}>0$.

\begin{proposition}
On $\mathbb{R}^d$, the unstable Fourier support is compact in $k$-space provided $\kappa_{\rm eff}>0$.
\end{proposition}

\subsection*{Absence of Ultraviolet Catastrophe}

Since
\[
\sigma(\mathcal{L}_\kappa)(k)
\sim
\kappa_{\rm eff} |k|^4
\quad\text{as}\quad |k|\to\infty,
\]
the operator is positive definite in the ultraviolet.

Thus high-frequency modes are strictly damped in the linearized theory.

\subsection*{Spectral Gap Structure}

If the lowest eigenvalue $\lambda_0<0$ in a spinodal window, then there exists a spectral gap
\[
\lambda_0 < \lambda_1 \le \cdots.
\]

The time scale of irruption is governed by
\[
\tau^{-1} = \max_n \gamma_n = \alpha - \lambda_0.
\]

Higher modes decay with rates increasing as $n^{4/d}$.

\subsection*{Corollary: Structured Condensation}

Because only finitely many modes are unstable and the spectrum is discrete, nonlinear saturation generically produces finite-mode condensation patterns rather than broadband turbulence in the linear regime.

\subsection*{Functional Calculus Representation}

The solution operator for the linearized equation
\[
\partial_t u = -\mathcal{L}_\kappa u
\]
is
\[
u(t) = e^{-t\mathcal{L}_\kappa} u(0).
\]

Since $\mathcal{L}_\kappa$ is self-adjoint with discrete spectrum,
\[
e^{-t\mathcal{L}_\kappa}
=
\sum_{n=0}^{\infty}
e^{-t\lambda_n}
P_n,
\]
where $P_n$ are orthogonal spectral projectors.

Growth or decay is entirely controlled by the sign of $\lambda_n$.

The fourth-order regulator yields:

\[
\text{(i)}\ \text{strong ellipticity},\qquad
\text{(ii)}\ \text{discrete spectrum on compact manifolds},\qquad
\text{(iii)}\ \text{finite unstable mode set},\qquad
\text{(iv)}\ \text{ultraviolet positivity}.
\]

Hence the regulated scalar irruption operator is spectrally well-posed in both compact and noncompact settings.

\section{Nonlinear Stability, Lyapunov Structure, and Global Attractors}

\subsection*{Gradient-Flow Structure}

Consider the coarse-grained free-energy functional
\[
\mathcal{F}[n]
=
\int_{\mathcal{M}}
\left(
f(n)
+
\frac{\kappa_{\rm eff}}{2}
|\nabla n|^2
\right)
d\mu_g,
\]
with $\kappa_{\rm eff}>0$ and $f\in C^3(\mathbb{R})$.

The chemical potential is
\[
\mu
=
\frac{\delta \mathcal{F}}{\delta n}
=
f'(n)
-
\kappa_{\rm eff}\Delta_g n.
\]

The conserved Cahn--Hilliard-type evolution reads
\[
\partial_t n
=
\nabla\cdot\!\left(M\nabla\mu\right),
\qquad
M>0.
\]

\subsection*{Lyapunov Dissipation}

Multiply the evolution equation by $\mu$ and integrate:

\[
\int_{\mathcal{M}} \mu\,\partial_t n\,d\mu_g
=
\int_{\mathcal{M}} \mu\,\nabla\cdot(M\nabla\mu)\,d\mu_g.
\]

Integrating by parts and using zero-flux boundary conditions gives
\[
\frac{d}{dt}\mathcal{F}[n]
=
-\int_{\mathcal{M}} M |\nabla \mu|^2\,d\mu_g
\le 0.
\]

\begin{theorem}
The free energy $\mathcal{F}[n]$ is a Lyapunov functional for the nonlinear regulated dynamics.
\end{theorem}

Thus
\[
\mathcal{F}[n(t)] \le \mathcal{F}[n(0)]
\quad \text{for all } t\ge0.
\]

\subsection*{Mass Conservation Constraint}

Because
\[
\partial_t n = \nabla\cdot(M\nabla\mu),
\]
integration over $\mathcal{M}$ yields
\[
\frac{d}{dt}\int_{\mathcal{M}} n\,d\mu_g = 0.
\]

Hence the dynamics evolve on the affine hyperplane
\[
\mathcal{H}_m
=
\left\{
n\in L^2(\mathcal{M}) :
\int_{\mathcal{M}} n\,d\mu_g = m
\right\}.
\]

\subsection*{Coercivity and Boundedness}

Assume $f(n)$ satisfies a polynomial lower bound:
\[
f(n) \ge -C_1 + C_2 |n|^p,
\qquad
p>1.
\]

Then
\[
\mathcal{F}[n]
\ge
C_2 \|n\|_{L^p}^p
+
\frac{\kappa_{\rm eff}}{2}\|\nabla n\|_{L^2}^2
-
C_1|\mathcal{M}|.
\]

Hence boundedness of $\mathcal{F}$ implies boundedness in $H^1(\mathcal{M})$.

\subsection*{Existence of Global Weak Solutions}

Standard theory for fourth-order parabolic equations yields:

\begin{theorem}
If $n_0\in H^1(\mathcal{M})$, then there exists a global weak solution
\[
n \in L^\infty_{\rm loc}(0,\infty;H^1(\mathcal{M}))
\cap
L^2_{\rm loc}(0,\infty;H^2(\mathcal{M}))
\]
to the regulated nonlinear equation.
\end{theorem}

\subsection*{Compactness and Attractor Structure}

Because $\mathcal{F}$ is decreasing and bounded below, trajectories are precompact in $H^1$ modulo subsequences. Define the $\omega$-limit set
\[
\omega(n_0)
=
\left\{
n_* :
\exists t_k\to\infty,
n(t_k)\to n_*
\right\}.
\]

\begin{proposition}
Every element $n_*\in\omega(n_0)$ satisfies
\[
\nabla\cdot(M\nabla\mu_*)=0,
\]
hence is a stationary solution.
\end{proposition}

Thus late-time states correspond to critical points of $\mathcal{F}$ under the mass constraint.

\subsection*{Nonlinear Stability of Minimizers}

Let $n_*$ be a strict local minimizer of $\mathcal{F}$ on $\mathcal{H}_m$.

\begin{theorem}
If the second variation satisfies
\[
\delta^2\mathcal{F}[n_*](\varphi,\varphi) \ge c\|\varphi\|_{H^1}^2
\]
for all $\varphi$ with zero mean, then $n_*$ is nonlinearly asymptotically stable.
\end{theorem}

\subsection*{Spinodal Decomposition and Basin Structure}

If $f''(\bar n)<0$ at a homogeneous state $\bar n$, then
\[
\delta^2\mathcal{F}[\bar n]
=
\int_{\mathcal{M}}
\left(
f''(\bar n)\varphi^2
+
\kappa_{\rm eff}|\nabla\varphi|^2
\right)d\mu_g
\]
is indefinite, and $\bar n$ is a saddle.

Nonlinear evolution then moves the system toward phase-separated minimizers.

\subsection*{Energy Gap and Pattern Selection}

Let $\lambda_0<0$ be the most unstable linear eigenvalue.

The characteristic nonlinear saturation scale satisfies
\[
\|n(t)-\bar n\|_{H^1}
\sim
\sqrt{\frac{|\lambda_0|}{\kappa_{\rm eff}}}.
\]

Only finitely many unstable modes contribute to the emerging pattern, due to the spectral bound established previously.

\subsection*{Inclusion of Gravitational Coupling}

If gravity is reintroduced, the total energy functional becomes
\[
\mathcal{E}[n,\Phi]
=
\mathcal{F}[n]
+
\frac{1}{8\pi G}
\int_{\mathcal{M}} |\nabla\Phi|^2 d\mu_g,
\]
subject to Poisson constraint
\[
\Delta_g \Phi = 4\pi G (n-\bar n).
\]

The combined system remains variational in the quasi-static limit.

\subsection*{Causal Relaxation Extension}

With relaxation time $\tau>0$, the hyperbolic extension reads
\[
\tau \partial_t^2 n + \partial_t n = \nabla\cdot(M\nabla\mu).
\]

Energy functional:
\[
\mathcal{E}_\tau
=
\mathcal{F}[n]
+
\frac{\tau}{2}\|\partial_t n\|_{L^2}^2.
\]

Then
\[
\frac{d}{dt}\mathcal{E}_\tau
=
-
\|\partial_t n\|_{L^2}^2
-
\int M|\nabla\mu|^2.
\]

Thus the hyperbolic regulator preserves dissipativity.

\subsection*{Absence of Finite-Time Blow-Up}

Because $\mathcal{F}$ is bounded below and coercive, and the highest-order term remains elliptic, solutions cannot develop ultraviolet blow-up in finite time within the continuum model.

The regulated scalar irruption system admits:

\[
\text{(i)}\ \text{global weak solutions},\qquad
\text{(ii)}\ \text{Lyapunov monotonicity},\qquad
\text{(iii)}\ \text{compact global attractor},\qquad
\text{(iv)}\ \text{finite-mode pattern selection}.
\]

Thus the instability mechanism extends beyond linear analysis into a globally well-posed nonlinear phase-separation dynamics.

\section{Functional-Integral Formulation and Semiclassical Expansion Near the Spinodal Point}

\subsection*{Euclidean Functional Integral}

Consider the regulated scalar action on a compact Riemannian manifold $(\mathcal{M},g)$:
\[
S_E[\phi]
=
\int_{\mathcal{M}}
\left(
\frac{c^2}{2} |\nabla \phi|^2
+
V_{\rm eff}(\phi)
\right)
d\mu_g,
\]
where the effective potential satisfies
\[
V_{\rm eff}''(\phi_0) = f''(\phi_0).
\]

The formal Euclidean partition function is
\[
Z
=
\int \mathcal{D}\phi\;
e^{-\frac{1}{\hbar} S_E[\phi]}.
\]

Critical points satisfy
\[
-c^2 \Delta_g \phi + V_{\rm eff}'(\phi) = 0.
\]

\subsection*{Quadratic Expansion Around a Homogeneous Background}

Let $\phi = \phi_0 + \eta$ with $\phi_0$ constant.

Expand the action to quadratic order:
\[
S_E[\phi]
=
S_E[\phi_0]
+
\frac{1}{2}
\int_{\mathcal{M}}
\eta
\left(
- c^2 \Delta_g + V_{\rm eff}''(\phi_0)
\right)
\eta
\, d\mu_g
+
O(\eta^3).
\]

Define the fluctuation operator
\[
\mathcal{O}
=
- c^2 \Delta_g + V_{\rm eff}''(\phi_0).
\]

Let $\Delta_g \psi_k = -\lambda_k \psi_k$. Then eigenvalues of $\mathcal{O}$ are
\[
\omega_k^2
=
c^2 \lambda_k + V_{\rm eff}''(\phi_0).
\]

\subsection*{Spinodal Regime and Negative Modes}

In the spinodal regime,
\[
V_{\rm eff}''(\phi_0) < 0.
\]

Then
\[
\omega_k^2
=
c^2 \lambda_k - |V_{\rm eff}''(\phi_0)|.
\]

Negative eigenvalues occur when
\[
\lambda_k < \frac{|V_{\rm eff}''(\phi_0)|}{c^2}.
\]

Thus only finitely many modes contribute negative directions in the Euclidean action.

\begin{proposition}
The Hessian of $S_E$ has finite Morse index in the regulated spinodal regime.
\end{proposition}

This contrasts with unregulated backward-parabolic models, where the number of unstable modes is infinite.

\subsection*{Gaussian Functional Determinant}

To quadratic order,
\[
Z
\approx
e^{-\frac{1}{\hbar} S_E[\phi_0]}
\left( \det \mathcal{O} \right)^{-1/2}.
\]

Split the determinant into positive and negative sectors:
\[
\det \mathcal{O}
=
\prod_{k \in \mathcal{S}_+} \omega_k^2
\prod_{k \in \mathcal{S}_-} (-|\omega_k^2|).
\]

The finite number of negative eigenvalues implies
\[
Z
=
|Z| e^{i \frac{\pi}{2} N_-},
\]
where $N_-$ is the Morse index.

Thus the spinodal point corresponds to a finite-dimensional instability in configuration space.

\subsection*{Semiclassical Time-Dependent Amplification}

Return to Lorentzian signature.

For each unstable mode,
\[
\ddot{\eta}_k - |\omega_k^2| \eta_k = 0,
\]
with solutions
\[
\eta_k(t)
=
A_k e^{\sqrt{|\omega_k^2|}\, t}
+
B_k e^{-\sqrt{|\omega_k^2|}\, t}.
\]

Quantum mechanically, the Hamiltonian for each unstable mode reads
\[
\hat{H}_k
=
\frac{1}{2}\hat{\pi}_k^2
-
\frac{1}{2}|\omega_k^2| \hat{\eta}_k^2.
\]

This is the inverted harmonic oscillator.

\subsection*{Mode Occupation Growth}

Let the initial state be Gaussian with variance $\sigma_0^2$.

Under time evolution,
\[
\langle \hat{\eta}_k^2(t) \rangle
=
\sigma_0^2 \cosh(2\sqrt{|\omega_k^2|}\, t)
+
\frac{\hbar^2}{4\sigma_0^2 |\omega_k^2|}
\sinh(2\sqrt{|\omega_k^2|}\, t).
\]

Hence particle number grows exponentially:
\[
n_k(t)
\sim
e^{2\sqrt{|\omega_k^2|}\, t}.
\]

\subsection*{Finite-Mode Quantum Instability}

Because the regulator enforces
\[
\omega_k^2 \to +\infty \quad \text{as } k\to\infty,
\]
only finitely many modes exhibit inverted-oscillator behavior.

Therefore:

\begin{theorem}
In the regulated scalar irruption model, quantum instability is finite-dimensional at any fixed background configuration.
\end{theorem}

\subsection*{Effective One-Loop Potential Near the Convexity Boundary}

The one-loop correction is
\[
V_{\rm eff}^{(1)}(\phi_0)
=
\frac{\hbar}{2}
\sum_k
\log \omega_k^2.
\]

In the spinodal regime the negative modes generate an imaginary part,
\[
\operatorname{Im} V_{\rm eff}^{(1)}
=
\frac{\hbar}{2}
\sum_{k\in \mathcal{S}_-} \pi.
\]

This imaginary part signals decay of the unstable homogeneous phase.

\subsection*{Renormalization-Scale Interpretation of $\kappa_{\rm eff}$}

In Fourier space,
\[
\omega_k^2
=
c^2 k^2 + \kappa_{\rm eff} k^4 + V''(\phi_0).
\]

The quartic term improves ultraviolet behavior of loop integrals:

\[
\int \frac{d^d k}{(2\pi)^d}
\frac{1}{c^2 k^2 + \kappa_{\rm eff} k^4}
\sim
\int \frac{d^d k}{k^4}
\quad (k\to\infty).
\]

Thus in $d\le 3$ the theory becomes super-renormalizable at one loop.

\subsection*{Effective Field-Theoretic Interpretation}

The gradient regulator corresponds to the leading higher-derivative operator in a derivative expansion:
\[
\mathcal{L}
=
\frac{1}{2}(\partial\phi)^2
+
\frac{\kappa_{\rm eff}}{2}(\Box\phi)^2
-
V(\phi).
\]

Its presence ensures:

\[
\text{(i) finite Morse index,}
\qquad
\text{(ii) improved UV behavior,}
\qquad
\text{(iii) controlled semiclassical instability.}
\]

\subsection*{Semiclassical Picture of Irruption}

Near the convexity boundary,
\[
V''(\phi_0) \to 0^-,
\]
the lowest unstable eigenvalue approaches zero, and the amplification rate scales as
\[
\gamma_{\min}
\sim
\sqrt{|V''(\phi_0)|}.
\]

Hence scalar irruption may be interpreted as a semiclassical tunneling-free phase transition triggered by loss of convexity, with exponential amplification replacing barrier penetration.

\subsection*{Conclusion of Functional Appendix}

The regulated scalar irruption model:

\[
\text{(i)}\ \text{admits a well-defined Gaussian path integral},
\]
\[
\text{(ii)}\ \text{has finite Morse index in the spinodal regime},
\]
\[
\text{(iii)}\ \text{exhibits inverted-oscillator amplification in a finite mode set},
\]
\[
\text{(iv)}\ \text{improves ultraviolet convergence through the } k^4 \text{ term}.
\]

\section{Hamiltonian Structure and Symplectic Geometry of the Regulated Scalar Sector}

\subsection*{Canonical Variables and Phase Space}

Consider the regulated scalar Lagrangian density on a fixed background spacetime:
\[
\mathcal{L}
=
\frac{1}{2}\dot{\phi}^2
-
\frac{c^2}{2}|\nabla\phi|^2
-
\frac{\kappa_{\rm eff}}{2}(\Delta \phi)^2
-
V(\phi).
\]

The canonical momentum is
\[
\pi
=
\frac{\partial \mathcal{L}}{\partial \dot{\phi}}
=
\dot{\phi}.
\]

The phase space is
\[
\mathcal{P}
=
\left\{ (\phi,\pi) \in H^2(\mathcal{M}) \times L^2(\mathcal{M}) \right\},
\]
where $H^2$ regularity is required because of the $(\Delta \phi)^2$ term.

\subsection*{Hamiltonian Functional}

The Hamiltonian is
\[
H[\phi,\pi]
=
\int_{\mathcal{M}}
\left(
\frac{1}{2}\pi^2
+
\frac{c^2}{2}|\nabla\phi|^2
+
\frac{\kappa_{\rm eff}}{2}(\Delta\phi)^2
+
V(\phi)
\right)
d\mu_g.
\]

Functional derivatives yield
\[
\frac{\delta H}{\delta \pi} = \pi,
\qquad
\frac{\delta H}{\delta \phi}
=
- c^2 \Delta \phi
+
\kappa_{\rm eff} \Delta^2 \phi
+
V'(\phi).
\]

Hamilton’s equations are therefore
\[
\dot{\phi} = \pi,
\]
\[
\dot{\pi}
=
c^2 \Delta \phi
-
\kappa_{\rm eff} \Delta^2 \phi
-
V'(\phi).
\]

Combining yields the regulated fourth-order equation
\[
\ddot{\phi}
-
c^2 \Delta \phi
+
\kappa_{\rm eff} \Delta^2 \phi
+
V'(\phi)
=
0.
\]

\subsection*{Symplectic Form}

The canonical symplectic 2-form on phase space is
\[
\Omega
=
\int_{\mathcal{M}}
\delta \pi(x) \wedge \delta \phi(x)
\, d\mu_g.
\]

For functionals $F,G$,
\[
\{F,G\}
=
\int_{\mathcal{M}}
\left(
\frac{\delta F}{\delta \phi}
\frac{\delta G}{\delta \pi}
-
\frac{\delta F}{\delta \pi}
\frac{\delta G}{\delta \phi}
\right)
d\mu_g.
\]

The flow generated by $H$ satisfies
\[
\dot{F}
=
\{F,H\}.
\]

\subsection*{Energy Positivity and Spinodal Instability}

The quadratic Hamiltonian around a homogeneous state $\phi_0$ is
\[
H^{(2)}
=
\frac{1}{2}
\int
\left(
\pi^2
+
c^2 |\nabla\eta|^2
+
\kappa_{\rm eff} (\Delta \eta)^2
+
V''(\phi_0)\eta^2
\right)
d\mu_g,
\]
where $\phi = \phi_0 + \eta$.

Fourier decomposition yields
\[
H^{(2)}
=
\frac{1}{2}
\sum_k
\left(
|\pi_k|^2
+
\omega_k^2 |\eta_k|^2
\right),
\]
with
\[
\omega_k^2
=
c^2 k^2
+
\kappa_{\rm eff} k^4
+
V''(\phi_0).
\]

If $V''(\phi_0) < 0$, then for sufficiently small $k$,
\[
\omega_k^2 < 0.
\]

Thus $H^{(2)}$ becomes indefinite but remains bounded from below in the ultraviolet because
\[
\omega_k^2 \to +\infty \quad \text{as } k \to \infty.
\]

\subsection*{Finite-Dimensional Unstable Subspace}

Define the unstable subspace
\[
\mathcal{U}
=
\mathrm{span}
\left\{
\psi_k \mid \omega_k^2 < 0
\right\}.
\]

Because the $k^4$ term dominates at large $k$,
\[
\dim \mathcal{U} < \infty.
\]

Hence the Hamiltonian flow decomposes as
\[
\mathcal{P}
=
\mathcal{U}
\oplus
\mathcal{S},
\]
where $\mathcal{S}$ is the stable infinite-dimensional subspace.

\subsection*{Stable and Unstable Manifolds}

Let $(\phi_0,0)$ denote the homogeneous equilibrium.

Linearizing Hamilton’s equations yields
\[
\dot{\mathbf{X}}
=
\mathbb{J}
\mathbb{H}
\mathbf{X},
\]
where
\[
\mathbf{X}
=
\begin{pmatrix}
\eta \\
\pi
\end{pmatrix},
\qquad
\mathbb{J}
=
\begin{pmatrix}
0 & 1 \\
-1 & 0
\end{pmatrix}.
\]

Eigenvalues are
\[
\lambda_k = \pm \sqrt{-\omega_k^2}.
\]

Thus unstable directions generate exponential growth:
\[
\eta_k(t)
\sim
e^{\sqrt{|\omega_k^2|}\,t}.
\]

The local unstable manifold is finite-dimensional and symplectically orthogonal to the stable manifold.

\subsection*{No Ostrogradsky Instability}

Although the equation is fourth order in space, it remains second order in time.

Therefore the Hamiltonian involves only first-order time derivatives, and no additional canonical momentum is introduced.

Hence there is no Ostrogradsky instability associated with higher time derivatives.

\subsection*{Conserved Quantities}

Time-translation invariance implies
\[
\frac{dH}{dt} = 0.
\]

If $V$ depends only on $\phi^2$, the theory admits a $\mathbb{Z}_2$ symmetry
\[
\phi \mapsto -\phi.
\]

If spatial translations are present,
\[
P_i
=
\int \pi \partial_i \phi\, d\mu_g
\]
is conserved.

\subsection*{Linear Mode Decoupling}

The Hamiltonian splits into independent oscillators:
\[
H^{(2)}
=
\sum_k H_k,
\quad
H_k
=
\frac{1}{2}
\left(
|\pi_k|^2
+
\omega_k^2 |\eta_k|^2
\right).
\]

For $\omega_k^2>0$, each mode is a stable harmonic oscillator.

For $\omega_k^2<0$, each mode is an inverted oscillator.

\subsection*{Phase-Space Volume and Liouville Theorem}

The Hamiltonian flow preserves the symplectic volume form
\[
\Omega^{\wedge N},
\]
where $N$ is the (formal) dimension of phase space.

Thus scalar irruption corresponds to hyperbolic flow in a finite-dimensional sector of phase space while preserving global Liouville measure.

The regulated scalar sector:

\[
\text{(i)}\ \text{admits a well-defined symplectic structure,}
\]
\[
\text{(ii)}\ \text{has a conserved Hamiltonian functional,}
\]
\[
\text{(iii)}\ \text{contains a finite-dimensional unstable manifold in the spinodal regime,}
\]
\[
\text{(iv)}\ \text{remains free of higher-time-derivative instabilities.}
\]

Thus the dynamical instability underlying scalar irruption is a controlled hyperbolic sector embedded within a globally Hamiltonian field theory.

\section{Spectral Geometry of the Regulated Laplacian and Weyl Asymptotics}

\subsection*{Elliptic Operators on Compact Manifolds}

Let $(\mathcal{M},g)$ be a smooth compact $d$-dimensional Riemannian manifold without boundary. The Laplace--Beltrami operator
\[
\Delta_g : C^\infty(\mathcal{M}) \to C^\infty(\mathcal{M})
\]
is essentially self-adjoint on $L^2(\mathcal{M})$ with discrete spectrum
\[
0 = \lambda_0 \le \lambda_1 \le \lambda_2 \le \cdots \to \infty,
\]
where
\[
-\Delta_g \psi_k = \lambda_k \psi_k,
\qquad
\{\psi_k\}
\text{ forms an orthonormal basis of } L^2(\mathcal{M}).
\]

Consider the regulated elliptic operator
\[
\mathcal{A}
=
- c^2 \Delta_g
+
\kappa_{\rm eff} \Delta_g^2
+
m^2,
\]
with $c^2>0$ and $\kappa_{\rm eff}>0$.

\subsection*{Principal Symbol and Strong Ellipticity}

The principal symbol of $\mathcal{A}$ is
\[
\sigma_{\rm pr}(\mathcal{A})(x,\xi)
=
\kappa_{\rm eff} |\xi|^4,
\]
which is positive for all $\xi\neq 0$.

Hence $\mathcal{A}$ is strongly elliptic of order four.

Strong ellipticity implies:

\[
\text{(i)}\ \text{compact resolvent},
\qquad
\text{(ii)}\ \text{discrete spectrum},
\qquad
\text{(iii)}\ \text{finite multiplicities}.
\]

\subsection*{Spectral Representation}

Applying $\mathcal{A}$ to Laplacian eigenfunctions yields
\[
\mathcal{A}\psi_k
=
\left(
c^2 \lambda_k
+
\kappa_{\rm eff} \lambda_k^2
+
m^2
\right)\psi_k.
\]

Define
\[
\omega_k^2
=
c^2 \lambda_k
+
\kappa_{\rm eff} \lambda_k^2
+
m^2.
\]

Thus the regulated operator diagonalizes in the Laplacian basis.

\subsection*{Spinodal Condition and Instability Band}

Instability occurs when
\[
\omega_k^2 < 0.
\]

Let $m^2 = V''(\phi_0)$ with $m^2<0$.

The condition becomes
\[
\kappa_{\rm eff} \lambda_k^2
+
c^2 \lambda_k
+
m^2
<
0.
\]

This is a quadratic inequality in $\lambda_k$.

The roots are
\[
\lambda_{\pm}
=
\frac{-c^2 \pm \sqrt{c^4 - 4\kappa_{\rm eff} m^2}}{2\kappa_{\rm eff}}.
\]

Because $m^2<0$ and $\kappa_{\rm eff}>0$, the discriminant satisfies
\[
c^4 - 4\kappa_{\rm eff} m^2 > c^4,
\]
and both roots are real with
\[
\lambda_- < 0 < \lambda_+.
\]

Thus instability requires
\[
0 < \lambda_k < \lambda_+.
\]

\subsection*{Finiteness of the Unstable Sector}

Because $\lambda_k \to \infty$ as $k\to\infty$,
there exist only finitely many eigenvalues satisfying
\[
\lambda_k < \lambda_+.
\]

Hence the unstable sector
\[
\mathcal{U}
=
\mathrm{span}\{\psi_k : \lambda_k < \lambda_+\}
\]
is finite-dimensional.

This establishes ultraviolet control rigorously.

\subsection*{Weyl Asymptotics}

Weyl’s law states that the counting function
\[
N(\Lambda)
=
\#\{\lambda_k \le \Lambda\}
\]
satisfies
\[
N(\Lambda)
\sim
\frac{\omega_d}{(2\pi)^d}
\operatorname{Vol}(\mathcal{M})
\,
\Lambda^{d/2}
\quad
\text{as }\Lambda\to\infty,
\]
where $\omega_d$ is the volume of the unit ball in $\mathbb{R}^d$.

Therefore,
\[
\dim \mathcal{U}
=
N(\lambda_+)
\sim
C_d\, \operatorname{Vol}(\mathcal{M})\, \lambda_+^{d/2}.
\]

Substituting $\lambda_+$ gives
\[
\dim \mathcal{U}
\sim
C_d\, \operatorname{Vol}(\mathcal{M})
\left(
\frac{-c^2 + \sqrt{c^4 - 4\kappa_{\rm eff} m^2}}
{2\kappa_{\rm eff}}
\right)^{d/2}.
\]

Thus the number of unstable modes scales polynomially with system size and with the regulator parameters.

\subsection*{Short-Wavelength Stability}

As $\lambda_k \to \infty$,
\[
\omega_k^2
\sim
\kappa_{\rm eff} \lambda_k^2 \to +\infty.
\]

Hence
\[
\omega_k^2 > 0
\quad
\text{for all sufficiently large } k.
\]

The fourth-order term guarantees spectral positivity in the ultraviolet.

\subsection*{Resolvent Bounds}

For $\lambda \notin \mathrm{Spec}(\mathcal{A})$,
the resolvent
\[
(\mathcal{A}-\lambda)^{-1}
:
L^2(\mathcal{M}) \to H^4(\mathcal{M})
\]
is compact.

Thus the spectrum consists only of isolated eigenvalues of finite multiplicity.

\subsection*{Boundary Effects}

If $\mathcal{M}$ has boundary,
impose either:

Dirichlet:
\[
\phi|_{\partial\mathcal{M}}=0,
\]

or Neumann:
\[
\nabla_n \phi|_{\partial\mathcal{M}}=0.
\]

Strong ellipticity persists under standard boundary conditions, and discreteness of the spectrum remains valid.

\subsection*{Band-Limited Growth}

Let $\gamma_k = \sqrt{-\omega_k^2}$ for unstable modes.

Then
\[
\gamma_k^2
=
-
\left(
c^2 \lambda_k
+
\kappa_{\rm eff} \lambda_k^2
+
m^2
\right).
\]

The maximum growth rate occurs at
\[
\frac{d}{d\lambda}
\left(
-
c^2 \lambda
-
\kappa_{\rm eff} \lambda^2
-
m^2
\right)
=
0,
\]
yielding
\[
\lambda_*
=
-\frac{c^2}{2\kappa_{\rm eff}}.
\]

Because $\lambda_k>0$, the physically relevant extremum lies within the admissible range only when $m^2$ is sufficiently negative.

Thus the fastest-growing eigenmode lies strictly within the finite band.

The regulated operator:

\[
\text{(i)}\ \text{is strongly elliptic of order four,}
\]
\[
\text{(ii)}\ \text{has discrete spectrum on compact manifolds,}
\]
\[
\text{(iii)}\ \text{admits only finitely many unstable modes,}
\]
\[
\text{(iv)}\ \text{satisfies Weyl asymptotics in the ultraviolet.}
\]

Scalar irruption therefore corresponds to a finite-dimensional spectral bifurcation within an otherwise elliptic geometric operator.

\section{Analytic Semigroup Generation and Well-Posedness}

\subsection*{Abstract Evolution Formulation}

Let $(\mathcal{M},g)$ be a compact Riemannian manifold without boundary.  
Define the Hilbert space
\[
\mathcal{H} := L^2(\mathcal{M}),
\]
with domain
\[
D(\mathcal{A}) := H^4(\mathcal{M}),
\]
where $\mathcal{A}$ is the fourth-order regulated operator
\[
\mathcal{A}
=
- c^2 \Delta_g
+
\kappa_{\rm eff} \Delta_g^2
+
m^2,
\qquad
\kappa_{\rm eff}>0.
\]

The linearized scalar evolution equation may be written as
\[
\partial_t \phi = -\mathcal{A}\phi.
\]

\subsection*{Sectoriality}

Because $\mathcal{A}$ is strongly elliptic of order four and self-adjoint on $L^2(\mathcal{M})$, it is sectorial. That is, there exists $\theta \in (0,\pi/2)$ and $M>0$ such that the resolvent estimate
\[
\|(\lambda I + \mathcal{A})^{-1}\|
\le
\frac{M}{|\lambda|}
\]
holds for all $\lambda$ outside a sector
\[
\{ \lambda \in \mathbb{C} : |\arg \lambda| < \pi - \theta \}.
\]

\subsection*{Generation of Analytic Semigroup}

By standard results in semigroup theory for sectorial operators,  
$-\mathcal{A}$ generates a strongly continuous analytic semigroup
\[
e^{-t\mathcal{A}} : \mathcal{H} \to \mathcal{H},
\qquad t\ge 0.
\]

Thus for any initial data
\[
\phi_0 \in \mathcal{H},
\]
there exists a unique mild solution
\[
\phi(t) = e^{-t\mathcal{A}} \phi_0.
\]

If $\phi_0 \in H^4(\mathcal{M})$, the solution is classical and satisfies
\[
\phi \in C^1([0,T];L^2)
\cap
C([0,T];H^4).
\]

\subsection*{Continuous Dependence}

The semigroup estimate
\[
\|e^{-t\mathcal{A}}\|
\le
C e^{\omega t}
\]
holds for some $\omega \in \mathbb{R}$ determined by the spectral bound of $\mathcal{A}$.

Therefore
\[
\|\phi(t)\|_{L^2}
\le
C e^{\omega t}
\|\phi_0\|_{L^2},
\]
establishing continuous dependence on initial data.

\subsection*{Spectral Decomposition of the Semigroup}

Using the Laplacian eigenbasis $\{\psi_k\}$,
\[
\phi_0 = \sum_k a_k \psi_k,
\]
the solution evolves as
\[
\phi(t)
=
\sum_k a_k e^{-t \omega_k^2} \psi_k,
\]
where
\[
\omega_k^2
=
c^2 \lambda_k
+
\kappa_{\rm eff} \lambda_k^2
+
m^2.
\]

Unstable modes correspond precisely to $\omega_k^2<0$ and produce exponential growth
\[
e^{|\omega_k^2|t}.
\]

Because only finitely many such $k$ exist, instability remains finite-dimensional.

\subsection*{Energy Functional and Dissipativity}

Define the quadratic energy
\[
E[\phi]
=
\frac{1}{2}
\int_{\mathcal{M}}
\left(
\kappa_{\rm eff} |\Delta_g \phi|^2
+
c^2 |\nabla \phi|^2
+
m^2 \phi^2
\right)
d\mu_g.
\]

Then
\[
\frac{d}{dt}E[\phi]
=
-
\|\mathcal{A}^{1/2}\phi\|_{L^2}^2
\]
for the stable sector.

Thus the semigroup is dissipative on the complement of the unstable eigenspace.

\subsection*{Nonlinear Extension}

Consider a nonlinear perturbation
\[
\partial_t \phi
=
-\mathcal{A}\phi
+
\mathcal{N}(\phi),
\]
with $\mathcal{N}$ locally Lipschitz from $H^2$ to $L^2$ and of at most polynomial growth.

By standard fixed-point arguments in Banach spaces,  
there exists $T>0$ such that a unique local solution exists for initial data in $H^2$.

Global existence holds provided
\[
\|\phi(t)\|_{H^2}
\]
remains bounded.

\subsection*{Band-Limited Instability and Stable Manifold}

Let
\[
\mathcal{H}
=
\mathcal{U} \oplus \mathcal{S}
\]
be the decomposition into unstable and stable spectral subspaces.

There exists a finite-dimensional unstable manifold tangent to $\mathcal{U}$ at $\phi=0$.

The center-stable manifold theorem ensures that nonlinear dynamics near criticality are governed by the finite-dimensional unstable eigenspace.

The regulated scalar evolution:

\[
\text{(i)}\ \text{generates an analytic semigroup,}
\]
\[
\text{(ii)}\ \text{admits unique solutions for admissible initial data,}
\]
\[
\text{(iii)}\ \text{has finite-dimensional instability,}
\]
\[
\text{(iv)}\ \text{remains ultraviolet well-posed.}
\]

Scalar irruption is therefore a controlled spectral bifurcation within a rigorously well-defined evolution system.

\section{Center Manifold Reduction and Amplitude Equation Near Criticality}

\subsection*{Spectral Setup Near the Critical Threshold}

Let the regulated linear operator be
\[
\mathcal{A}_\mu
=
- c^2 \Delta_g
+
\kappa_{\rm eff} \Delta_g^2
+
\mu,
\]
where $\mu$ is a control parameter proportional to the effective entropy curvature (for example $\mu \sim f''(\bar n)$ in the coarse-grained description).

Assume there exists a critical value $\mu=\mu_c$ such that
\[
\omega_{k_*}^2(\mu_c)=0
\]
for a single eigenmode $\psi_{k_*}$, while all other eigenvalues satisfy
\[
\omega_k^2(\mu_c) > 0.
\]

Thus the spectrum decomposes as
\[
\mathcal{H}
=
\mathrm{span}\{\psi_{k_*}\}
\oplus
\mathcal{H}_s,
\]
with $\mathcal{H}_s$ strictly stable.

\subsection*{Nonlinear Evolution Equation}

Consider the nonlinear evolution
\[
\partial_t \phi
=
-\mathcal{A}_\mu \phi
+
\mathcal{N}(\phi),
\]
where $\mathcal{N}$ is smooth and at least quadratic:
\[
\mathcal{N}(\phi)
=
\lambda_2 \phi^2
+
\lambda_3 \phi^3
+
\cdots.
\]

\subsection*{Center Manifold Existence}

At $\mu=\mu_c$, the operator has a simple zero eigenvalue.  
Standard center manifold theory implies:

There exists a one-dimensional center manifold
\[
\mathcal{W}^c
=
\{ A \psi_{k_*} + h(A) : A \in \mathbb{R} \},
\]
where $h(A)\in \mathcal{H}_s$ and
\[
h(A)=O(A^2).
\]

Dynamics near $\phi=0$ reduce to dynamics on this manifold.

\subsection*{Amplitude Equation}

Write
\[
\phi(x,t)
=
A(t)\psi_{k_*}(x)
+
h(A(t)).
\]

Projecting the evolution equation onto $\psi_{k_*}$ yields
\[
\dot A
=
\sigma(\mu) A
-
g A^3
+
O(A^5),
\]
where

\[
\sigma(\mu)
=
-\omega_{k_*}^2(\mu),
\qquad
g>0
\]
for a supercritical bifurcation (assuming stabilizing cubic nonlinearity).

\subsection*{Normal Form}

Near criticality,
\[
\sigma(\mu)
\approx
\alpha(\mu-\mu_c),
\]
with $\alpha>0$.

Thus the reduced normal form becomes
\[
\dot A
=
\alpha(\mu-\mu_c) A
-
g A^3.
\]

\subsection*{Steady States}

For $\mu<\mu_c$:
\[
A=0
\]
is stable.

For $\mu>\mu_c$:
\[
A=0
\]
is unstable and two nontrivial equilibria appear:
\[
A_{\pm}
=
\pm
\sqrt{\frac{\alpha(\mu-\mu_c)}{g}}.
\]

This is a supercritical pitchfork bifurcation.

\subsection*{Finite-Dimensional Reduction}

Near the instability threshold, the full PDE reduces to the single ODE
\[
\dot A
=
\alpha(\mu-\mu_c)A
-
gA^3,
\]
which governs the onset of scalar irruption.

Thus the instability is dynamically finite-dimensional in the precise sense that only a bounded subset of spectral modes lies within the unstable band determined by the regulator and background parameters. The nonlinear evolution saturates at finite amplitude because higher-order terms in the free-energy functional render the homogeneous branch energetically unfavorable but prevent unbounded descent along any direction in configuration space. Growth is therefore excluded at arbitrarily small scales: the ultraviolet sector is controlled by the positive-definite higher-order operator, which forces the dispersion relation to become strictly stabilizing as the wavenumber increases. The ensuing nonlinear phase transition is structurally stable under small perturbations of parameters and initial data, since the bifurcation arises from a change in the sign of a finite number of eigenvalues rather than from any essential singularity in the governing operator.

\subsection*{Spatial Pattern Formation}

The leading-order solution in the broken-symmetry phase is
\[
\phi(x)
\approx
A_{\pm}\psi_{k_*}(x),
\]
so the spatial structure of the irrupted domain is inherited from the critical Laplacian eigenmode.

If the critical eigenspace is multi-dimensional (e.g., degeneracy on symmetric manifolds), the reduced dynamics generalize to a vector amplitude equation
\[
\dot{\mathbf{A}}
=
\sigma \mathbf{A}
-
g |\mathbf{A}|^2 \mathbf{A},
\]
yielding pattern-selection dynamics.

\subsection*{Interpretation}

The scalar irruption transition may therefore be characterized in several mathematically equivalent ways. It appears first as a spectral instability occurring at the critical condition , where a discrete mode crosses from positive to negative effective frequency squared. In parameter space, this transition constitutes a codimension-one bifurcation, since it is triggered by the variation of a single effective control parameter---such as the curvature  or the regulator-weighted compressibility---through a critical value. Near threshold, the dynamics reduce to a finite-dimensional amplitude system governed by the unstable eigenspace, with higher modes remaining linearly stable and slaved to the dominant growing sector. Under generic stabilizing nonlinearities in the free-energy functional, the bifurcation is supercritical, leading to bounded pattern formation rather than runaway divergence and thus defining a controlled phase transition in the scalar sector.

No ultraviolet catastrophe arises because:

\[
\text{(i)}\ \text{only finitely many modes destabilize,}
\]
\[
\text{(ii)}\ \text{higher modes remain strictly damped,}
\]
\[
\text{(iii)}\ \text{nonlinear saturation bounds amplitude.}
\]

\subsection*{Corollary: Entropic Criticality}

If $\mu$ is identified with the coarse-grained entropy curvature parameter,
\[
\mu \propto -f''(\bar n),
\]
then the transition $\mu=\mu_c$ corresponds precisely to crossing the spinodal boundary.

Scalar irruption is thus the dynamical realization of entropic criticality.

The regulated scalar system undergoes a mathematically standard supercritical bifurcation with:

\[
\text{finite-dimensional instability,}
\quad
\text{bounded amplitude,}
\quad
\text{structurally stable phase selection.}
\]

The instability is therefore not merely spectral but dynamically organized by a universal amplitude equation near threshold.

\section{Derived Critical Locus and Homotopy-Type Transition}

\subsection*{Variational Setup}

Let $\mathcal{M}$ be a compact Riemannian manifold without boundary and consider the classical action functional

S_{\mathrm{cl}}[\phi]
=
\int_{\mathcal{M}}
\left(
\frac{c^2}{2} |\nabla \phi|^2
+
\frac{\kappa_{\rm eff}}{2} |\Delta_g \phi|^2
+
V(\phi;\mu)
\right)
\, d\mu_g,

V''(0;\mu) = \mu.

The Euler–Lagrange equation is

\mathcal{E}(\phi)
=
- c^2 \Delta_g \phi
+
\kappa_{\rm eff} \Delta_g^2 \phi
+
V'(\phi;\mu)
=
0.

Define the critical locus

\mathrm{Crit}(S_{\mathrm{cl}})
=
\{ \phi \in H^4(\mathcal{M}) : \mathcal{E}(\phi)=0 \}.

\subsection*{Hessian and Morse Index}

The second variation at $\phi=0$ is

\delta^2 S_{\mathrm{cl}}(0)
=
\int_{\mathcal{M}}
\left(
c^2 |\nabla \psi|^2
+
\kappa_{\rm eff} |\Delta_g \psi|^2
+
\mu \psi^2
\right)
\, d\mu_g.

In spectral coordinates relative to Laplacian eigenfunctions $\psi_k$ satisfying

-\Delta_g \psi_k = \lambda_k \psi_k,

Q_k(\mu)
=
c^2 \lambda_k
+
\kappa_{\rm eff} \lambda_k^2
+
\mu.

The Morse index of the trivial critical point is

\mathrm{Ind}(0;\mu)
=
\#\{ k : Q_k(\mu) < 0 \}.

For $\mu$ sufficiently positive, all $Q_k>0$ and the index vanishes.
At critical parameter values satisfying

Q_{k_*}(\mu_c)=0,

\subsection*{Change in Critical Topology}

Let $\mu<\mu_c$ so that $Q_{k_*}(\mu)<0$ for a finite set of modes. Then the trivial solution $\phi=0$ is no longer a local minimum but a saddle with Morse index equal to the number of unstable eigenvalues.

By the Morse lemma, near nondegenerate critical points the action is locally equivalent to

S_{\mathrm{cl}}
=
- x_1^2 - \cdots - x_m^2
+
y_1^2 + \cdots + y_n^2
+
\text{higher-order terms}.

When a single eigenvalue crosses zero, the topology of the sublevel sets

\{ \phi : S_{\mathrm{cl}}[\phi] \leq c \}

Thus the lamphron–lamphrodyne transition corresponds to a change in the Morse index of the trivial configuration and therefore to a change in the homotopy type of the configuration space sublevel sets.

\subsection*{Derived Critical Locus}

In the BV formalism, the classical equations define the critical locus as a derived scheme

\mathbb{R}\mathrm{Crit}(S_{\mathrm{cl}})
=
\mathrm{Spec}
\left(
\mathrm{Sym}\big(T_{\phi}[-1]\big)
\right)

\mathcal{D}
=
D\mathcal{E}(\phi).

At $\phi=0$, this operator is

\mathcal{D}_0
=
- c^2 \Delta_g
+
\kappa_{\rm eff} \Delta_g^2
+
\mu.

For $\mu\neq\mu_c$, the complex

H^4(\mathcal{M})
\xrightarrow{\mathcal{D}_0}
L^2(\mathcal{M})

At $\mu=\mu_c$, the kernel acquires a nontrivial finite-dimensional component spanned by the critical eigenmodes.

Thus the derived critical locus acquires higher homology in degree $-1$, corresponding to new ghost-number-one directions in the BV complex.

\subsection*{Homotopy-Type Transition}

Let $X_\mu = \mathrm{Crit}(S_{\mathrm{cl}};\mu)$ denote the classical solution space. Then as $\mu$ crosses $\mu_c$:

The trivial solution loses nondegeneracy.

New nontrivial branches of solutions emerge, parameterized by the center manifold.

The homotopy type of $X_\mu$ changes by attachment of cells corresponding to unstable modes.

In derived language, the cotangent complex

\mathbb{L}_{X_\mu}

\subsection*{Geometric Interpretation of Lamphron and Lamphrodyne States}

Lamphron states correspond to regions in parameter space where the Hessian of the action is positive definite and the critical locus is discrete and nondegenerate.

Lamphrodyne states correspond to regions where the Hessian becomes indefinite, the Morse index increases, and new branches of solutions appear.

Thus the lamphron–lamphrodyne distinction is equivalent to the statement that the classical moduli space undergoes a bifurcation with a change in derived structure.

\subsection*{Spectral Flow Interpretation}

Let $\mathcal{D}(\mu)$ denote the linearized operator.
The spectral flow of $\mathcal{D}(\mu)$ through zero counts the net number of eigenvalues changing sign as $\mu$ varies.

The spectral flow equals the change in Morse index:

\mathrm{SF}(\mathcal{D};\mu_c)
=
\Delta \mathrm{Ind}.

This spectral flow measures precisely the number of new unstable directions and therefore the dimension of the emerging center manifold.

The lamphron–lamphrodyne transition is not merely analytic instability but a genuine change in the topology of the critical locus of the action functional.

In classical terms, it is a Morse bifurcation.

In derived terms, it is a modification of the cotangent complex and of the homotopy type of the moduli space of solutions.

In dynamical terms, it coincides with the emergence of finite-dimensional amplitude dynamics.

The entropic sign reversal therefore manifests simultaneously as spectral flow, Morse index jump, center-manifold bifurcation, and derived geometric transition.

\section{Renormalization of the Gradient Regulator and Effective Field Theory Scaling}

Consider the regulated action on a compact Riemannian manifold $(\mathcal{M},g)$

S[\phi]
=
\int_{\mathcal{M}}
\left(
\frac{1}{2} (\partial \phi)^2
+
\frac{\kappa}{2} (\Delta_g \phi)^2
+
V(\phi)
\right)
\, d\mu_g,

In flat space with Euclidean signature and metric $\delta_{ij}$, the quadratic part reads

S_2
=
\frac{1}{2}
\int d^d x
\left(
\partial_i \phi \partial_i \phi
+
\kappa (\Delta \phi)^2
+
m^2 \phi^2
\right),

Passing to momentum space,

S_2
=
\frac{1}{2}
\int \frac{d^d k}{(2\pi)^d}
\,
\tilde{\phi}(k)
\left(
k^2 + \kappa k^4 + m^2
\right)
\tilde{\phi}(-k).

The propagator is therefore

G(k)
=
\frac{1}{k^2 + \kappa k^4 + m^2}.

For $k \to \infty$, one has

G(k) \sim \frac{1}{\kappa k^4},

Let $\Lambda$ denote a momentum cutoff. Integrating out modes in the shell $\Lambda/b < |k| < \Lambda$ with $b>1$ generates renormalized couplings. Under the Wilsonian rescaling

k \mapsto k' = b k,
\qquad
x \mapsto x' = x/b,

\phi(x) = b^{\Delta_\phi} \phi'(x'),

\int d^d x\, (\partial \phi)^2
\mapsto
b^{d-2+2\Delta_\phi}
\int d^d x' (\partial' \phi')^2.

The regulator term scales as

\int d^d x\, \kappa (\Delta \phi)^2
\mapsto
\kappa\, b^{d-4+2\Delta_\phi}
\int d^d x' (\Delta' \phi')^2.

Using $\Delta_\phi = (d-2)/2$, the scaling exponent of $\kappa$ is

d - 4 + d - 2 = 2d - 6,

[\kappa] = -1.

Thus $\kappa$ has dimensions of length squared and is irrelevant in the renormalization-group sense at low momenta but dominates in the ultraviolet. This establishes that the gradient regulator is naturally interpreted as a higher-derivative correction in an effective field theory valid below a coarse-graining scale $\ell \sim \sqrt{\kappa}$.

In curved spacetime, one replaces $k^2$ by eigenvalues of $-\Delta_g$. The same asymptotic argument applies because Weyl’s law implies that large eigenvalues scale as $\lambda_k \sim k^{2/d}$ in counting measure, and the quartic term dominates asymptotically.

Therefore the regulator defines an effective theory with controlled ultraviolet behavior and a natural coarse-graining scale. Its renormalization produces subleading corrections to $m^2$ and to the quartic coupling in $V(\phi)$ but does not destabilize the ultraviolet completeness of the linearized operator so long as $\kappa>0$.

\section{Measure-Theoretic Entropy Functional and Coarse-Graining Limits}

Let $(\mathcal{M},\mathcal{B},\mu)$ be a finite measure space with $\mu(\mathcal{M})=1$ for normalization. Let $\rho \in L^1(\mathcal{M},\mu)$ with $\rho \ge 0$ and $\int \rho, d\mu = 1$.

Define the Shannon entropy functional

\mathcal{S}[\rho]
=
- \int_{\mathcal{M}} \rho \log \rho \, d\mu.

For $\rho$ absolutely continuous with respect to $\mu$, $\mathcal{S}$ is well-defined and concave on the convex set of probability densities.

Let $\mathcal{P}\ell$ denote a coarse-graining operator defined by convolution with a mollifier $K\ell$,

\rho_\ell(x)
=
(\mathcal{P}_\ell \rho)(x)
=
\int_{\mathcal{M}} K_\ell(x,y) \rho(y) \, d\mu(y),

Then $\rho_\ell$ defines a smoothed density. By Jensen’s inequality,

\mathcal{S}[\rho_\ell]
\ge
\mathcal{S}[\rho],

Consider now a free-energy functional

\mathcal{F}[\rho]
=
\int
\left(
f(\rho)
+
\frac{\kappa}{2} |\nabla \rho|^2
\right)
d\mu_g,

In the limit $\ell \to 0$, one has $\rho_\ell \to \rho$ in $L^1$, and the gradient term converges in the sense of distributions provided $\rho \in H^1(\mathcal{M})$.

Under successive coarse-graining scales $\ell_n \to \ell_\infty$, the functional $\mathcal{F}$ flows to an effective functional

\mathcal{F}_{\mathrm{eff}}[\rho]
=
\int
\left(
f_{\mathrm{eff}}(\rho)
+
\frac{\kappa_{\mathrm{eff}}}{2} |\nabla \rho|^2
\right)
d\mu_g,

In measure-theoretic terms, the lamphron–lamphrodyne transition corresponds to the change in sign of the second variation

\delta^2 \mathcal{F}[\rho_0](\psi,\psi)
=
\int
\left(
f''(\rho_0)\psi^2
+
\kappa |\nabla \psi|^2
\right)
d\mu_g.

When $f''(\rho_0)>0$, the quadratic form is positive definite in $H^1(\mathcal{M})$. When $f''(\rho_0)<0$, the form becomes indefinite but remains bounded below due to the gradient term.

Thus in the weak topology of $H^1$, the entropy functional admits a well-defined second variation and a controlled instability band characterized by the competition between the concavity of $f$ and the Sobolev norm induced by $\kappa$.

From the standpoint of probability measures, the instability does not violate conservation of total mass

\int_{\mathcal{M}} \rho \, d\mu = 1,

This measure-theoretic formulation makes explicit that the instability is not a violation of normalization or of conservation but a redistribution in the space of densities governed by a concave–convex competition in the free-energy functional.

\section{Well-Posedness, Semigroup Generation, and Spectral Band Structure of the Regulated Operator}

Let $(\mathcal{M},g)$ be a smooth compact Riemannian manifold without boundary. Consider the linearized regulated scalar operator acting on $H^2(\mathcal{M})$,

\mathcal{L}
=
- \kappa \Delta_g^2
+
a \Delta_g
+
m^2,

\partial_t \phi
=
- \mathcal{L} \phi.

\subsection*{Self-adjointness and ellipticity}

Since $-\Delta_g$ is a positive self-adjoint operator on $L^2(\mathcal{M})$ with discrete spectrum ${\lambda_k}_{k=0}^\infty$, and since $\Delta_g^2$ is also self-adjoint with domain $H^4(\mathcal{M})$, it follows that $\mathcal{L}$ is essentially self-adjoint on $C^\infty(\mathcal{M})$ and extends to a self-adjoint operator on $L^2(\mathcal{M})$ with domain $H^4(\mathcal{M})$.

The principal symbol of $\mathcal{L}$ is

\sigma_{\mathrm{prin}}(\mathcal{L})(x,\xi)
=
\kappa |\xi|_g^4,

\subsection*{Spectral decomposition}

Let ${-\Delta_g \psi_k = \lambda_k \psi_k}$ be an orthonormal eigenbasis of $L^2(\mathcal{M})$, with $0=\lambda_0 \le \lambda_1 \le \lambda_2 \le \dots$ and $\lambda_k \to \infty$.

Then

\mathcal{L} \psi_k
=
\left(
\kappa \lambda_k^2
-
a \lambda_k
+
m^2
\right)\psi_k.

Denote the eigenvalues of $\mathcal{L}$ by

\mu_k
=
\kappa \lambda_k^2
-
a \lambda_k
+
m^2.

The spectrum is discrete and unbounded above because $\mu_k \sim \kappa \lambda_k^2$ as $k\to\infty$.

\subsection*{Semigroup generation}

Define the operator

A = -\mathcal{L}.

Since $\mathcal{L}$ is self-adjoint and bounded below, $A$ is self-adjoint and bounded above. The Hille–Yosida theorem implies that $A$ generates a strongly continuous semigroup $e^{tA}$ on $L^2(\mathcal{M})$.

Explicitly, for initial data $\phi_0 \in L^2(\mathcal{M})$,

\phi(t)
=
e^{-t\mathcal{L}} \phi_0
=
\sum_{k=0}^\infty
e^{-t\mu_k}
\langle \phi_0,\psi_k\rangle
\psi_k.

Hence the Cauchy problem is well posed in $L^2(\mathcal{M})$, and in fact in $H^s(\mathcal{M})$ for all $s\ge 0$ by elliptic regularity.

\subsection*{Band-limited instability}

Instability corresponds to modes with $\mu_k < 0$. Since $\mu_k$ is a quadratic polynomial in $\lambda_k$,

\mu(\lambda)
=
\kappa \lambda^2
-
a \lambda
+
m^2.

If $a>0$, the quadratic has a minimum at

\lambda_* = \frac{a}{2\kappa},

\mu_{\min}
=
m^2
-
\frac{a^2}{4\kappa}.

Instability occurs if and only if

m^2
<
\frac{a^2}{4\kappa}.

In that case the inequality $\mu(\lambda)<0$ holds for $\lambda$ in the interval

\lambda_- < \lambda < \lambda_+,

\lambda_{\pm}
=
\frac{a \pm \sqrt{a^2 - 4\kappa m^2}}{2\kappa}.

Since $\lambda_k \to \infty$, only finitely many eigenvalues lie in this interval. Therefore the unstable spectrum is finite-dimensional.

In particular, the number of unstable modes equals the number of Laplacian eigenvalues $\lambda_k$ contained in $(\lambda_-,\lambda_+)$.

\subsection*{Ultraviolet control}

Because $\mu_k \sim \kappa \lambda_k^2$ as $\lambda_k \to \infty$, one has

\mu_k \to +\infty
\quad\text{as}\quad
k\to\infty.

Thus high-frequency modes are strictly stable. The instability, when present, is necessarily band-limited and cannot extend to arbitrarily large wavenumber.

\subsection*{Energy estimate}

Define the quadratic energy functional

E[\phi]
=
\frac{1}{2}
\int_{\mathcal{M}}
\left(
\kappa (\Delta_g \phi)^2
-
a |\nabla \phi|^2
+
m^2 \phi^2
\right)
d\mu_g.

Then for smooth solutions of $\partial_t \phi = -\mathcal{L}\phi$,

\frac{d}{dt}
\|\phi\|_{L^2}^2
=
-2
\langle \phi, \mathcal{L}\phi \rangle
=
-2
\sum_k
\mu_k
|\langle \phi,\psi_k\rangle|^2.

Therefore, growth occurs only along the finite set of eigenfunctions with $\mu_k<0$, and the growth rate is exactly $|\mu_k|$ in each such eigendirection.

The regulated fourth-order operator is uniformly elliptic, generates a strongly continuous semigroup, possesses a discrete spectrum with finite unstable band when parameters satisfy the instability condition, and enforces strict ultraviolet stabilization through the $\kappa \lambda_k^2$ term. The instability is therefore spectrally confined, mathematically well posed, and structurally compatible with conservative evolution in a compact spatial domain.

\newpage
\begin{thebibliography}{99}

\bibitem{Arnold1989}
Arnold, V. I. (1989).
\textit{Mathematical Methods of Classical Mechanics}.
Springer.

\bibitem{BatalinVilkovisky1981}
Batalin, I. A., and Vilkovisky, G. A. (1981).
Gauge algebra and quantization.
\textit{Physics Letters B}, 102(1), 27--31.

\bibitem{HenneauxTeitelboim1992}
Henneaux, M., and Teitelboim, C. (1992).
\textit{Quantization of Gauge Systems}.
Princeton University Press.

\bibitem{AlexandrovKontsevichSchwarzZaboronsky1997}
Alexandrov, M., Kontsevich, M., Schwarz, A., and Zaboronsky, O. (1997).
The geometry of the master equation and topological quantum field theory.
\textit{International Journal of Modern Physics A}, 12(07), 1405--1429.

\bibitem{Evans2010}
Evans, L. C. (2010).
\textit{Partial Differential Equations}.
American Mathematical Society.

\bibitem{Taylor1996}
Taylor, M. E. (1996).
\textit{Partial Differential Equations I--III}.
Springer.

\bibitem{Zeidler1990}
Zeidler, E. (1990).
\textit{Nonlinear Functional Analysis and Its Applications}.
Springer.

\bibitem{Turing1952}
Turing, A. M. (1952).
The chemical basis of morphogenesis.
\textit{Philosophical Transactions of the Royal Society B}, 237, 37--72.

\bibitem{CrossHohenberg1993}
Cross, M. C., and Hohenberg, P. C. (1993).
Pattern formation outside of equilibrium.
\textit{Reviews of Modern Physics}, 65(3), 851--1112.

\bibitem{Ising1925}
Ising, E. (1925).
Contribution to the theory of ferromagnetism.
\textit{Zeitschrift für Physik}, 31, 253--258.

\bibitem{Onsager1944}
Onsager, L. (1944).
Crystal statistics. I. A two-dimensional model with an order-disorder transition.
\textit{Physical Review}, 65, 117--149.

\bibitem{Strogatz1994}
Strogatz, S. H. (1994).
\textit{Nonlinear Dynamics and Chaos}.
Westview Press.

\bibitem{Gilmore1993}
Gilmore, R. (1993).
\textit{Catastrophe Theory for Scientists and Engineers}.
Dover.

\bibitem{Kolmogorov1965}
Kolmogorov, A. N. (1965).
Three approaches to the quantitative definition of information.
\textit{Problems of Information Transmission}, 1, 1--7.

\bibitem{Shannon1948}
Shannon, C. E. (1948).
A mathematical theory of communication.
\textit{Bell System Technical Journal}, 27, 379--423.

\bibitem{Weinberg1995}
Weinberg, S. (1995).
\textit{The Quantum Theory of Fields, Volume I}.
Cambridge University Press.

\bibitem{BirrellDavies1982}
Birrell, N. D., and Davies, P. C. W. (1982).
\textit{Quantum Fields in Curved Space}.
Cambridge University Press.

\bibitem{Kadanoff2000}
Kadanoff, L. P. (2000).
\textit{Statistical Physics: Statics, Dynamics and Renormalization}.
World Scientific.

\bibitem{BratteliRobinson1987}
Bratteli, O., and Robinson, D. W. (1987).
\textit{Operator Algebras and Quantum Statistical Mechanics}.
Springer.

\bibitem{MacLane1998}
Mac Lane, S. (1998).
\textit{Categories for the Working Mathematician}.
Springer.

\bibitem{May1999}
May, J. P. (1999).
\textit{A Concise Course in Algebraic Topology}.
University of Chicago Press.

\bibitem{GelfandFomin2000}
Gelfand, I. M., and Fomin, S. V. (2000).
\textit{Calculus of Variations}.
Dover.

\bibitem{Smale1967}
Smale, S. (1967).
Differentiable dynamical systems.
\textit{Bulletin of the American Mathematical Society}, 73, 747--817.

\bibitem{Hawking1975}
Hawking, S. W. (1975).
Particle creation by black holes.
\textit{Communications in Mathematical Physics}, 43, 199--220.

\bibitem{Penrose2004}
Penrose, R. (2004).
\textit{The Road to Reality}.
Jonathan Cape.

\end{thebibliography}

\end{document}

